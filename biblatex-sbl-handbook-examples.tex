\DocumentMetadata{lang=en}
\documentclass[a4paper]{article}

\usepackage{microtype}
\usepackage{parskip}
\usepackage{xcolor}
\usepackage{fancyvrb,fvextra}
\usepackage{enverb}
\usepackage{tocloft}
\setlength{\cftbeforesecskip}{2pt plus 0.5pt}
\setcounter{tocdepth}{1}
\RecustomVerbatimEnvironment{verbatim}{Verbatim}{backgroundcolor=black!15,fontsize=\small}
\setcounter{secnumdepth}{0}
\usepackage[american, bidi=basic]{babel}
\babelprovide[import, onchar=ids]{polytonicgreek}
\babelprovide[import, onchar=ids]{russian}
\babelprovide[import, onchar=ids fonts]{armenian}
\babelprovide[import, onchar=ids fonts]{hebrew}
\babelprovide[import, onchar=ids fonts]{syriac}
\babelfont{rm}{Brill}
\babelfont{tt}[Scale=MatchLowercase]{DejaVu Sans Mono}
\babelfont[armenian]{rm}[Scale=0.85]{Noto Serif Armenian}
\babelfont[hebrew]{rm}[Scale=MatchLowercase, Contextuals=Alternate]{SBL Hebrew}
\babelfont[syriac]{rm}[Scale=MatchLowercase]{Estrangelo Edessa}
\babelfont[hebrew]{tt}[Scale=MatchLowercase]{Miriam Libre Medium}
\babelfont[syriac]{tt}[Scale=0.85]{Noto Sans Syriac}
\usepackage{csquotes}
\usepackage[colorlinks]{hyperref}
\usepackage[style=sbl, refsection=subsection+, locallabelwidth, citetracker=true]{biblatex}
\addbibresource{biblatex-sbl.bib}

\makeatletter

\colorlet{cmdcolour}{black!65}

\newcommand*{\pkg}[1]{\textsf{#1}}

\ExplSyntaxOn
% Thanks to https://tex.stackexchange.com/a/44444/87678
\cs_new_protected:Npn \sblhsblog_verb_print:n #1
  {
    \tl_set:Nn \l_tmpa_tl {#1}
    \regex_replace_all:nnN { . } { \c{string} \0 } \l_tmpa_tl
    \tl_set:Nx \l_tmpb_tl { \l_tmpa_tl }
    \tl_use:N \l_tmpb_tl
  }
\cs_set_eq:NN \verbprint \sblhsblog_verb_print:n
\ExplSyntaxOff

% hack to include a space after a control sequence that would normally be
% removed by \verbprint above
\usepackage{newunicodechar}
\newunicodechar{ }{\relax}% this is U+00A0

% redefine hyperlink anchors to be independent of \iffootnote
\DeclareFieldFormat{bibhyperlink}{%
  \bibhyperlink{\cbx@resetcount:\thefield{entrykey}}{#1}}
\DeclareFieldFormat{bibhypertarget}{%
  \ifboolexpr{
    not test {\iffieldundef{crossref}}
    and
    not test {\ifcrossrefseen}
  }
    {\bibhypertarget{\cbx@resetcount:\thefield{crossref}}{}}
    {}%
  \bibhypertarget{\cbx@resetcount:\thefield{entrykey}}{#1}}
\DeclareFieldFormat{shorttitlelink}{%
  \bibhyperlink{\cbx@resetcount:\thefield{entrykey}}{#1}}

% Redefine \smartcite so autocite prints inline with a period.
\DeclareCiteCommand{\smartcite}[\iffootnote\mkbibparens\bibfootnotewrapper]
  {\usebibmacro{prenote}}
  {\usebibmacro{cite}}
  {\multicitedelim}
  {\usebibmacro{cite:postnote}}

\DeclareCiteCommand*{\smartcite}[\iffootnote\mkbibparens\bibfootnotewrapper]
  {\usebibmacro{prenote}}
  {\usebibmacro{cite:star}%
   \usebibmacro{cite}}
  {\multicitedelim}
  {\usebibmacro{cite:postnote}}

\NewDocumentCommand{\examplenobreak}{}{%
  \par
  \@afterheading
}

\NewDocumentCommand{\exampleciteauthor}{m}{%
  \par
  \textcolor{cmdcolour}{\texttt{\textbackslash citeauthor\{#1\}}}\par
  \@afterheading
  \citeauthor{#1}%
}

% \examplecite(<footnote number>){<citetype>}[<prenote>][<postnote>]{entryid}
% \examplecite*(<footnote number>){<citetype>}[<prenote>][<postnote>]{entryid}
\NewDocumentCommand{\examplecite}{sO{auto}d()oom}{%
  \par
  \textcolor{cmdcolour}{%
    \texttt{%
      \textbackslash #2cite%
      \IfBooleanT{#1}{*}%
      \IfNoValueF{#4}{[\verbprint{#4}]}%
      \IfNoValueF{#5}{[\verbprint{#5}]}%
      \{#6\}}}\par
  \@afterheading
  \IfNoValueF{#3}{\quad #3.\space\strut}%
  \IfBooleanTF{#1}
    {\printexamplecite*{#2cite}{#4}{#5}{#6}}
    {\printexamplecite{#2cite}{#4}{#5}{#6}}}

% \examplevolcite(<footnote number>){<citetype>}[<prenote>]{<volume>}[<postnote>]{entryid}
% \examplevolcite*(<footnote number>){<citetype>}[<prenote>]{<volume>}[<postnote>]{entryid}
\NewDocumentCommand{\examplevolcite}{sO{a}d()omom}{%
  \par
  \textcolor{cmdcolour}{%
    \texttt{%
      \textbackslash #2volcite%
      \IfBooleanT{#1}{*}%
      \IfNoValueF{#4}{[\verbprint{#4}]}%
      \{#5\}%
      \IfNoValueF{#6}{[\verbprint{#6}]}%
      \{#7\}}}\par
  \@afterheading
  \IfNoValueF{#3}{\quad #3.\space\strut}%
  \IfBooleanTF{#1}
    {\printexamplevolcite*{#2volcite}{#4}{#5}{#6}{#7}}
    {\printexamplevolcite{#2volcite}{#4}{#5}{#6}{#7}}}

\NewDocumentCommand{\printexamplecite}{smmmm}{%
  \IfNoValueTF{#4}
    {\IfNoValueTF{#3}
       {\IfBooleanTF{#1}
          {\csname #2\endcsname*{#5}}
          {\csname #2\endcsname{#5}}}
       {\IfBooleanTF{#1}
          {\csname #2\endcsname*[#3]{#5}}
          {\csname #2\endcsname[#3]{#5}}}}
    {\IfNoValueTF{#3}
       {\IfBooleanTF{#1}
          {\csname #2\endcsname*[#3][]{#5}}
          {\csname #2\endcsname[#3][]{#5}}}
       {\IfBooleanTF{#1}
          {\csname #2\endcsname*[#3][#4]{#5}}
          {\csname #2\endcsname[#3][#4]{#5}}}}}

\NewDocumentCommand{\printexamplevolcite}{smmmmm}{%
  \IfNoValueTF{#3}
    {\IfNoValueTF{#5}
       {\IfBooleanTF{#1}
          {\csname #2\endcsname*{#4}{#6}}
          {\csname #2\endcsname{#4}{#6}}}
       {\IfBooleanTF{#1}
          {\csname #2\endcsname*{#4}[#5]{#6}}
          {\csname #2\endcsname{#4}[#5]{#6}}}}
    {\IfNoValueTF{#5}
       {\IfBooleanTF{#1}
          {\csname #2\endcsname*[#3]{#4}{#6}}
          {\csname #2\endcsname[#3]{#4}{#6}}}
       {\IfBooleanTF{#1}
          {\csname #2\endcsname*[#3]{#4}[#5]{#6}}
          {\csname #2\endcsname[#3]{#4}[#5]{#6}}}}}

\NewDocumentEnvironment{verbcite}{}{%
  \enverb{}%
}{%
  \par
  \color{cmdcolour}%
  \enverbListing{Verbatim}{}%
  \par
  \@afterheading
  \normalcolor
  \enverbExecute
}

\NewDocumentEnvironment{fverbcite}{m}{%
  \enverb{}%
}{%
  \par
  \color{cmdcolour}%
  \enverbListing{Verbatim}{}%
  \par
  \@afterheading
  \normalcolor
  \renewcommand\footnote[1]{\toggletrue{blx@footnote}##1}%
  \quad #1.\space\strut\enverbExecute
}

\NewDocumentEnvironment{verbtext}{}{%
  \enverb{}%
}{%
  \par
  \color{cmdcolour}%
  \enverbListing{Verbatim}{}%
  \normalcolor
  \par
}

\NewDocumentCommand{\exampleabbreviations}{}{%
  \par
  \textcolor{cmdcolour}{\texttt{\textbackslash printbiblist\{abbreviations\}}}
  \@afterheading
  \printbiblist[heading=none]{abbreviations}}

\NewDocumentCommand{\exampleancientsources}{}{%
  \par
  \textcolor{cmdcolour}{%
    \texttt{\textbackslash printbiblist[heading=subbibliography, title=Ancient Sources,\\
      \strut\quad type=ancienttext]\{abbreviations\}}}
  \@afterheading
  \printbiblist[heading=subbibliography, title=Ancient Sources,
    type=ancienttext]{abbreviations}}

\NewDocumentCommand{\examplesecondarysources}{}{%
  \par
  \textcolor{cmdcolour}{%
    \texttt{\textbackslash printbiblist[heading=subbibliography, title=Secondary Sources,\\
      \strut\quad nottype=ancienttext]\{abbreviations\}}}
  \@afterheading
  \printbiblist[heading=subbibliography, title=Secondary Sources,
    nottype=ancienttext, nottype=abbreviation]{abbreviations}}

\NewDocumentCommand{\examplesigla}{}{%
  \par
  \textcolor{cmdcolour}{%
    \texttt{\textbackslash printbiblist[heading=subbibliography, title=Sigla
      and Grammatical \\
      \strut\quad Abbreviations, type=abbreviation]\{abbreviations\}}}
  \@afterheading
  \printbiblist[heading=subbibliography, title=Sigla and Grammatical
    Abbreviations, type=abbreviation]{abbreviations}}

\NewDocumentCommand{\examplebibliography}{}{%
  \par
  \textcolor{cmdcolour}{\texttt{\textbackslash printbibliography}}
  \@afterheading
  \printbibliography[heading=none]}

\NewDocumentCommand{\examplereferences}{m}{%
  \subsection*{References}
  \url{#1}}
\makeatother

\setcounter{tocdepth}{2}

\begin{document}
\title{SBL Handbook of Style}
\author{Second Edition}
\date{SBL Press}
\maketitle

\begin{abstract}
  This document contains the note and bibliography examples from the \emph{SBL
  Handbook of Style} Second Edition §§6.2–4. Each section shows what the
  various bib entries should look like and how to cite them.
\end{abstract}

\tableofcontents

\section{General Examples: Books}

\subsection{A Book by a Single Author}

\begin{verbatim}
@book{talbert:1992,
  author = {Talbert, Charles H.},
  title = {Reading John},
  subtitle = {A Literary and Theological Commentary on the Fourth Gospel and the
              Johannine Epistles},
  location = {New York},
  publisher = {Crossroad},
  date = {1992}
}
\end{verbatim}

\examplecite(15)[127]{talbert:1992}
\examplecite(19)[22]{talbert:1992}
\examplebibliography

\subsection{A Book by Two or Three Authors}

\begin{verbatim}
@book{robinson+koester:1971,
  author = {Robinson, James M. and Koester, Helmut},
  title = {Trajectories through Early Christianity},
  location = {Philadelphia},
  publisher = {Fortress},
  date = {1971}
}
\end{verbatim}

\examplecite(4)[237]{robinson+koester:1971}
\examplecite(12)[23]{robinson+koester:1971}
\examplebibliography

\subsection{A Book by More Than Three Authors}

\begin{verbatim}
@book{scott+etal:1993,
  author = {Scott, Bernard Brandon and Dean, Margaret and Sparks, Kristen and LaZar,
            Frances},
  title = {Reading New Testament Greek},
  location = {Peabody, MA},
  publisher = {Hendrickson},
  date = {1993}
}
\end{verbatim}

\examplecite(7)[53]{scott+etal:1993}
\examplecite(9)[42]{scott+etal:1993}
\examplebibliography

\subsection{A Translated Volume}

\begin{verbatim}
@book{egger:1996,
  author = {Egger, Wilhelm},
  title = {How to Read the New Testament},
  subtitle = {An Introduction to Linguistic and Historical-Critical Methodology},
  shorttitle = {How to Read},
  translator = {Heinegg, Peter},
  location = {Peabody, MA},
  publisher = {Hendrickson},
  date = {1996}
}
\end{verbatim}

\examplecite(14)[28]{egger:1996}
\examplecite(18)[291]{egger:1996}
\examplebibliography

\subsection{The Full History of a Translated Volume}

\begin{verbatim}
@book{wellhausen:1883,
  author = {Wellhausen, Julius},
  title = {Prolegomena zur Geschichte Israels},
  edition = {2},
  location = {Berlin},
  publisher = {Reimer},
  date = {1883},
  langid = {german}
}

@book{wellhausen:1885,
  author = {Wellhausen, Julius},
  title = {Prolegomena to the History of Israel},
  translator = {Black, J. Sutherland and Enzies, A.},
  preface = {Smith, W. Robertson},
  location = {Edinburgh},
  publisher = {Black},
  related = {wellhausen:1883},
  relatedtype = {translationof},
  date = {1885}
}

@book{wellhausen:1957,
  author = {Wellhausen, Julius},
  title = {Prolegomena to the History of Ancient Israel},
  location = {New York},
  publisher = {Meridian Books},
  related = {wellhausen:1885},
  relatedtype = {reprintof},
  date = {1957}
}
\end{verbatim}

\examplecite(3)[296]{wellhausen:1957}
\examplebibliography

\subsubsection{Notes}

The format of \citeshortauthor{wellhausen:1957} in the example bibliography
entry is incorrect in the handbook. See
\url{https://sblhs2.com/2016/06/02/first-last-name-order/}.

\subsection{A Book with One Editor}

\begin{verbatim}
@book{tigay:1985,
  editor = {Tigay, Jeffrey H.},
  title = {Empirical Models for Biblical Criticism},
  shorttitle = {Empiracle Models},
  location = {Philadelphia},
  publisher = {University of Pennsylvania Press},
  date = {1985}
}
\end{verbatim}

\examplecite(5)[35]{tigay:1985}
\examplecite(9)[38]{tigay:1985}
\examplebibliography

\subsection{A Book with Two or Three Editors}

\begin{verbatim}
@book{kaltner+mckenzie:2002,
  editor = {Kaltner, John and McKenzie, Steven L.},
  title = {Beyond Babel: A Handbook for Biblical Hebrew and Related Languages},
  series = {Resources for Biblical Study},
  shortseries = {RBS},
  number = {42},
  location = {Atlanta},
  publisher = {Society of Biblical Literature},
  date = {2002}
}
\end{verbatim}

\examplecite(44)[xii]{kaltner+mckenzie:2002}
\examplecite(47)[viii]{kaltner+mckenzie:2002}
\exampleabbreviations
\examplebibliography

\subsubsection{Notes}

The handbook leaves the short title out in the subsequent citation. But in
light of other examples and the post at
\url{https://sblhs2.com/2017/01/05/subsequent-bibliographic-references/} this
seems likely an error. \pkg{biblatex-sbl} includes the short title.

\subsection{A Book with Four or More Editors}

\begin{verbatim}
@book{oates+etal:2001,
  editor = {Oates, John F. and Willis, William H. and Bagnall, Roger S. and Worp,
            Klass A.},
  title = {Checklist of Editions of Greek and Latin Papyri, Ostraca, and Tablets},
  edition = {5},
  series = {Bulletin of the American Society of Papyrologists, Supplements},
  shortseries = {BASPSup},
  number = {9},
  location = {Oakville, CT},
  publisher = {American Society of Papyrologists},
  date = {2001}
}
\end{verbatim}

\examplecite(4)[10]{oates+etal:2001}
\exampleabbreviations
\examplebibliography

\subsection{A Book with Both Author and Editor}

\begin{verbatim}
@book{schillebeeckx:1986,
  author = {Schillebeeckx, Edward},
  title = {The Schillebeeckx Reader},
  editor = {Schreiter, Robert J.},
  location = {Edinburgh},
  publisher = {T\&T Clark},
  date = {1986}
}
\end{verbatim}

\examplecite(45)[20]{schillebeeckx:1986}
\examplebibliography

\subsection{A Book with Author, Editor, and Translator}

\begin{verbatim}
@book{blass+debrunner:1982,
  author = {Blass, Friedrich and Debrunner, Albert},
  title = {Grammatica del greco del Nuovo Testamento},
  editor = {Rehkopf, Friedrich},
  translator = {Pisi, Giordana},
  location = {Brescia},
  publisher = {Paideia},
  date = {1982},
  langid = {italian}
}
\end{verbatim}

\examplecite(3){blass+debrunner:1982}
\examplebibliography

\subsection{A Title in a Modern Work Citing Words in a Non-Latin Alphabet}

\begin{verbatim}
@article{irvine:2014,
  author = {Irvine, Stuart A.},
  title = {Idols \mkbibbrackets{\mkbibemph{ktbwnm}}},
  subtitle = {A Note on Hosea 13:2a},
  journaltitle = {Journal of Biblical Literature},
  shortjournal = {JBL},
  volume = {133},
  date = {2014},
  pages = {509-517}
}
\end{verbatim}

\examplecite(34){irvine:2014}
\exampleabbreviations
\examplebibliography

\subsection{An Article in an Edited Volume}

\begin{verbatim}
@collection{kraft+nickelsburg:1986,
  editor = {Kraft, Robert A. and Nickelsburg, George W. E.},
  title = {Early Judaism and Its Modern Interpreters},
  shorttitle = {Early Judaism},
  location = {Philadelphia and Atlanta},
  publisher = {Fortress and Scholars Press},
  date = {1986}
}

@incollection{attridge:1986,
  author = {Attridge, Harold W.},
  title = {Jewish Historiography},
  pages = {311-343},
  crossref = {kraft+nickelsburg:1986}
}

@incollection{collins:1986,
  author = {Collins, John J.},
  title = {The Testamentary Literature in Recent Scholarship},
  pages = {268-286},
  crossref = {kraft+nickelsburg:1986}
}

@article{knoppers:1995,
  author = {Knoppers, Gary N.},
  title = {Prayer and Propaganda: The Dedication of Solomon's Temple and the
           Deuteronomist's Program},
  journaltitle = {Catholic Biblical Quarterly},
  shortjournal = {CBQ},
  volume = {57},
  date = {1995},
  pages = {229-254}
}

@incollection{knoppers:2000,
  author = {Knoppers, Gary N.},
  title = {Prayer and Propaganda: The Dedication of Solomon's Temple and the
           Deuteronomist's Program},
  booktitle = {Reconsidering Israel and Judah: Recent Studies on the Deuteronomistic
               History},
  editor = {Knoppers, Gary N. and McConville, J. Gordon},
  location = {Winona Lake, IN},
  publisher = {Eisenbrauns},
  date = {2000},
  pages = {370-396}
}
\end{verbatim}

\examplecite(3){attridge:1986}
\examplecite(6)[314-317]{attridge:1986}
\examplecite(4){knoppers:1995}
\examplecite(5){knoppers:2000}
\examplecite(6){knoppers:2000}
\begin{verbcite}
  \citereset
\end{verbcite}
\examplecite(8){collins:1986}
\examplecite(9){attridge:1986}
\exampleabbreviations
\examplebibliography

\subsection{An Article in a Festschrift}

\begin{verbatim}
@incollection{vanseters:1995,
  author = {Van Seters, John},
  title = {The Theology of the Yahwist},
  subtitle = {A Preliminary Sketch},
  pages = {219-228},
  booktitle = {\mkbibquote{Wer ist wie du, Herr, unter den Göttern?}},
  booksubtitle = {Studien zur Theologie und Religionsgeschichte Israels für Otto
                  Kaiser zum 70.~Geburtstag},
  editor = {Kottsieper, Ingo and others},
  location = {Göttingen},
  publisher = {Vandenhoeck \& Ruprecht},
  date = {1995}
}
\end{verbatim}

\examplecite(8){vanseters:1995}
\examplecite(17)[222]{vanseters:1995}
\examplebibliography

\subsection{An Introduction, Preface, or Foreword Written by Someone Other Than the Author}

\begin{verbatim}
@suppbook{boers:1996,
  author = {Boers, Hendrikus},
  title = {introduction},
  booktitle = {How to Read the New Testament},
  booksubtitle = {An Introduction to Linguistic and Historical-Critical Methodology},
  bookauthor = {Egger, Wilhelm},
  translator = {Heinegg, Peter},
  location = {Peabody, MA},
  publisher = {Hendrickson},
  date = {1996}
}
\end{verbatim}

\examplecite(2)[xi-xxi]{boers:1996}
\examplecite(6)[xi-xx]{boers:1996}
\examplebibliography

\subsection{Multiple Publishers for a Single Book}

\begin{verbatim}
@book{gerhardsson:1961,
  author = {Gerhardsson, Birger},
  title = {Memory and Manuscript},
  subtitle = {Oral Tradition and Written Transmission in Rabbinic Judaism and Early
              Christianity},
  series = {Acta Seminarii Neotestamentici Upsaliensis},
  shortseries = {ASNU},
  number = {22},
  location = {Lund and Copenhagen},
  publisher = {Gleerup and Munksgaard},
  date = {1961}
}
\end{verbatim}

\begin{verbcite}
  \nocite{gerhardsson:1961}
\end{verbcite}
\exampleabbreviations
\examplebibliography

\subsubsection{Notes}

The format of \citeshortauthor{gerhardsson:1961} in the example bibliography
entry is incorrect in the handbook. See
\url{https://sblhs2.com/2016/06/02/first-last-name-order/}.

\subsection{A Revised Edition}

\begin{verbatim}
@book{pritchard:1969,
  editor = {Pritchard, James B.},
  title = {Ancient Near Eastern Texts Relating to the Old Testament},
  edition = {3},
  location = {Princeton},
  publisher = {Princeton University Press},
  date = {1969}
}

@book{blenkinsopp:1996,
  author = {Blenkinsopp, Joseph},
  title = {A History of Prophecy in Israel},
  edition = {rev.\@ and enl\@ ed.\isdot},
  location = {Louisville},
  publisher = {Westminster John Knox},
  date = {1996}
}
\end{verbatim}

\examplecite(87)[xxi]{pritchard:1969}
\examplecite(56)[81]{blenkinsopp:1996}
\examplebibliography

\subsection{Reprint of a Recent Title}

\begin{verbatim}
@book{vanseters:1997,
  author = {Van Seters, John},
  title = {In Search of History},
  subtitle = {Historiography in the Ancient World and the Origins of Biblical
              History},
  origlocation = {New Haven},
  origpublisher = {Yale University Press},
  origdate = {1983},
  location = {Winona Lake, IN},
  publisher = {Eisenbrauns},
  date = {1997}
}
\end{verbatim}

\examplecite(5)[35]{vanseters:1997}
\examplebibliography

\subsection{Reprint of a Title in the Public Domain}

\begin{verbatim}
@book{deissmann:1995,
  author = {Deissmann, Gustav Adolf},
  title = {Light from the Ancient East},
  subtitle = {The New Testament Illustrated by Recently Discovered Texts of the
              Graeco-Roman World},
  translator = {Strachan, Lionel R. M.},
  origdate = {1927},
  location = {Peabody, MA},
  publisher = {Hendrickson},
  date = {1995}
}
\end{verbatim}

\examplecite(5)[55]{deissmann:1995}
\examplebibliography

\subsubsection{Notes}

Placing the translator within parenthesis is likely carried over from the
first edition of the \emph{SBLHS}. The second edition leaves only the basic
facts of publication (city, publisher, date) inside parenthesis. See
\url{https://sblhs2.com/2016/10/04/separating-publication-information/}.

\subsection{A Forthcoming Book}

\begin{verbatim}
@book{harrison+welborn:forthcoming,
  editor = {Harrison, James R. and Welborn, L. L.},
  title = {The First Urban Churches 2},
  subtitle = {Roman Corinth},
  shorttitle = {Roman Corinth},
  series = {Writings from the Greco-Roman World Supplement Series},
  shortseries = {WGRWSup},
  location = {Atlanta},
  publisher = {SBL Press},
  pubstate = {forthcoming}
}
\end{verbatim}

\examplecite(9){harrison+welborn:forthcoming}
\examplecite(12)[201]{harrison+welborn:forthcoming}
\exampleabbreviations
\examplebibliography

\subsection{A Multivolume Work}

\begin{verbatim}
@mvbook{harnack:1896-1905,
  author = {Harnack, Adolf},
  title = {History of Dogma},
  translator = {Buchanan, Neil},
  origlanguage = {from the 3rd German ed.\isdot},
  volumes = {7},
  location = {Boston},
  publisher = {Little, Brown},
  date = {1896/1905}
}
\end{verbatim}

\examplecite(5){harnack:1896-1905}
\examplevolcite(9){2}[126]{harnack:1896-1905}
\examplebibliography

\subsection{A Titled Volume in a Multivolume Work}

\begin{verbatim}
@collection{winter+clarke:1993,
  editor = {Winter, Bruce W. and Clarke, Andrew D.},
  title = {The Book of Acts in Its Ancient Literary Setting},
  shorttitle = {Book of Acts},
  volume = {1},
  maintitle = {The Book of Acts in Its First Century Setting},
  maineditor = {Winter, Bruce W.},
  location = {Grand Rapids},
  publisher = {Eerdmans},
  date = {1993}
}
\end{verbatim}

\examplecite(5)[25]{winter+clarke:1993}
\examplecite(16)[25]{winter+clarke:1993}
\examplebibliography

\subsection{A Chapter within a Multivolume Work}

\begin{verbatim}
@incollection{mason:1996,
  author = {Mason, Steve},
  title = {Josephus on Canon and Scriptures},
  pages = {217-235},
  volume = {1},
  part = {1},
  maintitle = {Hebrew Bible\slash Old Testament},
  mainsubtitle = {The History of Its Interpretation},
  editor = {Sæbø, Magne},
  location = {Göttingen},
  publisher = {Vandenhoeck \& Ruprecht},
  date = {1996}
}
\end{verbatim}

\examplecite(24){mason:1996}
\examplecite(28)[224]{mason:1996}
\examplebibliography

\subsubsection{Notes}

The \emph{SBLHS} contains an error with the pages in the first citation. It
should be 217–35 not 217–335.

\subsection{A Chapter within a Titled Volume in a Multivolume Work}

\begin{verbatim}
@incollection{peterson:1993,
  author = {Peterson, David},
  title = {The Motif of Fulfilment and the Purpose of Luke-Acts},
  shorttitle = {Motif of Fulfilment},
  pages = {83-104},
  booktitle = {The Book of Acts in Its Ancient Literary Setting},
  bookeditor = {Winter, Bruce W. and Clarke, Andrew D.},
  volume = {1},
  maintitle = {The Book of Acts in Its First Century Setting},
  editor = {Winter, Bruce W.},
  location = {Grand Rapids},
  publisher = {Eerdmans},
  date = {1993}
}
\end{verbatim}

\examplecite(66){peterson:1993}
\examplecite(78)[92]{peterson:1993}
\examplebibliography

\subsection{A Work in a Series}

\begin{verbatim}
@book{hofius:1989,
  author = {Hofius, Otfried},
  title = {Paulusstudien},
  series = {Wissenschaftliche Untersuchungen zum Neuen Testament},
  shortseries = {WUNT},
  number = {51},
  location = {Tübingen},
  publisher = {Mohr Siebeck},
  date = {1989},
  langid = {german}
}

@book{jeremias:1967,
  author = {Jeremias, Joachim},
  title = {The Prayers of Jesus},
  shorttitle = {Prayers},
  series = {Studies in Biblical Theology},
  shortseries = {SBT},
  number = {2/6},
  location = {Naperville, IL},
  publisher = {Allenson},
  date = {1967}
}
\end{verbatim}

\examplecite(12)[122]{hofius:1989}
\examplecite(14)[124]{hofius:1989}
\examplecite(23)[123-127]{jeremias:1967}
\examplecite(32)[126]{jeremias:1967}
\exampleabbreviations
\examplebibliography

\subsection{Electronic Book}

\begin{verbatim}
@book{reventlow:2009,
  author = {Reventlow, Henning Graf},
  title = {From the Old Testament to Origen},
  volume = {1},
  maintitle = {History of Biblical Interpretation},
  translator = {Perdue, Leo G.},
  location = {Atlanta},
  publisher = {Society of Biblical Literature},
  date = {2009},
  eprint = {Nook},
  eprinttype = {ebook},
  pagination = {chapter}
}

@book{wright:2014,
  author = {Wright, Jacob L.},
  title = {David, King of Israel, and Caleb in Biblical Memory},
  shorttitle = {David, King of Israel},
  location = {Cambridge},
  publisher = {Cambridge University Press},
  date = {2014},
  eprint = {Kindle},
  eprinttype = {ebook},
  pagination = {chapter}
}

@book{killebrew+steiner:2014,
  editor = {Killebrew, Ann E. and Steiner, Margreet},
  title = {The Oxford Handbook of the Archaeology of the Levant},
  subtitle = {c.~8000--332 BCE},
  shorttitle = {Archaeology of the Levant},
  location = {Oxford},
  publisher = {Oxford University Press},
  date = {2014},
  doi = {10.1093/oxfordhb/9780199212972.001.0001}
}

@book{kaufman:1974,
  author = {Kaufman, Stephen},
  title = {The Akkadian Influences on Aramaic},
  series = {Assyriological Studies},
  shortseries = {AS},
  number = {19},
  location = {Chicago},
  publisher = {The Oriental Institute of the University of Chicago},
  date = {1974},
  url = {http://oi.uchicago.edu/pdf/as19.pdf}
}
\end{verbatim}

\examplecite(14)[1.3]{reventlow:2009}
\examplecite(18)[1.3]{reventlow:2009}
\examplecite(3)[\pnfmt{3}, \mkbibquote{Introducing David}]{wright:2014}
\examplecite(21)[\pnfmt{5}, \mkbibquote{Evidence from Qumran}]{wright:2014}
\examplecite(53){killebrew+steiner:2014}
\examplecite(55){killebrew+steiner:2014}
\examplecite(29){kaufman:1974}
\examplecite(32)[123]{kaufman:1974}
\exampleabbreviations
\examplebibliography

\subsubsection{Notes}

\begin{itemize}
  \item The period before ``Vol.~1'' in the first citation and the lack of
    abbreviation in in the bibliography of \citeshortauthor{reventlow:2009} is
    inconsistent with what the \emph{SBLHS} has done elsewhere and is assumed
    to be an error.
  \item SBL now prefers to use a full URL for the DOI. See
    \url{https://sblhs2.com/2018/05/03/electronic-journals-with-individually-paginated-articles/}.
\end{itemize}

\section{General Examples: Journal Articles, Reviews, and Dissertations}

\subsection{A Journal Article}

\begin{verbatim}
@article{leyerle:1993,
  author = {Leyerle, Blake},
  title = {John Chrysostom on the Gaze},
  shorttitle = {Chrysostom},
  journaltitle = {Journal of Early Christian Studies},
  shortjournal = {JECS},
  volume = {1},
  date = {1993},
  pages = {159-174}
}
\end{verbatim}

\examplecite(7){leyerle:1993}
\examplecite(23)[161]{leyerle:1993}
\exampleabbreviations
\examplebibliography

\subsection{A Journal Article with Multiple Page Locations and Multiple Volumes}

\begin{verbatim}
@article{wildberger:1965,
  author = {Wildberger, Hans},
  title = {Das Abbild Gottes: Gen 1:26--30},
  journaltitle = {Theologische Zeitschrift},
  shortjournal = {TZ},
  volume = {21},
  date = {1965},
  pages = {245-259, 481-501},
  langid = {german}
}

@article{wellhausen:1876-1877,
  author = {Wellhausen, Julius},
  title = {Die Composition des Hexateuchs},
  journaltitle = {Jahrbuch für deutsche Theologie},
  shortjournal = {JDT},
  related = {wellhausen:1876, wellhausen:1877},
  relatedtype = {multivolarticle},
  langid = {german}
}

@article{wellhausen:1876,
  volume = {21},
  date = {1876},
  pages = {392-450}
}

@article{wellhausen:1877,
  volume = {22},
  date = {1877},
  pages = {407-479}
}
\end{verbatim}

\examplecite(21){wildberger:1965}
\examplecite(24){wellhausen:1876-1877}
\exampleabbreviations
\examplebibliography

\subsection{A Journal Article Republished in a Collected Volume}

\begin{verbatim}
@article{freedman:1977,
  author = {Freedman, David Noel},
  title = {Pottery, Poetry, and Prophecy},
  subtitle = {An Essay on Biblical Poetry},
  journaltitle = {Journal of Biblical Literature},
  shortjournal = {JBL},
  volume = {96},
  date = {1977},
  pages = {5-26}
}

@incollection{freedman:1980,
  author = {Freedman, David Noel},
  title = {Pottery, Poetry, and Prophecy},
  subtitle = {An Essay on Biblical Poetry},
  booktitle = {Pottery, Poetry, and Prophecy},
  booksubtitle = {Studies in Early Hebrew Poetry},
  location = {Winona Lake, IN},
  publisher = {Eisenbrauns},
  date = {1980},
  pages = {1-22}
}
\end{verbatim}

\examplecite(20)[20]{freedman:1977}
\examplecite(20)[14]{freedman:1980}
\exampleabbreviations
\examplebibliography

\subsection{A Book Review}

\begin{verbatim}
@review{teeple:1966,
  author = {Teeple, Howard M.},
  revdauthor = {Robert, André and Feuillet, André},
  revdtitle = {Introduction to the New Testament},
  journaltitle = {Journal of Bible and Religion},
  shortjournal = {JBR},
  volume = {34},
  date = {1966},
  pages = {368-370}
}

@review{pelikan:1992,
  author = {Pelikan, Jaroslav},
  title = {The Things That You're Liable to Read in the Bible},
  revdeditor = {Freedman, David Noel},
  revdtitle = {The Anchor Bible Dictionary},
  journaltitle = {New York Times Review of Books},
  date = {1992-12-20},
  pages = {3}
}

@article{petersen:1988,
  author = {Petersen, David L.},
  title = {Hebrew Bible Textbooks},
  subtitle = {A Review Article},
  journaltitle = {Critical Review of Books in Religion},
  shortjournal = {CRBR},
  volume = {1},
  date = {1988},
  pages = {1-18}
}
\end{verbatim}

\examplecite(8){teeple:1966}
\examplecite(21)[369]{teeple:1966}
\examplecite(9){pelikan:1992}
\examplecite(7){petersen:1988}
\examplecite(14)[8]{petersen:1988}
\exampleabbreviations
\examplebibliography

\subsection{An Unpublished Dissertation or Thesis}

\begin{verbatim}
@thesis{klosinski:1988,
  author = {Klosinski, Lee E.},
  title = {Meals in Mark},
  type = {phdthesis},
  institution = {The Claremont Graduate School},
  date = {1988}
}
\end{verbatim}

\examplecite(21)[22-44]{klosinski:1988}
\examplecite(26)[23]{klosinski:1988}
\examplebibliography

\subsection{An Article in an Encyclopedia or a Dictionary}

\begin{verbatim}
@mvreference{IDB,
  shorthand = {IDB},
  editor = {Buttrick, George A.},
  title = {The Interpreter’s Dictionary of the Bible},
  volumes = {4},
  location = {New York},
  publisher = {Abingdon},
  date = {1962},
  pagination = {subverbo}
}

@inreference{stendahl:biblicaltheology,
  author = {Stendahl, Krister},
  title = {Biblical Theology, Contemporary},
  shorttitle = {Biblical Theology},
  xref = {IDB},
  volume = {1},
  pages = {418-432}
}
\end{verbatim}

\examplecite(33){stendahl:biblicaltheology}
\examplecite(36)[419]{stendahl:biblicaltheology}
\exampleabbreviations
\examplebibliography

\subsection{An Article in a Lexicon or a Theological Dictionary}

\begin{verbatim}
@mvreference{NIDNTT,
  shorthand = {NIDNTT},
  editor = {Brown, Colin},
  title = {New International Dictionary of New Testament Theology},
  volumes = {4},
  location = {Grand Rapids},
  publisher = {Zondervan},
  date = {1975/1985}
}

@inreference{dahn+liefeld:see+vision+eye,
  author = {Dahn, Karl and Liefeld, Walter L.},
  title = {See, Vision, Eye},
  xref = {NIDNTT},
  volume = {3},
  pages = {511-521}
}

@inreference{dahn:horao,
  author = {Dahn, Karl},
  title = {ὁράω},
  xref = {NIDNTT},
  volume = {3},
  pages = {511-518}
}

@mvreference{TDNT,
  shorthand = {TDNT},
  editor = {Kittel, Gerhard and Friedrich, Gerhard},
  title = {Theological Dictionary of the New Testament},
  translator = {Bromiley, Geoffrey W.},
  volumes = {10},
  location = {Grand Rapids},
  publisher = {Eerdmans},
  date = {1964/1976}
}

@inreference{beyer:diakoneo+diakonia+ktl,
  author = {Beyer, Hermann W.},
  title = {διακονέω, διακονία, κτλ},
  xref = {TDNT},
  volume = {2},
  pages = {81-93}
}

@inreference{beyer:diakoneo,
  author = {Beyer, Hermann W.},
  title = {διακονέω},
  xref = {TDNT},
  volume = {2},
  pages = {81-87}
}

@mvreference{TLNT,
  shorthand = {TLNT},
  author = {Spicq, Ceslas},
  title = {Theological Lexicon of the New Testament},
  editor = {Ernest, James D.},
  translator = {Ernest, James D.},
  volumes = {3},
  location = {Peabody, MA},
  publisher = {Hendrickson},
  date = {1994}
}

@inreference{spicq:atakteo+ataktos+ataktos,
  author = {Spicq, Ceslas},
  title = {ἀτακτέω, ἄτακτος, ἀτάκτως},
  xref = {TLNT},
  volume = {1},
  pages = {223-224}
}

@inreference{spicq:amoibe,
  author = {Spicq, Ceslas},
  title = {ἀμοιβή},
  xref = {TLNT},
  volume = {1},
  pages = {95-96}
}
\end{verbatim}

\examplecite(3){dahn+liefeld:see+vision+eye}
\examplecite(6){beyer:diakoneo+diakonia+ktl}
\examplecite(7){spicq:atakteo+ataktos+ataktos}
\examplecite(143){spicq:amoibe}
\examplecite(23){beyer:diakoneo}
\examplecite(26){dahn:horao}
\examplecite(25)[83]{beyer:diakoneo}
\examplecite(29)[511]{dahn:horao}
\examplecite(147)[95]{spicq:amoibe}
\exampleabbreviations
\examplebibliography

\subsubsection{Notes}

SBL recommendations for citing signed articles from lexicons and theological
dictionaries have changed significantly. See
\url{https://sblhs2.com/2017/04/04/citing-reference-works-3-dictionaries-word/}. 
\begin{itemize}
  \item SBL now prefers to include signed articles in the bibliography.
  \item SBL now prefers to include the title in subsequent citations of signed
    articles.
  \item Dictionaries cited by abbreviation should appear in the list of
    abbreviations, not the bibliography. The format of abbreviations now
    matches the bibliography format rather than placing the title first when
    the abbreviation relates to the title.
\end{itemize}

\subsection{A Paper Presented at a Professional Society}

\begin{verbatim}
@unpublished{niditch:1994,
  author = {Niditch, Susan},
  title = {Oral Culture, and Written Documents},
  shorttitle = {Oral Culture},
  type = {paper},
  eventtitle = {the Annual Meeting of the New England Region of the Society of
                Biblical Literature},
  venue = {Worcester, MA},
  eventdate = {1994-03-25}
}
\end{verbatim}

\examplecite(31)[13-17]{niditch:1994}
\examplecite(35)[14]{niditch:1994}
\examplebibliography

\subsection{An Article in a Magazine}

\begin{verbatim}
@article{saldarini:1998,
  author = {Saldarini, Anthony J.},
  title = {Babatha's Story},
  journaltitle = {Biblical Archaeology Review},
  shortjournal = {BAR},
  volume = {24},
  number = {2},
  date = {1998},
  pages = {23-33, 36-37, 72-74}
}
\end{verbatim}

\examplecite(8){saldarini:1998}
\examplecite(27)[28]{saldarini:1998}
\exampleabbreviations
\examplebibliography

\subsection{An Electronic Journal Article}

\begin{verbatim}
@article{springer:2014,
  author = {Springer, Carl P. E.},
  title = {Of Roosters and \mkbibemph{Repetitio}},
  subtitle = {Ambrose's \mkbibemph{Aeterne rerum conditor}},
  journaltitle = {Vigiliae Christianae},
  shortjournal = {VC},
  volume = {68},
  date = {2014},
  pages = {155-177},
  doi = {10.1163/15700720-12341158}
}

@article{truehart:1996,
  author = {Truehart, Charles},
  title = {Welcome to the Next Church},
  shorttitle = {Next Church},
  url = {http://www.theatlantic.com/atlantic/issues/96aug/nxtchrch/nxtchrch.htm},
  journaltitle = {Atlantic Monthly},
  volume = {278},
  date = {1996-08},
  pages = {37-58}
}

@article{kirk:2007,
  author = {Kirk, Alan},
  title = {Karl Polanyi, Marshall Sahlins, and the Study of Ancient Social Relations},
  shorttitle = {Karl Polanyi},
  journaltitle = {Journal of Biblical Literature},
  shortjournal = {JBL},
  volume = {126},
  date = {2007},
  pages = {182-191},
  doi = {10.2307/27638428},
  url = {http://www.jstor.org/stable/27638428}
}
\end{verbatim}

\examplecite(43){springer:2014}
\examplecite(45)[158]{springer:2014}
\examplecite(8){truehart:1996}
\examplecite(12)[37]{truehart:1996}
\examplecite(31){kirk:2007}
\examplecite(35)[186]{kirk:2007}
\exampleabbreviations
\examplebibliography

\subsubsection{Notes}

SBL now prefers to use a full URL for the DOI. See
\url{https://sblhs2.com/2018/05/03/electronic-journals-with-individually-paginated-articles/}.

\section{Special Examples}

\subsubsection{Notes}

\begin{itemize}
  \item There is a fair bit of variation in how SBL cites ancient texts from
    collections. The most common option shown in the blog in footnotes is:

    Author, Title SourceDivision (trans.\@ A. N. Translator,
    TextCollection Volume:Page).

    and in running text:

    (Author, Title SourceDivision, trans.\@ A. N. Translator,
    TextCollection Volume:Page)

    Titles are in \emph{italics} when attributed, “enquoted” when assigned by
    and editor or translator, and roman when unattributed.

    \pkg{biblatex-sbl} adopts this format even when it departs from the blog
    or handbook.
  \item The blog tends to prefer for text collections cited by abbreviation to
    appear in a list of abbreviations rather than in the bibliography.
  \item The blog helps considerably with citations in this section, but there
    are still some ambiguities for someone who is not familiar with the texts
    in question.
  \item Some adjustment will likely be needed to get the output suitable for
    your particular work. Most things are possible.
\end{itemize}

\subsection{Texts from the Ancient Near East}

\subsubsection{Citing \emph{COS}}

\begin{verbatim}
@mvcollection{COS,
  shorthand = {COS},
  editor = {Hallo, William W. and Younger, Jr., K. Lawson},
  title = {The Context of Scripture},
  volumes = {4},
  location = {Leiden},
  publisher = {Brill},
  date = {1997/2016}
}

@ancienttext{greathymnaten,
  entrysubtype = {inancientcollection},
  title = {The Great Hymn to the Aten},
  translator = {Lichtheim, Miriam},
  xref = {COS},
  volume = {1},
  text = {26},
  pages = {44-46}
}
\end{verbatim}

\examplecite[atrans](7){greathymnaten}
\examplecite(11){greathymnaten}
\exampleabbreviations

\paragraph{Notes}

Since the publication of the \emph{SBL Handbook of Style} a fourth volume of
\cite{COS} has been published.

\newrefsection

\subsubsection{Citing Other Texts}

\begin{verbatim}
@collection{ANET,
  shorthand = {ANET},
  editor = {Pritchard, James B.},
  title = {Ancient Near Eastern Texts Relating to the Old Testament},
  edition = {3},
  location = {Princeton},
  publisher = {Princeton University Press},
  date = {1969}
}

@ancienttext{suppiluliumas,
  entrysubtype = {inancientcollection},
  title = {Suppiluliumas and the Egyptian Queen},
  translator = {Goetz, Albrecht},
  xref = {ANET},
  pages = {319}
}

@book{dalley:1991,
  author = {Dalley, Stephanie},
  title = {Myths from Mesopotamia},
  location = {Oxford},
  publisher = {Oxford University Press},
  date = {1991}
}

@ancienttext{erraandishum:dalley,
  entrysubtype = {inancientcollection},
  title = {Erra and Ishum},
  xref = {dalley:1991},
  pages = {282-315}
}

@mvbook{foster:1993,
  author = {Foster, Benjamin},
  title = {Before the Muses},
  subtitle = {An Anthology of Akkadian Literature},
  volumes = {2},
  location = {Bethesda, MD},
  publisher = {CDL},
  date = {1993}
}

@ancienttext{erraandishum:foster,
  entrysubtype = {inancientcollection},
  title = {Erra and Ishum},
  xref = {foster:1993},
  volume = {1},
  pages = {771-805}
}

@mvbook{AEL,
  shorthand = {AEL},
  author = {Lichtheim, Miriam},
  title = {Ancient Egyptian Literature},
  volumes = {3},
  location = {Berkeley},
  publisher = {University of California Press},
  date = {1971/1980}
}

@ancienttext{doomedprince,
  entrysubtype = {inancientcollection},
  title = {The Doomed Prince},
  xref = {AEL},
  volume = {2},
  pages = {200-203}
}

@book{hoffner:1990,
  author = {Hoffner, Jr., Harry A.},
  title = {Hittite Myths},
  editor = {Beckman, Gary M.},
  series = {Writings from the Ancient World},
  shortseries = {WAW},
  number = {2},
  location = {Atlanta},
  publisher = {Scholars Press},
  date = {1990}
}

@ancienttext{disappearanceofsungod,
  entrysubtype = {inancientcollection},
  title = {The Disappearance of the Sun God},
  xref = {hoffner:1990}
}

@series{RIMA,
  series = {The Royal Inscriptions of Mesopotamia, Assyrian Periods},
  shortseries = {RIMA}
}

@ancienttext{ashurinscription:RIMA,
  title = {Ashur Inscription},
  xref = {RIMA},
  number = {2},
  pages = {142-145}
}

@book{ABC,
  shorthand = {ABC},
  author = {Grayson, Albert Kirk},
  title = {Assyrian and Babylonian Chronicles},
  series = {Texts from Cuneiform Sources},
  shortseries = {TCS},
  number = {5},
  location = {Locust Valley, NY},
  publisher = {Augustin},
  date = {1975}
}

@ancienttext{esarhaddonchronicle:ABC,
  title = {Esarhaddon Chronicle},
  xref = {ABC}
}

@book{georges:1967,
  author = {Dossin, Georges},
  title = {Lettres},
  series = {Archives royales de Mari},
  shortseries = {ARM},
  number = {1},
  origdate = {1946},
  location = {Paris},
  publisher = {Geuthner},
  date = {1967},
  langid = {french}
}

@book{georges:1950,
  author = {Dossin, Georges},
  title = {Correspondance de Šamši-Addu et de ses fils},
  series = {Archives royales de Mari, transcrite et traduite},
  shortseries = {ARMT},
  number = {1},
  location = {Paris},
  publisher = {Imprimerei nationale},
  date = {1950},
  langid = {french}
}

@series{ARM1,
  series = {Archives royales de Mari},
  shortseries = {ARM},
  xref = {georges:1967}
}

@series{ARMT1,
  series = {Archives royales de Mari, transcrite et traduite},
  shortseries = {ARMT},
  xref = {georges:1950}
}
\end{verbatim}

\examplecite[atrans](16){suppiluliumas}
\examplecite(5){erraandishum:dalley}
\examplecite(5){erraandishum:foster}
\examplecite(36){doomedprince}
\examplecite(12)[(§3 \mkbibparens{A I 11--17})26]{disappearanceofsungod}
\examplecite(34)[(obv.\ lines 10--17)143-144]{ashurinscription:RIMA}
\examplecite(33)[(lines 3--4)125]{esarhaddonchronicle:ABC}
\examplecite(45)[1.3]{ARM1}
\examplecite(45)[1.3]{ARMT1}
\exampleabbreviations
\examplebibliography

\paragraph{Notes}

\begin{itemize}
  \item A 2024 publication, \emph{The Labors of Idrimi}, from SBL Press,
    simply places the RIMA series name in the list of abbreviations without
    referring to the full details of the books anywhere.
  \item It is assumed that placing the editor and series inside the
    parenthesis for Hoffner is an error.
\end{itemize}

\subsection{Loeb Classical Library (Greek and Latin)}

\begin{verbatim}
@book{josephus:jewishantiquities,
  author = {Josephus},
  title = {The Jewish Antiquities},
  translator = {Thackery, Henry St.\@ J.},
  series = {Loeb Classical Library},
  shortseries = {LCL},
  location = {Cambridge},
  publisher = {Harvard University Press},
  date = {1930/1965}
}

@ancienttext{josephus:aj,
  author = {Josephus},
  title = {Antiquitates judaicae},
  shorttitle = {A.J.},
  xref = {josephus:jewishantiquities}
}

@mvbook{tacitus:histories,
  author = {Tacitus},
  title = {The Histories and The Annals},
  translator = {Moore, Clifford H. and Jackson, John},
  volumes = {4},
  series = {Loeb Classical Library},
  shortseries = {LCL},
  location = {Cambridge},
  publisher = {Harvard University Press},
  date = {1937}
}

@ancienttext{tacitus:ann,
  author = {Tacitus},
  title = {Annales},
  shorttitle = {Ann.},
  xref = {tacitus:histories}
}
\end{verbatim}

\examplecite[paren][(2.233-235)]{josephus:aj}
\examplecite(1)[(2.233-235)]{josephus:aj}
\examplecite(4)[(15.18-19)]{tacitus:ann}
\examplecite[ptrans][(2.233-235)]{josephus:aj}
\examplecite[atrans](5)[(2.233-235)]{josephus:aj}
\examplecite[atrans](6)[(15.18-19)]{tacitus:ann}
\exampleabbreviations
\examplebibliography

\subsubsection{Notes}

\begin{itemize}
  \item SBL has revised their guidelines to promote a single set of
    abbreviations based on the Latin titles. See
    \url{https://sblhs2.com/2018/02/15/josephus/}.
  \item SBL no longer recommends referring to a given author's entire corpus
    in a single entry. See
    \url{https://sblhs2.com/2018/01/18/citing-text-collections-10-lcl/}.
\end{itemize}

\subsection{Papyri, Ostraca, and Epigraphica}

\subsubsection{Papyri and Ostraca in General}

\begin{verbatim}
@mvbook{pcairzen,
  shorthand = {P.Cair.Zen.},
  editor = {Edgar, C. C.},
  title = {Zenon Papyri, Catalogue général des antiquités égyptiennes du Musée du
           Caire},
  volumes = {5},
  location = {Cairo},
  date = {1925/1940},
  options = {shorthandformat=roman}
}

@mvbook{hunt+edgar:1932,
  author = {Hunt, Arthur S. and Edgar, Campbell C.},
  title = {Select Papyri},
  series = {Loeb Classical Library},
  shortseries = {LCL},
  location = {Cambridge},
  publisher = {Harvard University Press},
  date = {1932}
}

@mvbook{huntandedgar,
  shorthand = {Hunt and Edgar},
  author = {Hunt, Arthur S. and Edgar, Campbell C.},
  title = {Select Papyri},
  series = {Loeb Classical Library},
  shortseries = {LCL},
  location = {Cambridge},
  publisher = {Harvard University Press},
  date = {1932},
  pagination = {section},
  options = {shorthandformat=roman}
}

@ancienttext{pcairzen:standard,
  title = {\citeshorthand{pcairzen}}
}

@ancienttext{pcairzen:hunt+edgar:1932,
  title = {\citeshorthand{pcairzen}},
  xref = {hunt+edgar:1932}
}

@ancienttext{pcairzen:huntandedgar,
  title = {\citeshorthand{pcairzen}},
  xref = {huntandedgar}
}
\end{verbatim}

\nocite{pcairzen}
\examplecite[paren][(59003)]{pcairzen:standard}
\examplecite(22)[(59003)]{pcairzen:standard}
\examplevolcite(22){1}[(59003)96]{pcairzen:hunt+edgar:1932}
\examplecite(22)[(59003)§31]{pcairzen:huntandedgar}
\exampleabbreviations
\examplebibliography

\newrefsection

\subsubsection{Greek Magical Papyri}

\begin{verbatim}
@book{betz:1996,
  editor = {Betz, Hans Dieter},
  title = {The Greek Magical Papyri in Translation, Including the Demotic Spells},
  edition = {2},
  location = {Chicago},
  publisher = {University of Chicago Press},
  date = {1996}
}

@mvbook{preisendaz:1973-1974,
  shorthand = {PGM},
  editor = {Preisendaz, Karl},
  translator = {Preisendaz, Karl},
  title = {Papyri Graecae Magicae: Die griechischen Zauberpapyri},
  edition = {2},
  volumes = {3},
  location = {Stuttgart},
  publisher = {Teubner},
  date = {1973/1974}
}

@ancienttext{PGM,
  title = {\citeshorthand{preisendaz:1973-1974}}
}

@ancienttext{PGM:betz,
  title = {\citeshorthand{preisendaz:1973-1974}},
  xref = {betz:1996}
}
\end{verbatim}

\nocite{preisendaz:1973-1974}
\examplecite[paren][(III. 1-164)]{PGM}
\examplecite(22)[(III. 1-164)]{PGM}
\examplecite(22)[(III. 1-164 \mkbibparens{Dillon in Betz})]{PGM:betz}
\exampleabbreviations
\examplebibliography

\paragraph{Notes}

SBL now prefers to cite the translator and edition, rather than just the
edition. See \url{https://sblhs2.com/2017/10/13/greek-magical-papyri/}. This
could be programmed into the bib entry, but then a different entry would be
needed for each spell.

\subsection{Ancient Epistles and Homilies}

\begin{verbatim}
@book{malherbe:1977,
  editor = {Malherbe, Abraham J.},
  title = {The Cynic Epistles},
  subtitle = {A Study Edition},
  series = {Stuttgarter Bibelstudien},
  shortseries = {SBS},
  number = {12},
  location = {Atlanta},
  publisher = {Scholars Press},
  date = {1977}
}

@bookinbook{heraclitus:epistle1:worley,
  author = {Heraclitus},
  title = {Epistle 1},
  translator = {Worley, David},
  pages = {187},
  crossref = {malherbe:1977}
}

@ancienttext{heraclitus:epistle1,
  author = {Heraclitus},
  title = {Epistle 1},
  xref = {heraclitus:epistle1:worley},
  options = {usexref=false}
}
\end{verbatim}

\examplecite[paren][10]{heraclitus:epistle1}
\examplecite(34)[10]{heraclitus:epistle1}
\examplecite[atrans](36)[10]{heraclitus:epistle1}
\begin{verbcite}
  \nocite{malherbe:1977}
\end{verbcite}
\exampleabbreviations
\examplebibliography

\subsection{\emph{ANF} and \emph{NPNF}, First and Second Series}

\begin{verbatim}
@mvcollection{ANF,
  shorthand = {ANF},
  editor = {Roberts, Alexander and Donaldson, James},
  title = {The Ante-Nicene Fathers},
  subtitle = {Translations of the Writings of the Fathers Down to A.D. 325},
  origdate = {1885/1887},
  volumes = {10},
  location = {Peabody, MA},
  publisher = {Hendrickson},
  date = {1994}
}

@ancienttext{clementinehomilies,
  entrysubtype = {ancientbook},
  title = {The Clementine Homilies},
  xref = {ANF},
  volume = {8},
  pages = {213-346}
}

@mvcollection{NPNF,
  shorthand = {NPNF},
  editor = {Schaff, Philip},
  title = {A Select Library of Nicene and Post-Nicene Fathers of the Christian
           Church},
  origdate = {1886/1889},
  volumes = {28},
  series = {2},
  location = {Peabody, MA},
  publisher = {Hendrickson},
  date = {1994}
}

@ancienttext{augustine:letters,
  entrysubtype = {ancientbook},
  author = {Augustine},
  title = {The Letters of St.\@ Augustin},
  translator = {Cunningham, J. G.},
  xref = {NPNF},
  volume = {1/1},
  pages = {209-593}
}
\end{verbatim}

\examplecite(14)[(1.3)223]{clementinehomilies}
\examplecite[atrans](44)[(28.3.5)252]{augustine:letters}
\exampleabbreviations

\subsubsection{Notes}

The blog introduces changes in citing ANF and NPNF. See
\url{https://sblhs2.com/2017/07/13/citing-text-collections-6-anf-and-npnf/}.

\begin{itemize}
  \item ANF and NPNF should appear in the list of abbreviations rather than
    the bibliography.
  \item The series of NPNF should be indicated by a 1 or 2 plus a solidus,
    rather than a superscript.
  \item The name of the translator should be included when citing the
    translator.
  \item The format of NPNF is slightly adjusted. (Note that later blog posts
    change the format of abbreviations to place the editor first rather than
    after the title.)
\end{itemize}

\subsection{J.-P. Migne’s Patrologia Latina and Patrologia Graeca}

\begin{verbatim}
@series{PL,
  shorthand = {PL},
  title = {Patrologia Latina},
  editor = {Migne, J.-P.},
  volumes = {217},
  location = {Paris},
  date = {1844/1855}
}

@series{PG,
  shorthand = {PG},
  title = {Patrologia Graeca},
  editor = {Migne, J.-P.},
  volumes = {161},
  location = {Paris},
  date = {1857/1886}
}

@ancienttext{gregory:orationestheologicae,
  author = {{Gregory of Nazianzus}},
  title = {Orationes theologicae},
  xref = {PG},
  volume = {36}
}
\end{verbatim}

\examplecite(6)[(4.19)128c]{gregory:orationestheologicae}
\begin{verbcite}
  \nocite{PL}
\end{verbcite}
\exampleabbreviations

\subsubsection{Notes}

It appears that the reference to PG 36:12c is an error. See
\url{https://sblhs2.com/2017/05/04/pg-citations-update/}. SBL suggests that it
may have meant to be PG 36:128c.

\subsection{Strack-Billerbeck, \emph{Kommentar zum Neuen Testament}}

\begin{verbatim}
@mvbook{Str-B,
  shorthand = {Str-B},
  author = {Strack, Hermann L. and Billerbeck, Paul},
  title = {Kommentar zum Neuen Testament aus Talmud und Midrasch},
  volumes = {6},
  location = {Munich},
  publisher = {Beck},
  date = {1922/1961},
  options = {shorthandformat=roman},
  langid = {german}
}
\end{verbatim}

\examplevolcite(3)[See the discussion of ἐκρατοῦντο in]{2}[271]{Str-B}
\exampleabbreviations
\examplebibliography

\subsection{\emph{Aufstieg und Niedergang der römischen Welt (ANRW)}}

\begin{verbatim}
@mvcollection{ANRW,
  shorthand = {ANRW},
  editor = {Temporini, Hildegard and Haase, Wolfgang},
  maintitle = {Aufstieg und Niedergang der römischen Welt},
  mainsubtitle = {Geschichte und Kultur Roms im Spiegel der neueren Forschung},
  part = {2},
  title = {Principat},
  location = {Berlin},
  publisher = {de Gruyter},
  date = {1972/},
  langid = {german},
  options = {usetitle=false}
}

@incollection{anderson:pepaideumenos,
  author = {Anderson, Graham},
  title = {The \mkbibemph{pepaideumenos} in Action},
  subtitle = {Sophists and Their Outlook in the Early Empire},
  shorttitle = {\mkbibemph{Pepaideumenos}},
  xref = {ANRW},
  volume = {33},
  part = {1},
  pages = {80-208}
}
\end{verbatim}

\examplecite(76){anderson:pepaideumenos}
\examplecite(79)[86]{anderson:pepaideumenos}
\exampleabbreviations
\examplebibliography

\subsubsection{Notes}

SBL now prefers a simplified citation in the bibliography as well as in notes.
See \url{https://sblhs2.com/2016/10/18/anrw/}. And as elsewhere the
abbreviation format should match the bibliography format.

\subsection{Bible Commentaries}

\begin{verbatim}
@commentary{hooker:1991,
  author = {Hooker, Morna},
  title = {The Gospel according to Saint Mark},
  series = {Black's New Testament Commentaries},
  shortseries = {BNTC},
  number = {2},
  location = {Peabody, MA},
  publisher = {Hendrickson},
  date = {1991}
}
\end{verbatim}

\examplecite(8)[223]{hooker:1991}
\exampleabbreviations
\examplebibliography

\subsection{SBL Seminar Papers}

\begin{verbatim}
@incollection{crenshaw:2001,
  author = {Crenshaw, James L.},
  title = {Theodicy in the Book of the Twelve},
  booktitle = {Society of Biblical Literature 2001 Seminar Papers},
  series = {Society of Biblical Literature Seminar Papers},
  shortseries = {SBLSP},
  number = {40},
  location = {Atlanta},
  publisher = {Society of Biblical Literature},
  date = {2001},
  pages = {1-18}
}
\end{verbatim}

\examplecite(33){crenshaw:2001}
\exampleabbreviations
\examplebibliography

\subsubsection{Notes}

The abbreviation for SBL Seminar Papers is incorrect in the handbook. It
should be SBLSP, not SBLSPS.

\subsection{Text Editions Published Online with No Print Counterpart}

\begin{verbatim}
@online{wilhelm:2013,
  editor = {Wilhelm, Gernot},
  title = {Der Vertrag Šuppiluliumas I. von Ḫatti mit Šattiwazza von Mitrani
           \mkbibparens{CTH 51.I}},
  shorttitle = {Der Vertrag Šuppiluliumas I.},
  datemodifier = {released},
  date = {2013-02-24},
  eprinttype = {hethiter},
  eprint = {CTH%2051.I},
  eprintclass = {INTR 2013-02-24},
  langid = {german}
}
\end{verbatim}

\examplecite(2){wilhelm:2013}
\examplecite(4){wilhelm:2013}
\examplebibliography

\subsubsection{Notes}

\begin{itemize}
  \item In line with
\url{https://sblhs2.com/2018/05/03/electronic-journals-with-individually-paginated-articles/}
\pkg{biblatex-sbl} uses a full URL for the DOI.
  \item The name format for Wilhelm in the bibliography example in the handbook
    is incorrect.
  \item The missing date in the bibliography entry is assumed to be an error.
    Compare with Caraher in §6.4.14.
\end{itemize}

\subsection{Online Database}

\begin{verbatim}
@online{cobb:figurines,
  author = {{Cobb Institute of Archaeology}},
  title = {The Figurines of Maresha, the Persian Era},
  shorttitle = {Figurines of Maresha},
  website = {DigMaster},
  url = {http://www.cobb.msstate.edu/dignew/Maresha/index.html},
  options = {indexing=false}
}

@online{caraher:2013,
  editor = {Caraher, William R.},
  title = {Pyla-Koutsopetria Archaeological Project},
  subtitle = {\mkbibparens{Overview}},
  website = {Open Context},
  datemodifier = {released},
  date = {2013-11-05},
  doi = {10.6078/M7B56GNS},
  url = {http://opencontext.org/projects/3F6DCD13-A476-488E-ED10-47D25513FCB2}
}
\end{verbatim}

\examplecite(37){cobb:figurines}
\examplecite(39){cobb:figurines}
\examplecite(15){caraher:2013}
\examplecite(17){caraher:2013}
\examplebibliography

\subsubsection{Notes}

The example in §6.3.10 of the \emph{SBLHS} places the DOI before the URL. For
consistency I have followed this format rather than the what the handbook does
in this section.

\subsection{Websites and Blogs}

\begin{verbatim}
@online{100cuneiform,
  title = {The One Hundred Most Important Cuneiform Objects},
  website = {cdli:wiki},
  url = {http://cdli.ox.ac.uk/wiki/doku.php?id=the_one_hundred_most_important_
         cuneiform_objects}
}

@online{goodacre:2014,
  entrysubtype = {blog},
  author = {Goodacre, Mark},
  title = {Jesus' Wife Fragment},
  subtitle = {Another Round-Up},
  website = {NT Blog},
  date = {2014-05-09},
  url = {http://ntweblog.blogspot.com}
}
\end{verbatim}

\examplecite(10){100cuneiform}
\examplecite(3){goodacre:2014}
\examplebibliography
\end{document}
