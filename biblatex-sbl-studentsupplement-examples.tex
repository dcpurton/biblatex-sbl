\DocumentMetadata{lang=en}
\documentclass[a4paper]{article}

\usepackage{microtype}
\usepackage{parskip}
\usepackage{xcolor}
\usepackage{fancyvrb,fvextra}
\usepackage{enverb}
\usepackage{tocloft}
\setlength{\cftbeforesecskip}{2pt plus 0.5pt}
\setcounter{tocdepth}{1}
\RecustomVerbatimEnvironment{verbatim}{Verbatim}{backgroundcolor=black!15,fontsize=\small}
\setcounter{secnumdepth}{0}
\usepackage[american, bidi=basic]{babel}
\babelprovide[import, onchar=ids]{polytonicgreek}
\babelprovide[import, onchar=ids]{russian}
\babelprovide[import, onchar=ids fonts]{armenian}
\babelprovide[import, onchar=ids fonts]{hebrew}
\babelprovide[import, onchar=ids fonts]{syriac}
\babelfont{rm}{Brill}
\babelfont{tt}[Scale=MatchLowercase]{DejaVu Sans Mono}
\babelfont[armenian]{rm}[Scale=0.85]{Noto Serif Armenian}
\babelfont[hebrew]{rm}[Scale=MatchLowercase, Contextuals=Alternate]{SBL Hebrew}
\babelfont[syriac]{rm}[Scale=MatchLowercase]{Estrangelo Edessa}
\babelfont[hebrew]{tt}[Scale=MatchLowercase]{Miriam Libre Medium}
\babelfont[syriac]{tt}[Scale=0.85]{Noto Sans Syriac}
\usepackage{csquotes}
\usepackage[colorlinks]{hyperref}
\usepackage[style=sbl, refsection=subsection+, locallabelwidth, citetracker=true]{biblatex}
\addbibresource{biblatex-sbl.bib}

\makeatletter

\colorlet{cmdcolour}{black!65}

\newcommand*{\pkg}[1]{\textsf{#1}}

\ExplSyntaxOn
% Thanks to https://tex.stackexchange.com/a/44444/87678
\cs_new_protected:Npn \sblhsblog_verb_print:n #1
  {
    \tl_set:Nn \l_tmpa_tl {#1}
    \regex_replace_all:nnN { . } { \c{string} \0 } \l_tmpa_tl
    \tl_set:Nx \l_tmpb_tl { \l_tmpa_tl }
    \tl_use:N \l_tmpb_tl
  }
\cs_set_eq:NN \verbprint \sblhsblog_verb_print:n
\ExplSyntaxOff

% hack to include a space after a control sequence that would normally be
% removed by \verbprint above
\usepackage{newunicodechar}
\newunicodechar{ }{\relax}% this is U+00A0

% redefine hyperlink anchors to be independent of \iffootnote
\DeclareFieldFormat{bibhyperlink}{%
  \bibhyperlink{\cbx@resetcount:\thefield{entrykey}}{#1}}
\DeclareFieldFormat{bibhypertarget}{%
  \ifboolexpr{
    not test {\iffieldundef{crossref}}
    and
    not test {\ifcrossrefseen}
  }
    {\bibhypertarget{\cbx@resetcount:\thefield{crossref}}{}}
    {}%
  \bibhypertarget{\cbx@resetcount:\thefield{entrykey}}{#1}}
\DeclareFieldFormat{shorttitlelink}{%
  \bibhyperlink{\cbx@resetcount:\thefield{entrykey}}{#1}}

% Redefine \smartcite so autocite prints inline with a period.
\DeclareCiteCommand{\smartcite}[\iffootnote\mkbibparens\bibfootnotewrapper]
  {\usebibmacro{prenote}}
  {\usebibmacro{cite}}
  {\multicitedelim}
  {\usebibmacro{cite:postnote}}

\DeclareCiteCommand*{\smartcite}[\iffootnote\mkbibparens\bibfootnotewrapper]
  {\usebibmacro{prenote}}
  {\usebibmacro{cite:star}%
   \usebibmacro{cite}}
  {\multicitedelim}
  {\usebibmacro{cite:postnote}}

\NewDocumentCommand{\examplenobreak}{}{%
  \par
  \@afterheading
}

\NewDocumentCommand{\exampleciteauthor}{m}{%
  \par
  \textcolor{cmdcolour}{\texttt{\textbackslash citeauthor\{#1\}}}\par
  \@afterheading
  \citeauthor{#1}%
}

% \examplecite(<footnote number>){<citetype>}[<prenote>][<postnote>]{entryid}
% \examplecite*(<footnote number>){<citetype>}[<prenote>][<postnote>]{entryid}
\NewDocumentCommand{\examplecite}{sO{auto}d()oom}{%
  \par
  \textcolor{cmdcolour}{%
    \texttt{%
      \textbackslash #2cite%
      \IfBooleanT{#1}{*}%
      \IfNoValueF{#4}{[\verbprint{#4}]}%
      \IfNoValueF{#5}{[\verbprint{#5}]}%
      \{#6\}}}\par
  \@afterheading
  \IfNoValueF{#3}{\quad #3.\space\strut}%
  \IfBooleanTF{#1}
    {\printexamplecite*{#2cite}{#4}{#5}{#6}}
    {\printexamplecite{#2cite}{#4}{#5}{#6}}}

% \examplevolcite(<footnote number>){<citetype>}[<prenote>]{<volume>}[<postnote>]{entryid}
% \examplevolcite*(<footnote number>){<citetype>}[<prenote>]{<volume>}[<postnote>]{entryid}
\NewDocumentCommand{\examplevolcite}{sO{a}d()omom}{%
  \par
  \textcolor{cmdcolour}{%
    \texttt{%
      \textbackslash #2volcite%
      \IfBooleanT{#1}{*}%
      \IfNoValueF{#4}{[\verbprint{#4}]}%
      \{#5\}%
      \IfNoValueF{#6}{[\verbprint{#6}]}%
      \{#7\}}}\par
  \@afterheading
  \IfNoValueF{#3}{\quad #3.\space\strut}%
  \IfBooleanTF{#1}
    {\printexamplevolcite*{#2volcite}{#4}{#5}{#6}{#7}}
    {\printexamplevolcite{#2volcite}{#4}{#5}{#6}{#7}}}

\NewDocumentCommand{\printexamplecite}{smmmm}{%
  \IfNoValueTF{#4}
    {\IfNoValueTF{#3}
       {\IfBooleanTF{#1}
          {\csname #2\endcsname*{#5}}
          {\csname #2\endcsname{#5}}}
       {\IfBooleanTF{#1}
          {\csname #2\endcsname*[#3]{#5}}
          {\csname #2\endcsname[#3]{#5}}}}
    {\IfNoValueTF{#3}
       {\IfBooleanTF{#1}
          {\csname #2\endcsname*[#3][]{#5}}
          {\csname #2\endcsname[#3][]{#5}}}
       {\IfBooleanTF{#1}
          {\csname #2\endcsname*[#3][#4]{#5}}
          {\csname #2\endcsname[#3][#4]{#5}}}}}

\NewDocumentCommand{\printexamplevolcite}{smmmmm}{%
  \IfNoValueTF{#3}
    {\IfNoValueTF{#5}
       {\IfBooleanTF{#1}
          {\csname #2\endcsname*{#4}{#6}}
          {\csname #2\endcsname{#4}{#6}}}
       {\IfBooleanTF{#1}
          {\csname #2\endcsname*{#4}[#5]{#6}}
          {\csname #2\endcsname{#4}[#5]{#6}}}}
    {\IfNoValueTF{#5}
       {\IfBooleanTF{#1}
          {\csname #2\endcsname*[#3]{#4}{#6}}
          {\csname #2\endcsname[#3]{#4}{#6}}}
       {\IfBooleanTF{#1}
          {\csname #2\endcsname*[#3]{#4}[#5]{#6}}
          {\csname #2\endcsname[#3]{#4}[#5]{#6}}}}}

\NewDocumentEnvironment{verbcite}{}{%
  \enverb{}%
}{%
  \par
  \color{cmdcolour}%
  \enverbListing{Verbatim}{}%
  \par
  \@afterheading
  \normalcolor
  \enverbExecute
}

\NewDocumentEnvironment{fverbcite}{m}{%
  \enverb{}%
}{%
  \par
  \color{cmdcolour}%
  \enverbListing{Verbatim}{}%
  \par
  \@afterheading
  \normalcolor
  \renewcommand\footnote[1]{\toggletrue{blx@footnote}##1}%
  \quad #1.\space\strut\enverbExecute
}

\NewDocumentEnvironment{verbtext}{}{%
  \enverb{}%
}{%
  \par
  \color{cmdcolour}%
  \enverbListing{Verbatim}{}%
  \normalcolor
  \par
}

\NewDocumentCommand{\exampleabbreviations}{}{%
  \par
  \textcolor{cmdcolour}{\texttt{\textbackslash printbiblist\{abbreviations\}}}
  \@afterheading
  \printbiblist[heading=none]{abbreviations}}

\NewDocumentCommand{\exampleancientsources}{}{%
  \par
  \textcolor{cmdcolour}{%
    \texttt{\textbackslash printbiblist[heading=subbibliography, title=Ancient Sources,\\
      \strut\quad type=ancienttext]\{abbreviations\}}}
  \@afterheading
  \printbiblist[heading=subbibliography, title=Ancient Sources,
    type=ancienttext]{abbreviations}}

\NewDocumentCommand{\examplesecondarysources}{}{%
  \par
  \textcolor{cmdcolour}{%
    \texttt{\textbackslash printbiblist[heading=subbibliography, title=Secondary Sources,\\
      \strut\quad nottype=ancienttext]\{abbreviations\}}}
  \@afterheading
  \printbiblist[heading=subbibliography, title=Secondary Sources,
    nottype=ancienttext, nottype=abbreviation]{abbreviations}}

\NewDocumentCommand{\examplesigla}{}{%
  \par
  \textcolor{cmdcolour}{%
    \texttt{\textbackslash printbiblist[heading=subbibliography, title=Sigla
      and Grammatical \\
      \strut\quad Abbreviations, type=abbreviation]\{abbreviations\}}}
  \@afterheading
  \printbiblist[heading=subbibliography, title=Sigla and Grammatical
    Abbreviations, type=abbreviation]{abbreviations}}

\NewDocumentCommand{\examplebibliography}{}{%
  \par
  \textcolor{cmdcolour}{\texttt{\textbackslash printbibliography}}
  \@afterheading
  \printbibliography[heading=none]}

\NewDocumentCommand{\examplereferences}{m}{%
  \subsection*{References}
  \url{#1}}
\makeatother


\begin{document}
\title{SBL Handbook of Style}
\author{Student Supplement}
\date{SBL Press}
\maketitle

\tableofcontents

\section{Biblical Citations}

\begin{verbatim}
@inbook{petersen:2006,
  author = {Petersen, David L.},
  title = {Ezekiel},
  pages = {1096-1167},
  booktitle = {The HarperCollins Study Bible Fully Revised and Updated: New Revised
               Standard Version, with the Apocryphal\slash Deuterocanonical Books},
  editor = {Attridge, Harold W. and others},
  location = {San Francisco},
  publisher = {HarperSanFrancisco},
  date = {2006}
}
\end{verbatim}

\examplecite(3)[1096]{petersen:2006}
\examplecite(5)[1096]{petersen:2006}
\examplebibliography

\section{Working with Biblical Commentaries}

\subsection{Series Title Volume Title}

\begin{verbatim}
@commentary{westermann:1995,
  author =  {Westermann, Claus},
  title = {Genesis 12--36},
  translator = {Scullion, John J.},
  series = {Continental Commentaries},
  shortseries = {CC},
  location = {Minneapolis},
  publisher = {Fortress},
  date = {1995}
}
\end{verbatim}

\examplecite(18)[25]{westermann:1995}
\examplecite(20)[44]{westermann:1995}
\exampleabbreviations
\examplebibliography

\subsection{Multivolume Commentaries}

\begin{verbatim}
@mvcommentary{dahood:1965-1970,
  author = {Dahood, Mitchell},
  title = {Psalms},
  volumes = {3},
  series = {Anchor Bible},
  shortseries = {AB},
  number = {16--17A},
  location = {Garden City, NY},
  publisher = {Doubleday},
  date = {1965/1970}
}
\end{verbatim}

\examplevolcite(4){3}[127]{dahood:1965-1970}
\examplevolcite(7){2}[121]{dahood:1965-1970}
\exampleabbreviations
\examplebibliography

\newrefsection

\begin{verbatim}
@commentary{dahood:1965,
  author = {Dahood, Mitchell},
  title = {Psalms I: 1--50},
  shorttitle = {Psalms I: 1--50},
  series = {Anchor Bible},
  shortseries = {AB},
  number = {16},
  location = {Garden City, NY},
  publisher = {Doubleday},
  date = {1965}
}

@commentary{dahood:1968,
  author = {Dahood, Mitchell},
  title = {Psalms II: 51--100},
  shorttitle = {Psalms II: 51--100},
  series = {Anchor Bible},
  shortseries = {AB},
  number = {17},
  location = {Garden City, NY},
  publisher = {Doubleday},
  date = {1968}
}
\end{verbatim}

\examplecite(78)[44]{dahood:1965}
\examplecite(79)[78]{dahood:1965}
\examplecite(82)[347]{dahood:1968}
\examplecite(86)[351]{dahood:1968}
\exampleabbreviations
\examplebibliography

\subsubsection{Notes}

\pkg{biblatex-sbl} follows the \emph{SBL Handbook of Style} blog and does
not include the entire multivolume work in the bibliography entries of the
individual volumes. See
\url{https://sblhs2.com/2018/05/10/series-volume-identifiers/}.

\newrefsection

\begin{verbatim}
@mvcommentary{NIB,
  shorthand = {NIB},
  editor = {Keck, Leander E.},
  title = {The New Interpreter's Bible},
  volumes = {12},
  location = {Nashville},
  publisher = {Abingdon},
  date = {1994/2004}
}

@incommentary{miller:jeremiah,
  author = {Miller, Patrick D.},
  title = {The Book of Jeremiah: Introduction, Commentary, and Reflections},
  xref = {NIB},
  volume = {6},
  pages = {553-926},
}
\end{verbatim}

\examplecite(1)[577]{miller:jeremiah}
\exampleabbreviations
\examplebibliography

\subsubsection{Notes}

\pkg{biblatex-sbl} follows more recent publications and includes the title in
these citations. See, for example,
\url{https://www.sbl-site.org/assets/pdfs/pubs/9781628375008_OA.pdf}.

\subsection{Single-Volume Commentaries on the Entire Bible}

\begin{verbatim}
@incommentary{partain:1995,
  author = {Partain, Jack G.},
  title = {Numbers},
  pages = {175-179},
  booktitle = {Mercer Commentary on the Bible},
  editor = {Mills, Watson E. and others},
  location = {Macon, GA},
  publisher = {Mercer University Press},
  date = {1995}
}
\end{verbatim}

\examplecite(5){partain:1995}
\examplecite(8)[175]{partain:1995}
\examplebibliography

\section{Bible Dictionaries and Encyclopedias}

\begin{verbatim}
@mvreference{ABD,
  shorthand = {ABD},
  editor = {Freedman, David Noel},
  title = {Anchor Bible Dictionary},
  volumes = {6},
  location = {New York},
  publisher = {Doubleday},
  date = {1992},
  pagination = {subverbo}
}

@inreference{walters:jacobnarrative,
  author = {Walters, Stanley D.},
  title = {Jacob Narrative},
  xref = {ABD},
  volume = {3},
  pages = {599-609}
}
\end{verbatim}

\examplecite(1)[599-609]{walters:jacobnarrative}
\exampleabbreviations
\examplebibliography

\subsubsection{Notes}

\begin{itemize}
  \item The pages for \cite{walters:jacobnarrative} are incorrect in the
    Student Supplement bibliography entry.
  \item SBL prefers the shorter form in the bibliography with \cite{ABD}
    appearing in a list of abbreviations. See point 5 in
    \url{https://sblhs2.com/2017/04/13/citing-reference-works-5-topical-dictionaries-and-encyclopedias/}.
    This includes when multiple articles are cited from the same dictionary or
    encyclopedia.
\end{itemize}

\section{Citations of Electronic Sources}

\begin{verbatim}
@article{mclay:2006,
  author = {McLay, R. Timothy},
  title = {The Goal of Teaching Biblical and Religious Studies in the Context of an
           Undergraduate Education},
  journaltitle = {SBL Forum},
  date = {2006/10/6},
  url = {http://www.sbl-site.org/publications/article.aspx?articleId=581}
}
\end{verbatim}

\examplecite(7){mclay:2006}
\examplecite(9){mclay:2006}
\examplebibliography

\section{Footnotes}

\begin{verbatim}
@book{vanseters:1992,
  author = {Van Seters, John},
  title = {Prologue to History: The Yahwist as Historian in Genesis},
  shorttitle = {Prologue},
  location = {Louisville},
  publisher = {Westminster John Knox},
  date = {1992}
}
\end{verbatim}

\examplecite(7)[115]{vanseters:1992}
\examplecite(8)[150]{vanseters:1992}
\examplebibliography

\section{Bibliography}

\begin{verbatim}
@mvreference{NIDNTT,
  shorthand = {NIDNTT},
  editor = {Brown, Colin},
  title = {New International Dictionary of New Testament Theology},
  volumes = {4},
  location = {Grand Rapids},
  publisher = {Zondervan},
  date = {1975/1985}
}

@mvcommentary{dahood:1965-1970,
  author = {Dahood, Mitchell},
  title = {Psalms},
  volumes = {3},
  series = {Anchor Bible},
  shortseries = {AB},
  number = {16--17A},
  location = {Garden City, NY},
  publisher = {Doubleday},
  date = {1965/1970}
}

@incollection{harrington:1986,
  author = {Harrington, Daniel},
  title = {Palestinian Adaptations of Biblical Narratives and Prophecies},
  pages = {239-247},
  booktitle = {Early Judaism and Its Modern Interpreters},
  editor = {Kraft, R. A. and Nickelsburg, G. W. E.},
  series = {The Bible and Its Modern Interpreters},
  shortseries = {BMI},
  number = {2},
  location = {Philadelphia},
  publisher = {Fortress},
  date = {1986}
}

@article{harrington:1970,
  author = {Harrington, Daniel},
  title = {The Original Language of Pseudo-Philo’s Liber Antiquitatum Biblicarum},
  journaltitle = {Harvard Theological Review},
  shortjournal = {HTR},
  volume = {63},
  date = {1970},
  pages = {503-514}
}

@reference{Jastrow,
  shorthand = {Jastrow},
  editor = {Jastrow, Morris},
  editortype = {compiler},
  title = {A Dictionary of the Targumim, the Talmud Babli and Yerushalmi, and the
           Midrashic Literature with an Index of Scriptural Quotations},
  location = {London and New York},
  publisher = {Luzac and G. P. Putnam’s Sons},
  date = {1903},
  pagination = {subverbo},
  options = {shorthandformat=roman}
}

@reference{DMBI,
  shorthand = {DMBI},
  editor = {McKim, Donald K.},
  title = {Dictionary of Major Biblical Interpreters},
  location = {Downers Grove, IN},
  publisher = {InterVarsity Press},
  date = {2007}
}

@article{mclay:2006,
  author = {McLay, R. Timothy},
  title = {The Goal of Teaching Biblical and Religious Studies in the Context of an
           Undergraduate Education},
  shorttitle = {Goal of Teaching},
  journaltitle = {SBL Forum},
  date = {2006-10-06},
  url = {http://www.sbl-site.org/publications/article.aspx?articleId=581}
}

@mvreference{DBI,
  shorthand = {DBI},
  editor = {Hayes, John},
  title = {Dictionary of Biblical Interpretation},
  volumes = {2},
  location = {Nashville},
  publisher = {Abingdon},
  date = {1999}
}

@inreference{oday:intertextuality,
  author = {O’Day, Gail},
  title = {Intertextuality},
  xref = {DBI},
  volume = {1},
  pages = {546-548}
}

@commentary{rad:1990,
  author = {von Rad, Gerhard},
  title = {Genesis: A Commentary},
  translator = {Marks, John H.},
  series = {Old Testament Library},
  shortseries = {OTL},
  location = {Philadelphia},
  publisher = {Westminster},
  date = {1990}
}
\end{verbatim}

\begin{verbcite}
  \nocite{NIDNTT, dahood:1965-1970, harrington:1986, harrington:1970, Jastrow,
    DMBI, mclay:2006, oday:intertextuality, rad:1990}
\end{verbcite}
\exampleabbreviations
\examplebibliography

\subsubsection{Notes}

\begin{itemize}
  \item Following the blog, \pkg{biblatex-sbl} prefers to use abbreviations
    for dictionaries and encyclopedias.
  \item The \emph{Student Supplement} includes the definite article in sorting
    the two Harrington bibliography entries. In line with more recent
    publications, \pkg{biblatex-sbl} ignores articles for the purposes of
    sort order in the bibliography. See
    \url{https://www.sbl-site.org/assets/pdfs/pubs/9781628375008_OA.pdf} for
    an example of this sort order. This is also the position taken by the
    \emph{Chicago Manual of Style}. See §14.71 “Alphabetical order for titles
    by the same author” in the 18th edition.
\end{itemize}
\end{document}
