\DocumentMetadata{lang=en}
\documentclass[a4paper]{article}
\usepackage{microtype}
\usepackage{parskip}
\usepackage{xcolor}
\usepackage{fancyvrb,fvextra}
\usepackage{enverb}
\usepackage{tocloft}
\setlength{\cftbeforesecskip}{2pt plus 0.5pt}
\RecustomVerbatimEnvironment{verbatim}{Verbatim}{backgroundcolor=black!15,fontsize=\small}
\setcounter{secnumdepth}{0}
\usepackage[american]{babel}
\babelprovide[import, onchar=ids]{polytonicgreek}
\babelprovide[import, onchar=ids]{russian}
\babelprovide[import, onchar=ids fonts]{armenian}
\babelfont{rm}{Brill}
\babelfont{tt}[Scale=MatchLowercase, Renderer=Harfbuzz]{DejaVu Sans Mono}
\babelfont[armenian]{rm}[Scale=0.85, Renderer=Harfbuzz]{Noto Serif Armenian}
\usepackage{csquotes}
\usepackage[colorlinks]{hyperref}
\usepackage[style=sbl, refsection=section, locallabelwidth, autocite=plain]{biblatex}
\addbibresource{biblatex-sbl.bib}

\makeatletter

\newcommand*{\pno}{\textbackslash pno }
\newcommand*{\ppno}{\textbackslash ppno }
\newcommand*{\nopp}{\textbackslash nopp }

\NewDocumentCommand{\examplenobreak}{}{%
  \par
  \@afterheading
}

\NewDocumentCommand{\examplenocite}{m}{%
  \par
  \texttt{\textbackslash nocite\{*\}}\par
  \@afterheading
  \nocite{#1}%
}

\NewDocumentCommand{\exampleciteauthor}{m}{%
  \par
  \texttt{\textbackslash citeauthor\{#1\}}\par
  \@afterheading
  \citeauthor{#1}%
}

% \examplecite(<footnote number>){<citetype>}[<prenote>][<postnote>]{entryid}
% \examplecite*(<footnote number>){<citetype>}[<prenote>][<postnote>]{entryid}
\NewDocumentCommand{\examplecite}{sO{auto}d()oom}{%
  \par
  \texttt{%
    \textbackslash #2cite%
    \IfBooleanT{#1}{*}%
    \IfNoValueF{#4}{[#4]}%
    \IfNoValueF{#5}{[#5]}%
    \{#6\}}\par
  \@afterheading
  \IfNoValueF{#3}{\quad #3.\space}%
  \IfBooleanTF{#1}
    {\printexamplecite*{#2cite}{#4}{#5}{#6}}
    {\printexamplecite{#2cite}{#4}{#5}{#6}}%
  \IfNoValueF{#3}{.}}

% \examplecite(<footnote number>){<citetype>}[<prenote>]{<volume>}[<postnote>]{entryid}
% \examplecite*(<footnote number>){<citetype>}[<prenote>]{<volume>}[<postnote>]{entryid}
\NewDocumentCommand{\examplevolcite}{sO{a}d()omom}{%
  \par
  \texttt{%
    \textbackslash #2volcite%
    \IfBooleanT{#1}{*}%
    \IfNoValueF{#4}{[#4]}%
    \{#5\}%
    \IfNoValueF{#6}{[#6]}%
    \{#7\}}\par
  \@afterheading
  \IfNoValueF{#3}{\quad #3.\space}%
  \IfBooleanTF{#1}
    {\printexamplevolcite*{#2volcite}{#4}{#5}{#6}{#7}}
    {\printexamplevolcite{#2volcite}{#4}{#5}{#6}{#7}}%
  \IfNoValueF{#3}{.}}

\NewDocumentCommand{\printexamplecite}{smmmm}{%
  \IfNoValueTF{#4}
    {\IfNoValueTF{#3}
       {\IfBooleanTF{#1}
          {\csname #2\endcsname*{#5}}
          {\csname #2\endcsname{#5}}}
       {\IfBooleanTF{#1}
          {\csname #2\endcsname*[#3]{#5}}
          {\csname #2\endcsname[#3]{#5}}}}
    {\IfNoValueTF{#3}
       {\IfBooleanTF{#1}
          {\csname #2\endcsname*[#3][]{#5}}
          {\csname #2\endcsname[#3][]{#5}}}
       {\IfBooleanTF{#1}
          {\csname #2\endcsname*[#3][#4]{#5}}
          {\csname #2\endcsname[#3][#4]{#5}}}}}

\NewDocumentCommand{\printexamplevolcite}{smmmmm}{%
  \IfNoValueTF{#3}
    {\IfNoValueTF{#5}
       {\IfBooleanTF{#1}
          {\csname #2\endcsname*{#4}{#6}}
          {\csname #2\endcsname{#4}{#6}}}
       {\IfBooleanTF{#1}
          {\csname #2\endcsname*{#4}[#5]{#6}}
          {\csname #2\endcsname{#4}[#5]{#6}}}}
    {\IfNoValueTF{#5}
       {\IfBooleanTF{#1}
          {\csname #2\endcsname*[#3]{#4}{#6}}
          {\csname #2\endcsname[#3]{#4}{#6}}}
       {\IfBooleanTF{#1}
          {\csname #2\endcsname*[#3]{#4}[#5]{#6}}
          {\csname #2\endcsname[#3]{#4}[#5]{#6}}}}}

\NewDocumentEnvironment{verbcite}{}{%
  \enverb{}%
}{%
  \par
  \enverbListing{Verbatim}{}%
  \par
  \@afterheading
  \enverbExecute
}

\NewDocumentEnvironment{fverbcite}{m}{%
  \enverb{}%
}{%
  \par
  \enverbListing{Verbatim}{}%
  \par
  \@afterheading
  \renewcommand\footnote[1]{##1}%
  \quad #1.\space\enverbExecute
}

\NewDocumentEnvironment{verbtext}{}{%
  \enverb{}%
}{%
  \par
  \enverbListing{Verbatim}{}%
  \par
}

\NewDocumentCommand{\exampleabbreviations}{}{%
  \par
  \texttt{\textbackslash printbiblist\{abbreviations\}}
  \@afterheading
  \printbiblist[heading=none]{abbreviations}}

\NewDocumentCommand{\exampleancientsources}{}{%
  \par
  \texttt{\textbackslash printbiblist[heading=subbibliography, title=Ancient Sources,\\
  \strut\quad type=ancienttext]\{abbreviations\}}
  \@afterheading
  \printbiblist[heading=subbibliography, title=Ancient Sources,
    type=ancienttext]{abbreviations}}

\NewDocumentCommand{\examplesecondarysources}{}{%
  \par
  \texttt{\textbackslash printbiblist[heading=subbibliography, title=Secondary Sources,\\
  \strut\quad nottype=ancienttext]\{abbreviations\}}
  \@afterheading
  \printbiblist[heading=subbibliography, title=Secondary Sources,
    nottype=ancienttext]{abbreviations}}

\NewDocumentCommand{\examplebibliography}{}{%
  \par
  \texttt{\textbackslash printbibliography}
  \@afterheading
  \printbibliography[heading=none]}

\NewDocumentCommand{\examplereferences}{m}{%
  \subsection*{References}
  \url{#1}}
\makeatother

\begin{document}
\title{SBL Handbook of Style}
\author{Explanations, Clarifications, and Expansions}
\date{SBL Press}
\maketitle

\tableofcontents

\section{\emph{Brill Dictionary of Ancient Greek} (20 April 2021)}

\begin{verbatim}
@reference{MGS,
  shorthand = {MGS},
  author = {Montanari, Franco},
  title = {The Brill Dictionary of Ancient Greek},
  editor = {Goh, Madeleine and Schroeder, Chad},
  location = {Leiden},
  publisher = {Brill},
  date = {2015},
  options = {shorthandformat=roman}
}
\end{verbatim}

\examplecite(1){MGS}
\exampleabbreviations
\examplereferences{https://sblhs2.com/2021/04/20/brill-dictionary-of-ancient-greek/}

\section{Snippet Text Collections (28 March 2019)}

\begin{verbatim}
@series{ACCS,
  series = {Ancient Christian Commentary on Scripture},
  shortseries = {ACCS}
}

@book{edwards:1999,
  editor = {Edwards, Mark J.},
  title = {Galatians, Ephesians, Philippians},
  series = {\citeseries{ACCS} New Testament},
  number = {8},
  location = {Downers Grove, IL},
  publisher = {InterVarsity Press},
  date = {1999}
}

@ancienttext{victorinus:ephesians,
  author = {Victorinus},
  title = {Epistle to the Ephesians},
  related = {edwards:1999},
  relatedstring = {quoted in}
}

@ancienttext{theodoret:galatians,
  author = {Theodoret},
  title = {Epistle to the Galatians},
  related = {edwards:1999},
  relatedstring = {quoted in}
}
\end{verbatim}

\examplecite(4)[(1.2.12)129]{victorinus:ephesians}
\examplecite(8)[(5.13)77]{theodoret:galatians}
\examplenocite{ACCS}
\exampleabbreviations
\examplebibliography
\examplereferences{https://sblhs2.com/2019/03/28/snippet-text-collections/}

\section{Update: Citing an Untitled Introduction (18 January 2019)}

\begin{verbatim}
@suppbook{boers:1996,
  author = {Boers, Hendrikus},
  title = {introduction},
  booktitle = {How to Read the New Testament},
  booksubtitle = {An Introduction to Linguistic and Historical-Critical Methodology},
  bookauthor = {Egger, Wilhelm},
  translator = {Heinegg, Peter},
  location = {Peabody, MA},
  publisher = {Hendrickson},
  date = {1996}
}
\end{verbatim}

\examplecite(15){boers:1996}
\examplebibliography
\examplereferences{https://sblhs2.com/2019/01/18/update-citing-an-untitled-introduction/}

\section{Citing a Chapter from a Single-Authored Work with Editors (10 January 2019)}

\begin{verbatim}
@inbook{lawson:2016,
  author = {Younger, Jr., K. Lawson},
  title = {The Origins of the Arameans},
  pages = {35-107},
  booktitle = {A Political History of the Arameans},
  booksubtitle = {From Their Origins to the End of Their Polities},
  series = {Archaeology and Biblical Studies},
  shortseries = {ABS},
  number = {13},
  location = {Atlanta},
  publisher = {SBL Press},
  date = {2016}
}

@book{matassa:2018,
  author = {Matassa, Lidia D.},
  title = {Invention of the First-Century Synagogue},
  editor = {Silverman, Jason M. and Watson, J. Murray},
  series = {Ancient Near East Monographs},
  shortseries = {ANEM},
  number = {22},
  location = {Atlanta},
  publisher = {SBL Press},
  date = {2018}
}

@inbook{matassa:delos:2018,
  author = {Matassa, Lidia D.},
  title = {Delos},
  pages = {37-77},
  crossref = {matassa:2018}
}
\end{verbatim}

\examplecite(16)[35-107]{lawson:2016}
\examplecite(12){matassa:delos:2018}
\citereset
\examplecite(12)[37-77]{matassa:2018}
\exampleabbreviations
\examplebibliography
\examplereferences{https://sblhs2.com/2019/01/10/citing-a-chapter-from-a-single-authored-work-with-editors/}

\section{Citing Reference Works 11: Cambridge History of Christianity (23 August 2018)}

\begin{verbatim}
@collection{CHC1,
 editor = {Mitchell, Margaret M. and Young, Frances M.},
 title = {Origins to Constantine},
 series = {Cambridge History of Christianity},
 shortseries = {CHC},
 number = {1},
 location = {Cambridge},
 publisher = {Cambridge University Press},
 date = {2006}
}

@collection{CHC2,
 editor = {Casiday, Augustine and Norris, Frederick W.},
 title = {Constantine to c.~600},
 series = {Cambridge History of Christianity},
 shortseries = {CHC},
 number = {2},
 location = {Cambridge},
 publisher = {Cambridge University Press},
 date = {2007}
}

@incollection{marcus:2006,
  author = {Marcus, Joel},
  title = {Jewish Christianity},
  pages = {87-102},
  crossref = {CHC1}
}

@incollection{freyne:2006,
  author = {Freyne, Sean},
  title = {Galilee and Judaea in the First Century},
  pages = {37-52},
  crossref = {CHC1}
}

@incollection{vandam:2007,
  author = {Van Dam, Raymond},
  title = {Bishops and Society},
  pages = {343-366},
  crossref = {CHC2}
}

@incollection{lohr:2007,
  author = {Löhr, Winrich},
  title = {Western Christanities},
  pages = {9-51},
  crossref = {CHC2}
}
\end{verbatim}

\examplecite(22){marcus:2006}
\examplecite(23){freyne:2006}
\examplecite(23){vandam:2007}
\examplecite(23){lohr:2007}
\exampleabbreviations
\examplebibliography
\examplereferences{https://sblhs2.com/2018/08/23/citing-reference-works-11-cambridge-history-of-christianity/}

\section{Special Footnotes (28 June 2018)}

\begin{verbatim}
@inbook{moore:2017,
  author = {Moore, Stephen D.},
  title = {Why the Johannine Jesus Weeps at the Tomb of Lazarus},
  booktitle = {Mixed Feelings and Vexed Passions: Exploring Emotions in Biblical
               Literature},
  editor = {Spencer, F. Scott},
  series = {Resources for Biblical Study},
  shortseries = {RBS},
  location = {Atlanta},
  publisher = {SBL Press},
  date = {2017}
}
\end{verbatim}

\texttt{An earlier version of this essay appears as \textbackslash
cite*\{moore:2017\}. Reused here\\ with permission.}

An earlier version of this essay appears as \cite*{moore:2017}. Reused here
with permission.

\exampleabbreviations
\examplebibliography
\examplereferences{https://sblhs2.com/2018/06/28/special-footnotes/}

\section{Abbreviations Lists (24 May 2018)}

\begin{verbatim}
@series{AB,
  series = {Anchor Bible},
  shortseries = {AB}
}

@mvreference{ABD,
  shorthand = {ABD},
  editor = {Freedman, David Noel},
  title = {Anchor Bible Dictionary},
  volumes = {6},
  location = {New York},
  publisher = {Doubleday},
  date = {1992}
}

@ancienttext{philo:abr,
  author = {Philo},
  title = {De Abrahamo},
  shorttitle = {Abr\adddot},
}

@ancienttext{philo:agr,
  author = {Philo},
  title = {De agricultura},
  shorttitle = {Agr\adddot},
}

@ancienttext{graniuslicinianus:ann,
  author = {{Granius Licinianus}},
  title = {Annales},
  shorttitle = {Ann\adddot}
}

@mvbook{tacitus:histories,
  author = {Tacitus},
  title = {The Histories and The Annals},
  translator = {Moore, Clifford H. and Jackson, John},
  volumes = {4},
  series = {Loeb Classical Library},
  shortseries = {LCL},
  location = {Cambridge},
  publisher = {Harvard University Press},
  date = {1937}
}

@ancienttext{tacitus:ann,
  author = {Tacitus},
  title = {Annales},
  shorttitle = {Ann\adddot},
  xref = {tacitus:histories}
}

@journal{AJSL,
  journaltitle = {American Journal of Semitic Languages and Literature},
  shortjournal = {AJSL}
}

@journal{atlantis,
  journaltitle = {Atlantis: Journal of the Spanish Association of Anglo-American
                  Studies},
  shortjournal = {Atlantis}
}

@series{AzTh,
  series = {Arbeiten zur Theologie},
  shortseries = {AzTh}
}

@journal{BibInt,
  journaltitle = {Biblical Interpretation},
  shortjournal = {BibInt}
}

@series{BibIntSeries,
  series = {Biblical Interpretation Series},
  shortseries = {BibInt}
}

@journal{BSac,
  journaltitle = {Bibliotheca Sacra},
  shortjournal = {BSac}
}

@series{JSOTSup,
  series = {Journal for the Study of the Old Testament Supplement Series},
  shortseries = {JSOTSup}
}

@ancienttext{justinmartyr:1apol,
  author = {{Justin Martyr}},
  title = {First Apology},
  shorttitle = {1~Apol\adddot}
}

@ancienttext{1en,
  title = {1~Enoch},
  shorttitle = {1~En\adddot}
}

@ancienttext{1QM,
  title = {War Scroll},
  shorttitle = {1QM}
}

@ancienttext{4QpNah,
  title = {Pesher Nahum},
  shorttitle = {4QpNah}
}

@ancienttext{livy:aburbe,
  author = {Livy},
  title = {Ab urbe condita},
  shorttitle = {Ab urbe cond\adddot}
}

@ancienttext{cicero:agr,
  author = {Cicero},
  title = {De lege agraria},
  shorttitle = {Agr\adddot}
}

@ancienttext{plutarch:ant,
  author = {Plutarch},
  title = {Antonius},
  shorttitle = {Ant\adddot}
}

@ancienttext{dionysius:ant,
  author = {{Dionysius of Halicarnassus}},
  title = {Antiquitates romanae},
  shorttitle = {Ant\adddotspace rom\adddot}
}
\end{verbatim}

\examplenocite{AB, ABD, philo:abr, philo:agr, tacitus:ann,
graniuslicinianus:ann, AJSL, atlantis, AzTh, BibIntSeries, BibInt, BSac,
JSOTSup, justinmartyr:1apol, 1en, 1QM, 4QpNah, livy:aburbe, cicero:agr,
plutarch:ant, dionysius:ant}
\exampleancientsources
\examplesecondarysources
\examplereferences{https://sblhs2.com/2018/05/24/abbreviations-lists/}

\section{Series Volume Identifiers: Old/New and Concurrent Series (17 May 2018)}

\begin{verbatim}
@book{robinson:1952,
  author = {Robinson, John A. T.},
  title = {The Body: A Study in Pauline Theology},
  series = {Studies in Biblical Theology},
  shortseries = {SBT},
  number = {1/5},
  location = {London},
  publisher = {SCM},
  date = {1952}
}

@book{jeremias:1967,
  author = {Jeremias, Joachim},
  title = {The Prayers of Jesus},
  shorttitle = {Prayers},
  series = {Studies in Biblical Theology},
  shortseries = {SBT},
  number = {2/6},
  location = {Naperville, IL},
  publisher = {Allenson},
  date = {1967}
}

@book{frances:2014,
  author = {Young, Frances},
  title = {Ways of Reading Scripture},
  series = {Wissenschaftliche Untersuchungen zum Neuen Testament},
  shortseries = {WUNT},
  number = {1/369},
  location = {Tübingen},
  publisher = {Mohr Siebeck},
  date = {2014}
}

@book{reynolds+etal:2014,
  editor = {Reynolds, Benjamin E. and Lugioyo, Brian and Vanhoozer, Kevin J.},
  title = {Reconsidering the Relationship between Biblical and Systematic Theology in
           the New Testament},
  series = {Wissenschaftliche Untersuchungen zum Neuen Testament},
  shortseries = {WUNT},
  number = {2/369},
  location = {Tübingen},
  publisher = {Mohr Siebeck},
  date = {2014}
}

@book{witte:2015,
  author = {Witte, Markus},
  title = {Texte und Kontexte des Sirachbuchs: Gesammelte Studien zu Ben Sira und zur
           frühjüdischen Weisheit},
  series = {Forschungen zum Alten Testament},
  shortseries = {FAT},
  number = {1/98},
  location = {Tübingen},
  publisher = {Mohr Siebeck},
  date = {2015},
  langid = {ngerman}
}

@book{tucker:2015,
  author = {Tucker, Paavo N.},
  title = {The Holiness Composition in the Book of Exodus},
  series = {Forschungen zum Alten Testament},
  shortseries = {FAT},
  number = {2/98},
  location = {Tübingen},
  publisher = {Mohr Siebeck},
  date = {2015}
}
\end{verbatim}

\examplenocite{robinson:1952, jeremias:1967, frances:2014, reynolds+etal:2014,
witte:2015, tucker:2015}
\exampleabbreviations
\examplebibliography
\examplereferences{https://sblhs2.com/2018/05/17/series-volume-identifiers-old-new-and-concurrent-series/}

\section{Series Volume Identifiers (10 May 2018)}

\begin{verbatim}
@book{johnson:2018,
  editor = {Johnson, Sara R. and Dupertuis, Rubén R. and Shea, Christine},
  title = {Reading and Teaching Ancient Fiction},
  subtitle = {Jewish, Christian, and Greco-Roman Narratives},
  series = {Writings from the Greco-Roman World Supplement Series},
  shortseries = {WGRWSup},
  number = {11},
  location = {Atlanta},
  publisher = {SBL Press},
  date = {2018}
}

@commentary{salters:2010,
  author = {Salters, R. B.},
  title = {Lamentations},
  series = {International Critical Commentary},
  shortseries = {ICC},
  location = {London},
  publisher = {T\&T Clark},
  date = {2010}
}

@commentary{aune:1997,
  author = {Aune, David E.},
  title = {Revelation 1--11},
  series = {Word Biblical Commentary},
  shortseries = {WBC},
  number = {52A},
  location = {Nashville},
  publisher = {Nelson},
  date = {1997}
}

@commentary{aune:1998,
  author = {Aune, David E.},
  title = {Revelation 17--22},
  series = {Word Biblical Commentary},
  shortseries = {WBC},
  number = {52C},
  location = {Nashville},
  publisher = {Nelson},
  date = {1998}
}

@commentary{seebass:1993,
  author = {Seebass, Horst},
  title = {Numeri},
  subtitle = {Kapitel 1,1--10,10},
  shorttitle = {Numeri: 1,1--10,10},
  series = {Biblischer Kommentar, Altes Testament},
  shortseries = {BKAT},
  number = {4.1},
  location = {Neukirchen-Vluyn},
  publisher = {Neukirchener Verlag},
  date = {1993},
  langid = {german}
}

@mvcommentary{aune:1997-1998,
  author = {Aune, David E.},
  title = {Revelation},
  volumes = {3},
  series = {Word Biblical Commentary},
  shortseries = {WBC},
  number = {52A--C},
  location = {Nashville},
  publisher = {Nelson},
  date = {1997/1998}
}

@mvcommentary{seebass:1993-2007,
  author = {Seebass, Horst},
  title = {Numeri},
  volumes = {3},
  series = {Biblischer Kommentar, Altes Testament},
  shortseries = {BKAT},
  number = {4.1--3},
  location = {Neukirchen-Vluyn},
  publisher = {Neukirchener Verlag},
  date = {1993/2007},
  langid = {german}
}
\end{verbatim}

\examplenocite{johnson:2018, salters:2010, aune:1998, seebass:1993,
seebass:1993-2007}
\examplecite(1)[589]{aune:1997}
\examplecite(2)[589]{aune:1997}
\examplevolcite(3){1}[589]{aune:1997-1998}
\examplevolcite(4){1}[589]{aune:1997-1998}
\exampleabbreviations
\examplebibliography
\examplereferences{https://sblhs2.com/2018/05/10/series-volume-identifiers/}

\section{Electronic Journals with Individually Paginated Articles (3 May 2018)}

\begin{verbatim}
@article{oswald:2012,
  author = {Oswald, Wolfgang},
  title = {Foreign Marriages and Citizenship in Persian Period Judah},
  shorttitle = {Foreign Marriages},
  journaltitle = {Journal of Hebrew Scriptures},
  shortjournal = {JHebS},
  volume = {12},
  date = {2012},
  eid = {6},
  pages = {1-17},
  doi = {10.5508/jhs.2012.v12.a6}
}
\end{verbatim}

\examplecite(16)[3]{oswald:2012}
\examplecite(18)[3]{oswald:2012}
\exampleabbreviations
\examplebibliography
\examplereferences{https://sblhs2.com/2018/05/03/electronic-journals-with-individually-paginated-articles/}

\section{Multiple Cities of Publication (26 April 2018)}

\begin{verbatim}
@book{hamori+stokl:2018,
  author = {Hamori, Esther J. and Stökl, Jonathan},
  title = {Perchance to Dream: Dream Divination in the Bible and the Ancient Near
           East},
  series = {Ancient Near East Monographs},
  shortseries = {ANEM},
  number = {21},
  location = {Atlanta},
  publisher = {SBL Press},
  date = {2018}
}

@book{wilken:2003,
  author = {Wilken, Robert Louis},
  title = {The Christians as the Romans Saw Them},
  edition = {2},
  location =  {New Haven and London},
  publisher = {Yale University Press},
  date = {2003}
}
\end{verbatim}

\examplenocite{hamori+stokl:2018, wilken:2003}
\exampleabbreviations
\examplebibliography
\examplereferences{https://sblhs2.com/2018/04/26/multiple-cities-of-publication/}

\section{Journals Identified by Issue Number (12 April 2018)}

\begin{verbatim}
@article{miller:1984,
  author = {Miller, Jr., Patrick D.},
  title = {Meter, Parallelism, and Tropes: The Search for Poetic Style},
  journaltitle = {Journal for the Study of the Old Testament},
  shortjournal = {JSOT},
  issue = {28},
  date = {1984},
  pages = {99-106}
}

@article{stott:2005-2006,
  author = {Stott, Katherine},
  title = {Finding the Lost Book of the Law: Re-reading the Story of \mkbibquote{The
           Book of the Law} (Deuteronomy--2~Kings) in Light of Classical Literature},
  journaltitle = {Journal for the Study of the Old Testament},
  shortjournal = {JSOT},
  volume = {30},
  date = {2005/2006},
  pages = {153-169}
}

@article{roth:1959-1960,
  author = {Roth, Cecil},
  title = {The Zealots and Qumran: The Basic Issue},
  journaltitle = {Revue de Qumran},
  shortjournal = {RevQ},
  volume = {2},
  issue = {5},
  date = {1959/1960},
  pages = {81-84}
}
\end{verbatim}

\examplecite(3){miller:1984}
\examplecite(4){stott:2005-2006}
\examplecite(5){roth:1959-1960}
\exampleabbreviations
\examplebibliography
\examplereferences{https://sblhs2.com/2018/04/12/journals-identified-by-issue-number/}

\section{Modern Author Names (6 April 2018)}

\begin{verbatim}
@book{wellhausen:1883,
  author = {Wellhausen, Julius},
  title = {Prolegomena zur Geschichte Israels},
  edition = {2},
  location = {Berlin},
  publisher = {Reimer},
  date = {1883},
  langid = {german}
}

@book{scott:1989,
  author = {Scott, Bernard Brandon},
  title = {Hear Then the Parable: A Commentary on the Parables of Jesus},
  location = {Philadelphia},
  publisher = {Fortress},
  date = {1989}
}

@book{logan+wedderburn:1983,
  editor = {Logan, Alastair H. B. and Wedderburn, Alexander J. M.},
  title = {The New Testament and Gnosis: Essays in Honour of Robert McL. Wilson},
  location = {Edinburgh},
  publisher = {T\&T Clark},
  date = {1983}
}

@book{barbour:2012,
  author = {Barbour, Jennie},
  title = {The Story of Israel in the Book of Qohelet: Ecclesiastes as Cultural
           Memory},
  location = {Oxford},
  publisher = {Oxford University Press},
  date = {2012}
}

@article{grillo:2017,
  author = {Grillo, Jennie},
  title = {\mkbibquote{From a Far Country}: Daniel in Isaiah’s Babylon},
  journaltitle = {Journal of Biblical Literature},
  shortjournal = {JBL},
  volume = {136},
  date = {2017},
  pages = {363-380}
}

@misc{barbour:seealso,
  author = {Barbour, Jennie},
  title = {See also \mkbibemph{Grillo, Jennie}},
  sorttitle = {zzz}
}

@misc{grillo:seealso,
  author = {Grillo, Jennie},
  title = {See also \mkbibemph{Barbour, Jennie}},
  sortyear = {zzz}
}

@article{eilberg-schwartz:1991,
  author = {Eilberg-Schwartz, Howard},
  title = {The Problem of the Body for the People of the Book},
  journaltitle = {Journal of the History of Sexuality},
  volume = {2},
  date = {1991},
  pages = {1-24}
}

@article{schusslerfiorenza:1986,
  author = {Schüssler Fiorenza, Elisabeth},
  title = {A Feminist Critical Interpretation for Liberation: Martha and Mary; Luke
           10:38–42},
  journaltitle = {Religion and Intellectual Life},
  shortjournal = {RIL},
  volume = {3},
  date = {1986},
  pages = {21-35}
}

@incollection{trebollebarrera:2013,
  author = {family=Trebolle Barrera, given=Julio, shortfamily=Barrera},
  title = {Agreements between LXX\textsuperscript{BL}, Medieval Hebrew Readings, and
           Variants of the Aramaic, Syriac and Vulgate Versions in \mkbibemph{Kaige}
           and Non-\mkbibemph{kaige} Sections of 3–4 Reigns},
  pages = {193-206},
  booktitle = {XIV Congress of the IOSCS: Helsinki, 2010},
  editor = {Peters, Melvin K. H.},
  series = {Septuagint and Cognate Studies},
  shortseries = {SCS},
  number = {59},
  location = {Atlanta},
  publisher = {Society of Biblical Literature},
  date = {2013}
}

@book{hooks:1990,
  author = {family=hooks, given=bell},
  title = {Yearning: Race, Gender, and Cultural Politics},
  location = {Boston},
  publisher = {South End},
  date = {1990}
}
\end{verbatim}

\exampleciteauthor{wellhausen:1883}
\exampleciteauthor{scott:1989}
\exampleciteauthor{tigay:1985}
\exampleciteauthor{wellhausen:1883, scott:1989, tigay:1985}
\begin{verbcite}
  \citeauthor{eilberg-schwartz:1991} has stated ….
\end{verbcite}
\begin{verbcite}
  \citeauthor{eilberg-schwartz:1991} goes on to argue ….
\end{verbcite}
\begin{verbcite}
  \citeauthor{schusslerfiorenza:1986} has stated ….
\end{verbcite}
\begin{verbcite}
  \citeauthor{schusslerfiorenza:1986} goes on to argue ….
\end{verbcite}
\begin{verbcite}
  \citeauthor{trebollebarrera:2013} has stated ….
\end{verbcite}
\begin{verbcite}
  \citeauthor{trebollebarrera:2013} goes on to argue ….
\end{verbcite}
\begin{verbcite}
  As \citeauthor{hooks:1990} offers ….
\end{verbcite}
\examplenocite{logan+wedderburn:1983, barbour:2012, grillo:2017,
barbour:seealso, grillo:seealso}
\exampleabbreviations
\examplebibliography
\examplereferences{https://sblhs2.com/2018/04/06/modern-author-names/}

\section{Citing Journals and Magazines: Issue Numbers (22 March 2018)}

\begin{verbatim}
@article{yee:2017,
  author = {Yee, Gale A.},
  title = {\mkbibquote{He Will Take the Best of Your Fields}: Royal Feasts and Rural
           Extraction},
  journaltitle = {Journal of Biblical Literature},
  shortjournal = {JBL},
  volume = {136},
  date = {2017},
  pages = {821-838}
}

@article{cross:1999,
  author = {Cross, Frank Moore},
  title = {King Hezekiah's Seal Bears Phoenician Imagery},
  journaltitle = {Biblical Archeology Review},
  shortjournal = {BAR},
  volume = {25},
  number = {2},
  date = {1999},
  pages = {42-45, 60}
}
\end{verbatim}

\examplecite(7){yee:2017}
\examplecite(8){cross:1999}
\exampleabbreviations
\examplebibliography
\examplereferences{https://sblhs2.com/2018/03/22/citing-journals-and-magazines-issue-numbers/}

\section{Citing Smyth's \emph{Greek Grammar} (8 March 2018)}

\begin{verbatim}
@book{smyth:1956,
  shorthand = {Smyth},
  author = {Smyth, Herbert Weir},
  title = {Greek Grammar},
  editor = {Messing, Gordon M.},
  editortype = {reviser},
  location = {Cambridge},
  publisher = {Harvard University Press},
  date = {1956},
  pagination = {section},
  options = {shorthandformat=roman}
}

@book{smyth:1920,
  author = {Smyth, Herbert Weir},
  title = {A Greek Grammar for Colleges},
  location = {New York},
  publisher = {American Book Company},
  date = {1920},
  pagination = {section}
}

@book{smyth:1916,
  author = {Smyth, Herbert Weir},
  title = {A Greek Grammar for Schools and Colleges},
  location = {New York},
  publisher = {American Book Company},
  date = {1916},
  pagination = {section}
}
\end{verbatim}

\examplecite(42)[\pno 1765a]{smyth:1956}
\examplenocite{smyth:1920,smyth:1916}
\exampleabbreviations
\examplebibliography
\examplereferences{https://sblhs2.com/2018/03/08/citing-smyths-greek-grammar/}

\section{Philo of Alexandria (1 March 2018)}

\begin{verbatim}
@book{philo:cherubim,
  author = {Philo},
  title = {On the Cherubim; The Sacrifices of Abel and Cain; The Worse Attacks the
           Better; On the Posterity and Exile of Cain; On the Giants},
  translator = {Colson, F. H. and Whitaker, G. H.},
  series = {Loeb Classical Library},
  shortseries = {LCL},
  location = {Cambridge},
  publisher = {Harvard University Press},
  date = {1929}
}

@ancienttext{philo:cher,
  author = {Philo},
  title = {De cherubim},
  shorttitle = {Cher\adddot},
  translator = {Colson},
  xref = {philo:cherubim}
}

@book{philo:questionsongenesis,
  author = {Philo},
  title = {Questions on Genesis},
  translator = {Marcus, Ralph},
  series = {Loeb Classical Library},
  shortseries = {LCL},
  location = {Cambridge},
  publisher = {Harvard University Press},
  date = {1953}
}

@ancienttext{philo:QG,
  author = {Philo},
  title = {Quaestiones et solutiones in Genesin},
  shorttitle = {QG},
  xref = {philo:questionsongenesis}
}

@book{geljion+runia:2013,
  author = {Geljion, Albert C. and Runia, David T.},
  title = {Philo of Alexandria: \mkbibquote{On Cultivation}; Introduction, Translation
           and Commentary},
  shorttitle = {Philo of Alexandria: \mkbibquote{On Cultivation}},
  series = {Philo of Alexandria Commentary Series},
  shortseries = {PACS},
  number = {4},
  location = {Leiden},
  publisher = {Brill},
  date = {2013}
}

@book{wilson:2011,
  author = {Wilson, Walter T.},
  title = {Philo of Alexandria: \mkbibquote{On Virtues}; Introduction, Translation,
           and Commentary},
  shorttitle = {Philo of Alexandria: \mkbibquote{On Virtues}},
  series = {Philo of Alexandria Commentary Series},
  shortseries = {PACS},
  number = {3},
  location = {Leiden},
  publisher = {Brill},
  date = {2011}
}
\end{verbatim}

\examplecite[paren][(50)]{philo:cher}
\examplecite[paren][(1.6)]{philo:QG}
\begin{verbcite}
  As \citeauthor{philo:cher} states, “when God consorts with the soul, He makes
  what before was a woman into a virgin again” \ptranscite*[(50)]{philo:cher}.
\end{verbcite}
\examplecite(1){geljion+runia:2013}
\examplecite(2){wilson:2011}
\exampleancientsources
\examplesecondarysources
\examplebibliography
\examplereferences{https://sblhs2.com/2018/03/01/philo-of-alexandria/}

\section{Titles in Non-Latin Alphabets (22 February 2018)}

\begin{verbatim}
@book{fidler:2005,
  author = {Fidler, Ruth},
  title = {\mkbibquote{Dreams Speak Falsely?} Dream Theophanies in the Bible: Their
           Place in Ancient Israelite Faith and Traditions},
  language = {Hebrew},
  location = {Jerusalem},
  publisher = {Magnes},
  date = {2005}
}

@article{niehoff:1993,
  author = {Niehoff, Maren R.},
  title = {Associative Thinking in Rabbinic Midrash: The Example of Abraham’s and
           Sarah’s Journey to Egypt},
  language = {Hebrew},
  journaltitle = {Tarbiz},
  volume = {62},
  date = {1993},
  pages = {339–361}
}

@book{taisija:2002,
  author = {Taisija},
  title = {Акафист святому преподобному Симеону Богоприимцу: Творение игум; Таисии
           Леушинской},
  language = {langrussian},
  translatedtitle = {Akathistos for the Holy Simeon, the God-Receiver: A Work by
                     Abbess Taisija of Leušino},
  location = {Saint Petersburg},
  publisher = {Leušinskoe izdatel’stvo},
  date = {2002},
  options = {nonlatintitle}
}

@incollection{cerenc:1995,
  author = {Cerenc, Grigor},
  title = {Յայլմէ ասացեալ բան վասն խորանաց աւետարանիս},
  translatedtitle = {The Writing of Another One on the Canon-Tables of the Gospel},
  booktitle = {Խորանների մեկնություններ},
  translatedbooktitle = {Commentaries on the Canon Tables},
  editor = {Łazaryan, V.},
  location = {Erevan},
  publisher = {Sargis Xač‘enc‘},
  date = {1995},
  pages = {68-78},
  options = {nonlatintitle, nonlatinbooktitle}
}

@article{havea:1998,
  author = {Havea, Jione},
  title = {Tau lave!},
  translatedtitle = {Let’s Talk},
  journaltitle = {Pacific Journal of Theology},
  shortjournal = {PJT},
  series = {2},
  volume = {20},
  date = {1998},
  pages = {63-73}
}
\end{verbatim}

\examplecite(3){fidler:2005}
\examplecite(7){niehoff:1993}
\examplecite(12){taisija:2002}
\examplecite(15){cerenc:1995}
\examplecite(15){havea:1998}
\exampleabbreviations
\examplebibliography
\examplereferences{https://sblhs2.com/2018/02/22/titles-in-non-latin-alphabets/}

\section{Josephus (15 February 2018)}

\begin{verbatim}
@book{josephus:life;againstapion,
  author = {Josephus},
  title = {The Life; Against Apion},
  translator = {Thackeray, Henry St.\@ J.},
  series = {Loeb Classical Library},
  shortseries = {LCL},
  location = {Cambridge},
  publisher = {Harvard University Press},
  date = {1926}
}

@book{josephus:antiquitatum,
  author = {Josephus},
  title = {Antiquitatum Iudaicarum libri VI--X},
  volume = {2},
  maintitle = {Flavii Iosephi opera},
  editor = {Niese, Benedictus},
  location = {Berlin},
  publisher = {Weidmann},
  date = {1888}
}

@book{josephus:judeanantiquities,
  author = {Josephus},
  title = {Judean Antiquities 15},
  editor = {van Henten, Jan Willem},
  series = {Flavius Josephus: Translation and Commentary},
  shortseries = {FJTC},
  number = {7b},
  location = {Leiden},
  publisher = {Brill},
  date = {2014}
}

@book{worksofjosephus,
  author = {Josephus},
  title = {The Works of Flavius Josephus},
  translator = {Whiston, A. M. William},
  volumes = {2},
  location = {London},
  publisher = {Bohn},
  date = {1862}
}

@book{josephus:jewishwar,
  author = {Josephus},
  title = {The Jewish War},
  translator = {Thackery, Henry St.\@ J.},
  series = {Loeb Classical Library},
  shortseries = {LCL},
  location = {Cambridge},
  publisher = {Harvard University Press},
  date = {1927/1928}
}

@book{josephus:jewishantiquities,
  author = {Josephus},
  title = {The Jewish Antiquities},
  translator = {Thackery, Henry St.\@ J.},
  series = {Loeb Classical Library},
  shortseries = {LCL},
  location = {Cambridge},
  publisher = {Harvard University Press},
  date = {1930/1965}
}

@ancienttext{josephus:bj,
  author = {Josephus},
  title = {Bellum judaicum},
  shorttitle = {B.J\adddot},
  xref = {josephus:jewishwar}
}

@ancienttext{josephus:aj,
  author = {Josephus},
  title = {Antiquitates judaicae},
  shorttitle = {A.J\adddot},
  xref = {josephus:jewishantiquities}
}

@ancienttext{josephus:cap,
  author = {Josephus},
  title = {Contra Apionem},
  shorttitle = {C.\@ Ap\adddot},
  xref = {josephus:life;againstapion}
}
\end{verbatim}

\examplecite[paren][(3.506–521)]{josephus:bj}
\examplecite[ptrans][(2.233–235)]{josephus:aj}
\examplecite[atrans](5)[See also][(3.506–521)]{josephus:bj}
\begin{fverbcite}{8}
  \footnote{All translations of Josephus’s \citetitle*{josephus:cap} follow the
  translation in \cite{josephus:life;againstapion}.}
\end{fverbcite}
\examplenocite{josephus:antiquitatum,josephus:judeanantiquities,worksofjosephus}
\exampleancientsources
\examplesecondarysources
\examplebibliography

\section{Citing Page Numbers for Chapters and Articles (8 February 2018)}

\begin{verbatim}
@incollection{geurts:2017,
  author = {Geurts, Bart},
  title = {Presupposition and Givenness},
  booktitle = {The Oxford Handbook of Pragmatics},
  editor = {Huang, Yan},
  location = {Oxford},
  publisher = {Oxford University Press},
  date = {2017},
  pages = {180-198}
}

@article{wellhausen:1876-1877,
  author = {Wellhausen, Julius},
  title = {Die Composition des Hexateuchs},
  journaltitle = {Jahrbuch für deutsche Theologie},
  shortjournal = {JDT},
  related = {wellhausen:1876, wellhausen:1877},
  relatedtype = {multivolarticle},
  langid = {german}
}

@article{wellhausen:1876,
  volume = {21},
  date = {1876},
  pages = {392-450}
}

@article{wellhausen:1877,
  volume = {22},
  date = {1877},
  pages = {407-479}
}
\end{verbatim}

\examplecite(5){geurts:2017}
\citereset
\examplecite(5)[181]{geurts:2017}
\examplevolcite(6){21}[434]{wellhausen:1876-1877}
\examplevolcite(7){21}[434]{wellhausen:1876-1877}
\exampleabbreviations
\examplebibliography
\citereset
\begin{verbtext}
  \usepackage[style=sbl,citepages=separate]{biblatex}
\end{verbtext}
\makeatletter
{\cbx@opt@citepages@separate
 \examplecite(5)[181]{geurts:2017}
 \examplevolcite(6){21}[434]{wellhausen:1876-1877}
}
\makeatother
\examplereferences{https://sblhs2.com/2018/02/08/citing-page-numbers-for-chapters-and-articles/}

\section{Citing Text Collections 10: LCL (18 January 2018)}

\begin{verbatim}
@mvbook{augustine:confessions,
  author = {Augustine},
  title = {Confessions},
  translator = {Hammond, Carolyn J.-B.},
  volumes = {2},
  series = {Loeb Classical Library},
  shortseries = {LCL},
  location = {Cambridge},
  publisher = {Harvard University Press},
  date = {2014/2016}
}

@ancienttext{augustine:conf,
  author = {Augustine},
  title = {Confessions},
  shorttitle = {Conf\adddot},
  xref = {augustine:confessions}
}

@mvbook{tacitus:histories,
  author = {Tacitus},
  title = {The Histories and The Annals},
  translator = {Moore, Clifford H. and Jackson, John},
  volumes = {4},
  series = {Loeb Classical Library},
  shortseries = {LCL},
  location = {Cambridge},
  publisher = {Harvard University Press},
  date = {1937}
}

@ancienttext{tacitus:ann,
  author = {Tacitus},
  title = {Annales},
  shorttitle = {Ann\adddot},
  xref = {tacitus:histories}
}

@mvbook{apuleius:golden,
  author = {Apuleius},
  title = {The Golden Ass},
  editora = {Adlington, W.},
  editoratype = {translator},
  editorb = {Gaselee, S.},
  editorbtype = {reviser},
  series = {Loeb Classical Library},
  shortseries = {LCL},
  location = {London and New York},
  publisher = {Heinemann and Putnam's Sons},
  date = {1922}
}

@ancienttext{apuleius:metam,
  author = {Apuleius},
  title = {Metamorphoses},
  shorttitle = {Metam\adddot},
  translator = {Adlington and Gaselee},
  xref = {apuleius:golden}
}

@book{aristotle:metaphysics:2,
  author = {Aristotle},
  maintitle = {Metaphysics},
  shorttitle = {Metaphysics},
  volume = {2},
  title = {Books 10–14; Oeconomica; Magna Moralia},
  translator = {Tredennick, Hugh and Armstrong, G. Cyril},
  series = {Loeb Classical Library},
  shortseries = {LCL},
  location = {Cambridge},
  publisher = {Harvard University Press},
  date = {1935},
  options = {usetitle=false},
  sorttitle = {Metaphysics 2}
}

@book{augustine:cityofgod:1,
  author = {Augustine},
  maintitle = {City of God},
  shorttitle = {City of God},
  volume = {1},
  title = {Books 1–3},
  translator = {McCracken, George E.},
  series = {Loeb Classical Library},
  shortseries = {LCL},
  location = {Cambridge},
  publisher = {Harvard University Press},
  date = {1957},
  options = {usetitle=false},
  sorttitle = {City of God 1}
}

@mvbook{aristotle:metaphysics,
  author = {Aristotle},
  title = {Metaphysics},
  translator = {Tredennick, Hugh and Armstrong, G. Cyril},
  volumes = {2},
  series = {Loeb Classical Library},
  shortseries = {LCL},
  location = {Cambridge},
  publisher = {Harvard University Press},
  date = {1933/1935}
}

@mvbook{augustine:cityofgod,
  author = {Augustine},
  title = {City of God},
  translator = {McCracken, George E. and Green, William M. and Wiesen, David S. and Levine, Philip and Sanford, Eva M.},
  volumes = {7},
  series = {Loeb Classical Library},
  shortseries = {LCL},
  location = {Cambridge},
  publisher = {Harvard University Press},
  date = {1957/1972}
}

@book{augustine:selectletters,
  author = {Augustine},
  title = {Select Letters},
  translator = {Baxter, James Houston},
  series = {Loeb Classical Library},
  shortseries = {LCL},
  location = {Cambridge},
  publisher = {Harvard University Press},
  date = {1930}
}

@book{aristotle:poetics:book,
  author = {Aristotle},
  title = {Poetics},
  editor = {Halliwell, Stephen},
  translator = {Halliwell, Stephen},
  related = {longinus:sublime:related,demetrius:style:related}
}

@book{longinus:sublime:related,
  author = {Longinus},
  title = {On the Sublime},
  editora = {Fyfe, W. H.},
  editoratype = {translator},
  editorb = {Russell, Donald},
  editorbtype = {reviser},
}

@book{demetrius:style:related,
  author = {Demetrius},
  title = {On Style},
  editor = {Innes, Doreen C.},
  translator = {Innes, Doreen C.},
  series = {Loeb Classical Library},
  shortseries = {LCL},
  location = {Cambridge},
  publisher = {Harvard University Press},
  date = {1995}
}

@book{aristotle:poetics:book:short,
  author = {Aristotle},
  title = {Poetics},
  options = {skipbib}
}

@book{longinus:sublime:book,
  author = {Longinus},
  title = {On the Sublime},
  related = {aristotle:poetics:book:short},
  relatedtype = {see},
  execute = {\nocite{aristotle:poetics:book,aristotle:poetics:book:short}}
}

@book{demetrius:style:book,
  author = {Demetrius},
  title = {On Style},
  related = {aristotle:poetics:book:short},
  relatedtype = {see},
  execute = {\nocite{aristotle:poetics:book,aristotle:poetics:book:short}}
}

@ancienttext{aristotle:poetics,
  author = {Aristotle},
  title = {Poetics},
  translator = {Halliwell, Stephen},
  xref = {aristotle:poetics:book},
  execute = {\nocite{aristotle:poetics:book}}
}

@ancienttext{longinus:sublime,
  author = {Longinus},
  title = {On the Sublime},
  translator = {Fyfe, W. H.},
  xref = {longinus:sublime:book},
  execute = {\nocite{aristotle:poetics:book,longinus:sublime:book}},
}

@ancienttext{demetrius:style,
  author = {Demetrius},
  title = {On Style},
  translator = {Innes, Doreen C.},
  xref = {demetrius:style:book},
  execute = {\nocite{aristotle:poetics:book,demetrius:style:book}},
}
\end{verbatim}

\examplecite[paren][(8.29)]{augustine:conf}
\begin{verbcite}
  \cite[(15)]{tacitus:ann} details the activities of Nero.
\end{verbcite}
\examplecite(4)[For a similar example, see][(11.2)]{apuleius:metam}
\begin{verbcite}
  “These were my words, and in grief of heart I wept bitterly”
  \ptranscite[(8.29)]{augustine:conf}.
\end{verbcite}
\begin{fverbcite}{4}
  \footnote{“Thou, which dost luminate all the cities of the earth by Thy
  feminine light; Thou, which nourishes all the seeds of the world by Thy Damp
  heat, giving Thy Changing light according to the wanderings, near or far, of
  the sun” \ptranscite[11.2]{apuleius:metam}.}
\end{fverbcite}
\begin{fverbcite}{3}
\footnote{Translations from the \citetitle{apuleius:golden} follow that of
\cite{apuleius:golden}}.
\end{fverbcite}
\cite{apuleius:golden}.
\examplenocite{aristotle:metaphysics:2, augustine:cityofgod:1,
aristotle:metaphysics, augustine:cityofgod, augustine:selectletters,
aristotle:poetics, longinus:sublime, demetrius:style,
aristotle:poetics:book:short}
\exampleancientsources
\examplesecondarysources
\examplebibliography
\examplereferences{https://sblhs2.com/2018/01/18/citing-text-collections-10-lcl/}

\section{Greek Magical Papyri (13 October 2017)}

\begin{verbatim}
@book{betz:1996,
  editor = {Betz, Hans Dieter},
  title = {The Greek Magical Papyri in Translation, Including the Demotic Spells},
  edition = {2},
  location = {Chicago},
  publisher = {University of Chicago Press},
  date = {1996}
}

@mvbook{preisendaz:1973-1974,
  shorthand = {PGM},
  editor = {Preisendaz, Karl},
  translator = {Preisendaz, Karl},
  title = {Papyri Graecae Magicae: Die griechischen Zauberpapyri},
  edition = {2},
  volumes = {3},
  location = {Stuttgart},
  publisher = {Teubner},
  date = {1973/1974}
}

@book{PDM:betz,
  shorthand = {PDM},
  title = {Papyri Demoticae Magicae},
  relatedstring = {Demotic texts in \citeshorthand{preisendaz:1973-1974} corpus as
                   collated in},
  related = {betz:1996},
  relatedoptions = {skipbiblist}
}

@ancienttext{PGM,
  title = {\citeshorthand{preisendaz:1973-1974}},
  xref = {betz:1996}
}

@ancienttext{PDM,
  title = {\citeshorthand{PDM:betz}},
  xref = {PDM:betz}
}
\end{verbatim}

\examplecite[][(III. 410-424)]{PGM}
\examplecite[][(xiv. 554-562)]{PDM}
\examplecite[][The prayer of deliverance in][(I. 195-222)]{PGM}
\examplecite[paren][(IV. 1275-1322; IV. 1331-1389; VII. 686-702)]{PGM}
\begin{verbcite}
  A fourth- or fifth-century prayer of deliverance begins as follows: “This,
  then, is the prayer of deliverance for the first-begotten and first-born
  god: ‘I call upon you, lord. Hear me, holy god who rest among the holy ones,
  at whose side the Glorious Ones stand continually. I call upon you’”
  \parencite[(\pnfmt{I. 195-200} \mkbibbrackets{O'Neil in Betz})]{PGM}
\end{verbcite}
\nocite{preisendaz:1973-1974}
\exampleabbreviations
\examplebibliography
\examplereferences{https://sblhs2.com/2017/10/13/greek-magical-papyri/}

\section{Vetus Latina (VLB) (21 September 2017)}

\begin{verbatim}
@book{VLB:26.2,
  editor = {Gryson, Roger},
  title = {Apocalypsis Johannis},
  series = {Vetus Latina Beuron},
  shortseries = {VLB},
  number = {26.2},
  location = {Freiburg im Breisgau},
  publisher = {Herder},
  date = {2000/2003}
}

@book{VLB:26.2.8,
  editor = {Gryson, Roger},
  title = {Apocalypsis Johannis},
  series = {Vetus Latina Beuron},
  shortseries = {VLB},
  number = {26.2.8},
  location = {Freiburg im Breisgau},
  publisher = {Herder},
  date = {2003}
}
\end{verbatim}

\examplecite(16){VLB:26.2}
\examplecite(16){VLB:26.2.8}
\citereset
\examplecite(16)[625]{VLB:26.2}
\citereset
\examplecite(16)[625 \mkbibparens{upper}]{VLB:26.2}
\examplecite(18)[625 \mkbibparens{middle}]{VLB:26.2}
\examplecite(20)[625 \mkbibparens{lower}]{VLB:26.2}
\exampleabbreviations
\examplebibliography
\examplereferences{https://sblhs2.com/2017/09/21/vetus-latina-vlb/}

\section{Studia Patristica (14 September 2017)}

\begin{verbatim}
@inproceedings{husek:2010,
  author = {Hušek, Vít},
  title = {Human Freedom according to the Earliest Latin Commentaries on Paul’s Letters},
  series = {Studia Patristica},
  shortseries = {StPatr},
  number = {44},
  date = {2010},
  pages = {385-390}
}

@inproceedings{tkacz:2010,
  author = {Tkacz, Catherine Brown},
  title = {Esther as a Type of Christ and the Jewish Celebration of Purim},
  series = {Studia Patristica},
  shortseries = {StPatr},
  number = {44},
  date = {2010},
  pages = {183-187}
}
\end{verbatim}

\examplecite(60)[See futher][]{husek:2010}
\examplecite(62){tkacz:2010}
\exampleabbreviations
\examplebibliography
\examplereferences{https://sblhs2.com/2017/09/14/studia-patristica/}

\section{Citing Text Collections 9: Kitchen’s \emph{Ramesside Inscriptions} (2 September 2017)}

\begin{verbatim}
@mvbook{KRI,
  shorthand = {KRI},
  author = {Kitchen, K. A.},
  title = {Ramesside Inscriptions, Historical and Biographical},
  volumes = {8},
  location = {Oxford},
  publisher = {Blackwell},
  date = {1969/1990}
}

@mvbook{RITA,
  shorthand = {RITA},
  author = {Kitchen, K. A.},
  title = {Ramesside Inscriptions Translated and Annotated: Translations},
  volumes = {7},
  location = {Oxford and Chichester},
  publisher = {Blackwell and Wiley-Blackwell},
  date = {1993/2014}
}

@mvbook{RITANC,
  shorthand = {RITANC},
  author = {Kitchen, K. A. and Davies, Benedict G.},
  title = {Ramesside Inscriptions Translated and Annotated: Notes and Comments},
  volumes = {4},
  location = {Oxford and Chichester},
  publisher = {Blackwell and Wiley-Blackwell},
  date = {1993/2014}
}
\end{verbatim}

\begin{verbcite}
  \citeshortauthor{KRI} writes in \volcite{1}{KRI}: “Citations of this work
  should be as KRI, as volume, page and line” \pvolcite{1}[xxxi]{KRI}. Thus,
  one would cite line 10 of the Merenptah Lachish bowl as \volcite{4}[39, line
  10]{KRI}.
\end{verbcite}
\examplenocite{RITA,RITANC}
\exampleabbreviations
\examplereferences{https://sblhs2.com/2017/09/02/citing-text-collections-9-kitchens-ramesside-inscriptions/}

\section{Citing Text Collections 8: \emph{NTApoc} (17 August 2017)}

\begin{verbatim}
@mvbook{NTApoc,
  shorthand = {NTApoc},
  editor = {Schneemelcher, Wilhelm},
  title = {New Testament Apocrypha},
  volumes = {2},
  editora = {Wilson, Robert McL.},
  editorastring = {Rev.\@ ed.\@ English trans.\@ ed\adddot},
  location = {Cambridge and Louisville},
  publisher = {Clarke and Westminster John Knox},
  date = {2003}
}

@ancienttext{actsandrew,
  title = {Acts of Andrew},
  related = {NTApoc},
  volume = {2},
  pages = {101-151}
}

@ancienttext{actsandrew,
  title = {Acts of Andrew},
  related = {NTApoc},
  volume = {2},
  pages = {101-151}
}

@ancienttext{murfrag,
  title = {Muratorian Fragment},
  shorttitle = {Mur.\@ Frag\adddot},
  translator = {Schneemelcher, Wilhelm},
  related = {NTApoc},
  volume = {1}
}

@ancienttext{actsjohn,
  title = {Acts John},
  related = {NTApoc},
  volume = {2},
  options={parens}
}
\end{verbatim}

\begin{fverbcite}{13}
  \footnote{For the \citetitle{actsandrew}, see \citecollection{actsandrew}.}
\end{fverbcite}
\begin{fverbcite}{22}
  \footnote{The Muratorian Fragment offers the following about Acts of the
  Apostles: “But the acts of all apostles are written in one book. For the
  ‘most excellent Theophilus’ Luke summarises the several things that in his
  own presence have come to pass, as also by the omission of the passion of
  Peter he makes quite clear, and equally by (the omission) of the journey of
  Paul, who from the city (of Rome) proceeded to Spain”
  \ptranscite[(\linesno~34-39)35]{murfrag}.}
\end{fverbcite}
\begin{fverbcite}{22}
  \autocite[For Lycomedes’s lament over his wife Cleopatra’s paralysis,
  see][(20)173]{actsjohn}
\end{fverbcite}
\exampleancientsources
\examplesecondarysources
\examplereferences{https://sblhs2.com/2017/08/17/citing-text-collections-8-_ntapoc-_/}

\section{Canon Muratori/The Muratorian Fragment (10 August 2017)}

\begin{verbatim}
@ancienttext{murfrag,
  title = {Muratorian Fragment},
  shorttitle = {Mur.\@ Frag\adddot},
}
\end{verbatim}

\begin{verbcite}
  The Epistle of Jude is referenced in \cite[(68)]{murfrag}.
\end{verbcite}
\begin{verbcite}
  \parencite[(\lineno~68)]{murfrag}
\end{verbcite}
\begin{verbcite}
  line~68 of the \citetitle*{murfrag}
\end{verbcite}
\exampleancientsources
\examplereferences{https://sblhs2.com/2017/08/10/canon-muratorithe-muratorian-fragment/}

\section{Corpus Caesarianum in \emph{BNP} (3 August 2017)}

\begin{verbatim}
@mvreference{BNP,
  shorthand = {BNP},
  editor = {Cancik, Hubert},
  title = {Brill’s New Pauly: Encyclopaedia of the Ancient World},
  volumes = {22},
  location = {Leiden},
  publisher = {Brill},
  date = {2002/2011}
}

@xdata{BNPonline,
  xref = {BNP},
  url = {http://referenceworks.brillonline.com/browse/brill-s-new-pauly}
}

@inreference{elabbadi:alexandria:history,
  author = {El-Abbadi, Mostafa},
  title = {Alexandria: History},
  xref = {BNP},
  volume = {A1},
  pages = {82-85}
}

@inreference{rupke:corpus:caesarianum,
  author = {Rüpke, Jörg},
  title = {Corpus Caesarianum},
  xdata = {BNPonline}
}
\end{verbatim}

\examplecite{elabbadi:alexandria:history}
\examplecite{rupke:corpus:caesarianum}
\exampleabbreviations
\examplebibliography
\examplereferences{https://sblhs2.com/2017/08/03/corpus-caesarianum-in-bnp/}

\section{Corpus Caesarianum: Anatomy of an Error (27 July 2017)}

\begin{verbatim}
@ancienttext{aulusgellius:noctatt,
  author = {{Aulus Gellius}},
  title = {Noctes attica},
  shorttitle = {Noct.\@ att\adddot}
}

@ancienttext{bellafr,
  title = {Bellum africum},
  shorttitle = {Bell.\@ afr\adddot}
}

@ancienttext{bellalex,
  title = {Bellum alexandrinum},
  shorttitle = {Bell.\@ alex\adddot}
}

@ancienttext{bellhisp,
  title = {Bellum hispaniense},
  shorttitle = {Bell.\@ hisp\adddot}
}

@ancienttext{bion:epitaphadon,
  author = {Bion},
  title = {Epitaphius Adonis},
  shorttitle = {Epitaph.\@ Adon\adddot}
}

@ancienttext{bion:epitaphachil,
  author = {Bion},
  title = {Epithalamium Achillis et Deidameiae},
  shorttitle = {Epith.\@ Achil\adddot}
}

@ancienttext{caesar:bellciv,
  author = {Caesar},
  title = {Bellum civile},
  shorttitle = {Bell.\@ civ.\adddot}
}

@ancienttext{caesar:bellgall,
  author = {Caesar},
  title = {Bellum gallicum},
  shorttitle = {Bell.\@ gall\adddot}
}
\end{verbatim}

\examplenocite{aulusgellius:noctatt, bellafr, bellalex, bellhisp,
bion:epitaphadon, bion:epitaphachil, caesar:bellciv, caesar:bellgall}
\exampleabbreviations
\examplereferences{https://sblhs2.com/2017/07/27/corpus-caesarianum-anatomy-of-an-error/}

\section{Citing Text Collections 7: \emph{ARAB}}

\begin{verbatim}
@mvbook{ARAB,
  shorthand = {ARAB},
  author = {Luckenbill, Daniel David},
  title = {Ancient Records of Assyria and Babylonia},
  volumes = {2},
  location = {Chicago},
  publisher = {University of Chicago Press},
  date = {1926/1927},
  pagination = {paragraph}
}
\end{verbatim}

\begin{fverbcite}{28}
  \footnote{For a translation of a bull inscription from Sennacherib’s palace,
  see \volcite{2}[407-416]{ARAB}.}
\end{fverbcite}
\begin{fverbcite}{5}
  \footnote{The Prism Inscription of Tiglath-pileser I concludes with a series
  of blessings and curses reminiscent of those found within the Hebrew Bible
  \pvolcite[see]{1}[265-266]{ARAB}.}
\end{fverbcite}
\exampleabbreviations
\examplereferences{https://sblhs2.com/2017/07/20/citing-text-collections-7-arab/}
\end{document}
