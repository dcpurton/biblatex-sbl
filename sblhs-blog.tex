\DocumentMetadata{lang=en}
\documentclass[a4paper]{article}
\usepackage{microtype}
\usepackage{parskip}
\usepackage{xcolor}
\usepackage{fancyvrb,fvextra}
\usepackage{enverb}
\usepackage{tocloft}
\setlength{\cftbeforesecskip}{2pt plus 0.5pt}
\setcounter{tocdepth}{1}
\RecustomVerbatimEnvironment{verbatim}{Verbatim}{backgroundcolor=black!15,fontsize=\small}
\setcounter{secnumdepth}{0}
\usepackage[american, bidi=basic]{babel}
\babelprovide[import, onchar=ids]{polytonicgreek}
\babelprovide[import, onchar=ids]{russian}
\babelprovide[import, onchar=ids fonts]{armenian}
\babelprovide[import, onchar=ids fonts]{hebrew}
\babelprovide[import, onchar=ids fonts]{syriac}
\babelfont{rm}{Brill}
\babelfont{tt}[Scale=MatchLowercase]{DejaVu Sans Mono}
\babelfont[armenian]{rm}[Scale=0.85]{Noto Serif Armenian}
\babelfont[hebrew]{rm}[Scale=MatchLowercase, Contextuals=Alternate]{SBL Hebrew}
\babelfont[syriac]{rm}[Scale=MatchLowercase]{Estrangelo Edessa}
\babelfont[hebrew]{tt}[Scale=MatchLowercase]{Miriam Libre Medium}
\babelfont[syriac]{tt}[Scale=0.85]{Noto Sans Syriac}
\usepackage{csquotes}
\usepackage[colorlinks]{hyperref}
\usepackage[style=sbl, refsection=section, locallabelwidth]{biblatex}
\addbibresource{biblatex-sbl.bib}

\makeatletter

\colorlet{cmdcolour}{black!65}

\newcommand*{\pkg}[1]{\textsf{#1}}

\ExplSyntaxOn
% Thanks to https://tex.stackexchange.com/a/44444/87678
\cs_new_protected:Npn \sblhsblog_verb_print:n #1
  {
    \tl_set:Nn \l_tmpa_tl {#1}
    \regex_replace_all:nnN { . } { \c{string} \0 } \l_tmpa_tl
    \tl_set:Nx \l_tmpb_tl { \l_tmpa_tl }
    \tl_use:N \l_tmpb_tl
  }
\cs_set_eq:NN \verbprint \sblhsblog_verb_print:n
\ExplSyntaxOff

% hack to include a space after a control sequence that would normally be
% removed by \verbprint above
\usepackage{newunicodechar}
\newunicodechar{ }{\relax}% this is U+00A0

% redefine hyperlink anchors to be independent of \iffootnote
\DeclareFieldFormat{bibhyperlink}{%
  \bibhyperlink{\cbx@resetcount:\thefield{entrykey}}{#1}}
\DeclareFieldFormat{bibhypertarget}{%
  \ifboolexpr{
    not test {\iffieldundef{crossref}}
    and
    not test {\ifcrossrefseen}
  }
    {\bibhypertarget{\cbx@resetcount:\thefield{crossref}}{}}
    {}%
  \bibhypertarget{\cbx@resetcount:\thefield{entrykey}}{#1}}
\DeclareFieldFormat{shorttitlelink}{%
  \bibhyperlink{\cbx@resetcount:\thefield{entrykey}}{#1}}

% Redefine \smartcite so autocite prints inline with a period.
\DeclareCiteCommand{\smartcite}[\iffootnote\mkbibparens\bibfootnotewrapper]
  {\usebibmacro{prenote}}
  {\usebibmacro{cite}}
  {\multicitedelim}
  {\usebibmacro{cite:postnote}}

\DeclareCiteCommand*{\smartcite}[\iffootnote\mkbibparens\bibfootnotewrapper]
  {\usebibmacro{prenote}}
  {\usebibmacro{cite:star}%
   \usebibmacro{cite}}
  {\multicitedelim}
  {\usebibmacro{cite:postnote}}

\NewDocumentCommand{\examplenobreak}{}{%
  \par
  \@afterheading
}

\NewDocumentCommand{\exampleciteauthor}{m}{%
  \par
  \textcolor{cmdcolour}{\texttt{\textbackslash citeauthor\{#1\}}}\par
  \@afterheading
  \citeauthor{#1}%
}

% \examplecite(<footnote number>){<citetype>}[<prenote>][<postnote>]{entryid}
% \examplecite*(<footnote number>){<citetype>}[<prenote>][<postnote>]{entryid}
\NewDocumentCommand{\examplecite}{sO{auto}d()oom}{%
  \par
  \textcolor{cmdcolour}{%
    \texttt{%
      \textbackslash #2cite%
      \IfBooleanT{#1}{*}%
      \IfNoValueF{#4}{[\verbprint{#4}]}%
      \IfNoValueF{#5}{[\verbprint{#5}]}%
      \{#6\}}}\par
  \@afterheading
  \IfNoValueF{#3}{\quad #3.\space\strut}%
  \IfBooleanTF{#1}
    {\printexamplecite*{#2cite}{#4}{#5}{#6}}
    {\printexamplecite{#2cite}{#4}{#5}{#6}}}

% \examplevolcite(<footnote number>){<citetype>}[<prenote>]{<volume>}[<postnote>]{entryid}
% \examplevolcite*(<footnote number>){<citetype>}[<prenote>]{<volume>}[<postnote>]{entryid}
\NewDocumentCommand{\examplevolcite}{sO{a}d()omom}{%
  \par
  \textcolor{cmdcolour}{%
    \texttt{%
      \textbackslash #2volcite%
      \IfBooleanT{#1}{*}%
      \IfNoValueF{#4}{[\verbprint{#4}]}%
      \{#5\}%
      \IfNoValueF{#6}{[\verbprint{#6}]}%
      \{#7\}}}\par
  \@afterheading
  \IfNoValueF{#3}{\quad #3.\space\strut}%
  \IfBooleanTF{#1}
    {\printexamplevolcite*{#2volcite}{#4}{#5}{#6}{#7}}
    {\printexamplevolcite{#2volcite}{#4}{#5}{#6}{#7}}}

\NewDocumentCommand{\printexamplecite}{smmmm}{%
  \IfNoValueTF{#4}
    {\IfNoValueTF{#3}
       {\IfBooleanTF{#1}
          {\csname #2\endcsname*{#5}}
          {\csname #2\endcsname{#5}}}
       {\IfBooleanTF{#1}
          {\csname #2\endcsname*[#3]{#5}}
          {\csname #2\endcsname[#3]{#5}}}}
    {\IfNoValueTF{#3}
       {\IfBooleanTF{#1}
          {\csname #2\endcsname*[#3][]{#5}}
          {\csname #2\endcsname[#3][]{#5}}}
       {\IfBooleanTF{#1}
          {\csname #2\endcsname*[#3][#4]{#5}}
          {\csname #2\endcsname[#3][#4]{#5}}}}}

\NewDocumentCommand{\printexamplevolcite}{smmmmm}{%
  \IfNoValueTF{#3}
    {\IfNoValueTF{#5}
       {\IfBooleanTF{#1}
          {\csname #2\endcsname*{#4}{#6}}
          {\csname #2\endcsname{#4}{#6}}}
       {\IfBooleanTF{#1}
          {\csname #2\endcsname*{#4}[#5]{#6}}
          {\csname #2\endcsname{#4}[#5]{#6}}}}
    {\IfNoValueTF{#5}
       {\IfBooleanTF{#1}
          {\csname #2\endcsname*[#3]{#4}{#6}}
          {\csname #2\endcsname[#3]{#4}{#6}}}
       {\IfBooleanTF{#1}
          {\csname #2\endcsname*[#3]{#4}[#5]{#6}}
          {\csname #2\endcsname[#3]{#4}[#5]{#6}}}}}

\NewDocumentEnvironment{verbcite}{}{%
  \enverb{}%
}{%
  \par
  \color{cmdcolour}%
  \enverbListing{Verbatim}{}%
  \par
  \@afterheading
  \normalcolor
  \enverbExecute
}

\NewDocumentEnvironment{fverbcite}{m}{%
  \enverb{}%
}{%
  \par
  \color{cmdcolour}%
  \enverbListing{Verbatim}{}%
  \par
  \@afterheading
  \normalcolor
  \renewcommand\footnote[1]{\toggletrue{blx@footnote}##1}%
  \quad #1.\space\strut\enverbExecute
}

\NewDocumentEnvironment{verbtext}{}{%
  \enverb{}%
}{%
  \par
  \color{cmdcolour}%
  \enverbListing{Verbatim}{}%
  \normalcolor
  \par
}

\NewDocumentCommand{\exampleabbreviations}{}{%
  \par
  \textcolor{cmdcolour}{\texttt{\textbackslash printbiblist\{abbreviations\}}}
  \@afterheading
  \printbiblist[heading=none]{abbreviations}}

\NewDocumentCommand{\exampleancientsources}{}{%
  \par
  \textcolor{cmdcolour}{%
    \texttt{\textbackslash printbiblist[heading=subbibliography, title=Ancient Sources,\\
      \strut\quad type=ancienttext]\{abbreviations\}}}
  \@afterheading
  \printbiblist[heading=subbibliography, title=Ancient Sources,
    type=ancienttext]{abbreviations}}

\NewDocumentCommand{\examplesecondarysources}{}{%
  \par
  \textcolor{cmdcolour}{%
    \texttt{\textbackslash printbiblist[heading=subbibliography, title=Secondary Sources,\\
      \strut\quad nottype=ancienttext]\{abbreviations\}}}
  \@afterheading
  \printbiblist[heading=subbibliography, title=Secondary Sources,
    nottype=ancienttext, nottype=abbreviation]{abbreviations}}

\NewDocumentCommand{\examplesigla}{}{%
  \par
  \textcolor{cmdcolour}{%
    \texttt{\textbackslash printbiblist[heading=subbibliography, title=Sigla
      and Grammatical \\
      \strut\quad Abbreviations, type=abbreviation]\{abbreviations\}}}
  \@afterheading
  \printbiblist[heading=subbibliography, title=Sigla and Grammatical
    Abbreviations, type=abbreviation]{abbreviations}}

\NewDocumentCommand{\examplebibliography}{}{%
  \par
  \textcolor{cmdcolour}{\texttt{\textbackslash printbibliography}}
  \@afterheading
  \printbibliography[heading=none]}

\NewDocumentCommand{\examplereferences}{m}{%
  \subsection*{References}
  \url{#1}}
\makeatother

\begin{document}
\title{SBL Handbook of Style}
\author{Explanations, Clarifications, and Expansions}
\date{SBL Press}
\maketitle

\begin{abstract}
  This document contains all note and bibliography examples from the
  \href{https://sblhs2.com/}{SBL Handbook of Style Blog}. Each section shows
  what the various bib entries should look like and how to cite them.
\end{abstract}

\tableofcontents

\section{\emph{Brill Dictionary of Ancient Greek} (20 April 2021)}

\begin{verbatim}
@reference{MGS,
  shorthand = {MGS},
  author = {Montanari, Franco},
  title = {The Brill Dictionary of Ancient Greek},
  editor = {Goh, Madeleine and Schroeder, Chad},
  location = {Leiden},
  publisher = {Brill},
  date = {2015},
  options = {shorthandformat=roman}
}
\end{verbatim}

\examplecite(1){MGS}
\exampleabbreviations
\examplereferences{https://sblhs2.com/2021/04/20/brill-dictionary-of-ancient-greek/}

\section{Snippet Text Collections (28 March 2019)}

\begin{verbatim}
@series{ACCS,
  series = {Ancient Christian Commentary on Scripture},
  shortseries = {ACCS}
}

@book{edwards:1999,
  editor = {Edwards, Mark J.},
  title = {Galatians, Ephesians, Philippians},
  series = {\citeseries{ACCS} New Testament},
  number = {8},
  location = {Downers Grove, IL},
  publisher = {InterVarsity Press},
  date = {1999}
}

@ancienttext{victorinus:ephesians,
  author = {{Marius Victorinus}},
  title = {Epistle to the Ephesians},
  xref = {edwards:1999},
  xrefstring = {quoted in}
}

@ancienttext{theodoret:galatians,
  author = {Theodoret},
  title = {Epistle to the Galatians},
  xref = {edwards:1999},
  xrefstring = {quoted in}
}
\end{verbatim}

\examplecite(4)[(1.2.12)129]{victorinus:ephesians}
\examplecite(8)[(5.13)77]{theodoret:galatians}
\nocite{ACCS}
\exampleabbreviations
\examplebibliography
\examplereferences{https://sblhs2.com/2019/03/28/snippet-text-collections/}

\section{Update: Citing an Untitled Introduction (18 January 2019)}

\begin{verbatim}
@suppbook{boers:1996,
  author = {Boers, Hendrikus},
  title = {introduction},
  booktitle = {How to Read the New Testament},
  booksubtitle = {An Introduction to Linguistic and Historical-Critical Methodology},
  bookauthor = {Egger, Wilhelm},
  translator = {Heinegg, Peter},
  location = {Peabody, MA},
  publisher = {Hendrickson},
  date = {1996}
}
\end{verbatim}

\examplecite(15){boers:1996}
\examplebibliography
\examplereferences{https://sblhs2.com/2019/01/18/update-citing-an-untitled-introduction/}

\section{Citing a Chapter from a Single-Authored Work with Editors (10 January 2019)}

\begin{verbatim}
@book{younger:2016,
  author = {Younger, Jr., K. Lawson},
  title = {A Political History of the Arameans},
  subtitle = {From Their Origins to the End of Their Polities},
  series = {Archaeology and Biblical Studies},
  shortseries = {ABS},
  number = {13},
  location = {Atlanta},
  publisher = {SBL Press},
  date = {2016}
}

@inbook{younger:origins:2016,
  author = {Younger, Jr., K. Lawson},
  title = {The Origins of the Arameans},
  pages = {35-107},
  crossref = {younger:2016}
}

@book{matassa:2018,
  author = {Matassa, Lidia D.},
  title = {Invention of the First-Century Synagogue},
  editor = {Silverman, Jason M. and Watson, J. Murray},
  series = {Ancient Near East Monographs},
  shortseries = {ANEM},
  number = {22},
  location = {Atlanta},
  publisher = {SBL Press},
  date = {2018}
}

@inbook{matassa:delos:2018,
  author = {Matassa, Lidia D.},
  title = {Delos},
  pages = {37-77},
  crossref = {matassa:2018}
}
\end{verbatim}

\examplecite(16)[35-107]{younger:origins:2016}
\examplecite(12){matassa:delos:2018}
\citereset
\examplecite(12)[37-77]{matassa:2018}
\exampleabbreviations
\examplebibliography
\examplereferences{https://sblhs2.com/2019/01/10/citing-a-chapter-from-a-single-authored-work-with-editors/}

\section{Citing Reference Works 11: Cambridge History of Christianity (23 August 2018)}

\begin{verbatim}
@collection{CHC1,
 editor = {Mitchell, Margaret M. and Young, Frances M.},
 title = {Origins to Constantine},
 series = {Cambridge History of Christianity},
 shortseries = {CHC},
 number = {1},
 location = {Cambridge},
 publisher = {Cambridge University Press},
 date = {2006}
}

@collection{CHC2,
 editor = {Casiday, Augustine and Norris, Frederick W.},
 title = {Constantine to c.~600},
 series = {Cambridge History of Christianity},
 shortseries = {CHC},
 number = {2},
 location = {Cambridge},
 publisher = {Cambridge University Press},
 date = {2007}
}

@incollection{marcus:2006,
  author = {Marcus, Joel},
  title = {Jewish Christianity},
  pages = {87-102},
  crossref = {CHC1}
}

@incollection{freyne:2006,
  author = {Freyne, Sean},
  title = {Galilee and Judaea in the First Century},
  pages = {37-52},
  crossref = {CHC1}
}

@incollection{vandam:2007,
  author = {Van Dam, Raymond},
  title = {Bishops and Society},
  pages = {343-366},
  crossref = {CHC2}
}

@incollection{lohr:2007,
  author = {Löhr, Winrich},
  title = {Western Christanities},
  pages = {9-51},
  crossref = {CHC2}
}
\end{verbatim}

\examplecite(22){marcus:2006}
\examplecite(23){freyne:2006}
\examplecite(23){vandam:2007}
\examplecite(23){lohr:2007}
\exampleabbreviations
\examplebibliography
\examplereferences{https://sblhs2.com/2018/08/23/citing-reference-works-11-cambridge-history-of-christianity/}

\section{Special Footnotes (28 June 2018)}

\begin{verbatim}
@inbook{moore:2017,
  author = {Moore, Stephen D.},
  title = {Why the Johannine Jesus Weeps at the Tomb of Lazarus},
  booktitle = {Mixed Feelings and Vexed Passions: Exploring Emotions in Biblical
               Literature},
  editor = {Spencer, F. Scott},
  series = {Resources for Biblical Study},
  shortseries = {RBS},
  location = {Atlanta},
  publisher = {SBL Press},
  date = {2017}
}
\end{verbatim}

\texttt{An earlier version of this essay appears as \textbackslash
cite*\{moore:2017\}. Reused here\\ with permission.}

An earlier version of this essay appears as \cite*{moore:2017}. Reused here
with permission.

\exampleabbreviations
\examplebibliography
\examplereferences{https://sblhs2.com/2018/06/28/special-footnotes/}

\section{Abbreviations Lists (24 May 2018)}

\begin{verbatim}
@series{AB,
  series = {Anchor Bible},
  shortseries = {AB}
}

@mvreference{ABD,
  shorthand = {ABD},
  editor = {Freedman, David Noel},
  title = {Anchor Bible Dictionary},
  volumes = {6},
  location = {New York},
  publisher = {Doubleday},
  date = {1992},
  pagination = {subverbo}
}

@ancienttext{philo:abr,
  author = {Philo},
  title = {De Abrahamo},
  shorttitle = {Abr\adddot},
}

@ancienttext{philo:agr,
  author = {Philo},
  title = {De agricultura},
  shorttitle = {Agr\adddot},
}

@ancienttext{graniuslicinianus:ann,
  author = {{Granius Licinianus}},
  title = {Annales},
  shorttitle = {Ann\adddot}
}

@mvbook{tacitus:histories,
  author = {Tacitus},
  title = {The Histories and The Annals},
  translator = {Moore, Clifford H. and Jackson, John},
  volumes = {4},
  series = {Loeb Classical Library},
  shortseries = {LCL},
  location = {Cambridge},
  publisher = {Harvard University Press},
  date = {1937}
}

@ancienttext{tacitus:ann,
  author = {Tacitus},
  title = {Annales},
  shorttitle = {Ann\adddot},
  xref = {tacitus:histories}
}

@journal{AJSL,
  journaltitle = {American Journal of Semitic Languages and Literature},
  shortjournal = {AJSL}
}

@journal{atlantis,
  journaltitle = {Atlantis: Journal of the Spanish Association of Anglo-American
                  Studies},
  shortjournal = {Atlantis}
}

@series{AzTh,
  series = {Arbeiten zur Theologie},
  shortseries = {AzTh}
}

@journal{BibInt,
  journaltitle = {Biblical Interpretation},
  shortjournal = {BibInt}
}

@series{BibIntSeries,
  series = {Biblical Interpretation Series},
  shortseries = {BibInt}
}

@journal{BSac,
  journaltitle = {Bibliotheca Sacra},
  shortjournal = {BSac}
}

@series{JSOTSup,
  series = {Journal for the Study of the Old Testament Supplement Series},
  shortseries = {JSOTSup}
}

@ancienttext{justinmartyr:1apol,
  author = {{Justin Martyr}},
  title = {First Apology},
  shorttitle = {1~Apol\adddot}
}

@ancienttext{1en,
  title = {1~Enoch},
  shorttitle = {1~En\adddot}
}

@ancienttext{1QM,
  title = {War Scroll},
  shorttitle = {1QM}
}

@ancienttext{4QpNah,
  title = {Pesher Nahum},
  shorttitle = {4QpNah}
}

@ancienttext{livy:aburbe,
  author = {Livy},
  title = {Ab urbe condita},
  shorttitle = {Ab urbe cond\adddot}
}

@ancienttext{cicero:agr,
  author = {Cicero},
  title = {De lege agraria},
  shorttitle = {Agr\adddot}
}

@ancienttext{plutarch:ant,
  author = {Plutarch},
  title = {Antonius},
  shorttitle = {Ant\adddot}
}

@ancienttext{dionysius:ant,
  author = {{Dionysius of Halicarnassus}},
  title = {Antiquitates romanae},
  shorttitle = {Ant\adddotspace rom\adddot}
}

@abbreviation{abl.,
  entrysubtype = {abbreviation},
  shorthand = {abl.},
  definition = {ablative}
}

@abbreviation{>,
  entrysubtype = {sigla},
  shorthand = {>},
  definition = {omits the lemma}
}
\end{verbatim}

\begin{verbcite}
  \nocite{AB, ABD, philo:abr, philo:agr, tacitus:ann, graniuslicinianus:ann,
    AJSL, atlantis, AzTh, BibIntSeries, BibInt, BSac, JSOTSup,
    justinmartyr:1apol, 1en, 1QM, 4QpNah, livy:aburbe, cicero:agr,
    plutarch:ant, dionysius:ant, abl., >}
\end{verbcite}
\exampleancientsources
\examplesecondarysources
\examplesigla
\examplereferences{https://sblhs2.com/2018/05/24/abbreviations-lists/}

\section{Series Volume Identifiers: Old/New and Concurrent Series (17 May 2018)}

\begin{verbatim}
@book{robinson:1952,
  author = {Robinson, John A. T.},
  title = {The Body: A Study in Pauline Theology},
  series = {Studies in Biblical Theology},
  shortseries = {SBT},
  number = {1/5},
  location = {London},
  publisher = {SCM},
  date = {1952}
}

@book{jeremias:1967,
  author = {Jeremias, Joachim},
  title = {The Prayers of Jesus},
  shorttitle = {Prayers},
  series = {Studies in Biblical Theology},
  shortseries = {SBT},
  number = {2/6},
  location = {Naperville, IL},
  publisher = {Allenson},
  date = {1967}
}

@book{frances:2014,
  author = {Young, Frances},
  title = {Ways of Reading Scripture},
  series = {Wissenschaftliche Untersuchungen zum Neuen Testament},
  shortseries = {WUNT},
  number = {1/369},
  location = {Tübingen},
  publisher = {Mohr Siebeck},
  date = {2014}
}

@book{reynolds+etal:2014,
  editor = {Reynolds, Benjamin E. and Lugioyo, Brian and Vanhoozer, Kevin J.},
  title = {Reconsidering the Relationship between Biblical and Systematic Theology in
           the New Testament},
  series = {Wissenschaftliche Untersuchungen zum Neuen Testament},
  shortseries = {WUNT},
  number = {2/369},
  location = {Tübingen},
  publisher = {Mohr Siebeck},
  date = {2014}
}

@book{witte:2015,
  author = {Witte, Markus},
  title = {Texte und Kontexte des Sirachbuchs: Gesammelte Studien zu Ben Sira und zur
           frühjüdischen Weisheit},
  series = {Forschungen zum Alten Testament},
  shortseries = {FAT},
  number = {1/98},
  location = {Tübingen},
  publisher = {Mohr Siebeck},
  date = {2015},
  langid = {german}
}

@book{tucker:2015,
  author = {Tucker, Paavo N.},
  title = {The Holiness Composition in the Book of Exodus},
  series = {Forschungen zum Alten Testament},
  shortseries = {FAT},
  number = {2/98},
  location = {Tübingen},
  publisher = {Mohr Siebeck},
  date = {2015}
}
\end{verbatim}

\begin{verbcite}
  \nocite{robinson:1952, jeremias:1967, frances:2014, reynolds+etal:2014,
    witte:2015, tucker:2015}
\end{verbcite}
\exampleabbreviations
\examplebibliography
\examplereferences{https://sblhs2.com/2018/05/17/series-volume-identifiers-old-new-and-concurrent-series/}

\section{Series Volume Identifiers (10 May 2018)}

\begin{verbatim}
@book{johnson:2018,
  editor = {Johnson, Sara R. and Dupertuis, Rubén R. and Shea, Christine},
  title = {Reading and Teaching Ancient Fiction},
  subtitle = {Jewish, Christian, and Greco-Roman Narratives},
  series = {Writings from the Greco-Roman World Supplement Series},
  shortseries = {WGRWSup},
  number = {11},
  location = {Atlanta},
  publisher = {SBL Press},
  date = {2018}
}

@commentary{salters:2010,
  author = {Salters, R. B.},
  title = {Lamentations},
  series = {International Critical Commentary},
  shortseries = {ICC},
  location = {London},
  publisher = {T\&T Clark},
  date = {2010}
}

@commentary{aune:1997,
  author = {Aune, David E.},
  title = {Revelation 1--11},
  series = {Word Biblical Commentary},
  shortseries = {WBC},
  number = {52A},
  location = {Nashville},
  publisher = {Nelson},
  date = {1997}
}

@commentary{aune:1998,
  author = {Aune, David E.},
  title = {Revelation 17--22},
  series = {Word Biblical Commentary},
  shortseries = {WBC},
  number = {52C},
  location = {Nashville},
  publisher = {Nelson},
  date = {1998}
}

@commentary{seebass:1993,
  author = {Seebass, Horst},
  title = {Numeri},
  subtitle = {Kapitel 1,1--10,10},
  shorttitle = {Numeri: 1,1--10,10},
  series = {Biblischer Kommentar, Altes Testament},
  shortseries = {BKAT},
  number = {4.1},
  location = {Neukirchen-Vluyn},
  publisher = {Neukirchener Verlag},
  date = {1993},
  langid = {german}
}

@mvcommentary{aune:1997-1998,
  author = {Aune, David E.},
  title = {Revelation},
  volumes = {3},
  series = {Word Biblical Commentary},
  shortseries = {WBC},
  number = {52A--C},
  location = {Nashville},
  publisher = {Nelson},
  date = {1997/1998}
}

@mvcommentary{seebass:1993-2007,
  author = {Seebass, Horst},
  title = {Numeri},
  volumes = {3},
  series = {Biblischer Kommentar, Altes Testament},
  shortseries = {BKAT},
  number = {4.1--3},
  location = {Neukirchen-Vluyn},
  publisher = {Neukirchener Verlag},
  date = {1993/2007},
  langid = {german}
}
\end{verbatim}

\begin{verbcite}
  \nocite{johnson:2018, salters:2010, aune:1998, seebass:1993,
    seebass:1993-2007}
\end{verbcite}
\examplecite(1)[589]{aune:1997}
\examplecite(2)[589]{aune:1997}
\examplevolcite(3){1}[589]{aune:1997-1998}
\examplevolcite(4){1}[589]{aune:1997-1998}
\exampleabbreviations
\examplebibliography
\examplereferences{https://sblhs2.com/2018/05/10/series-volume-identifiers/}

\section{Electronic Journals with Individually Paginated Articles (3 May 2018)}

\begin{verbatim}
@article{oswald:2012,
  author = {Oswald, Wolfgang},
  title = {Foreign Marriages and Citizenship in Persian Period Judah},
  shorttitle = {Foreign Marriages},
  journaltitle = {Journal of Hebrew Scriptures},
  shortjournal = {JHebS},
  volume = {12},
  date = {2012},
  eid = {6},
  pages = {1-17},
  doi = {10.5508/jhs.2012.v12.a6}
}
\end{verbatim}

\examplecite(16)[3]{oswald:2012}
\examplecite(18)[3]{oswald:2012}
\exampleabbreviations
\examplebibliography
\examplereferences{https://sblhs2.com/2018/05/03/electronic-journals-with-individually-paginated-articles/}

\section{Multiple Cities of Publication (26 April 2018)}

\begin{verbatim}
@book{hamori+stokl:2018,
  author = {Hamori, Esther J. and Stökl, Jonathan},
  title = {Perchance to Dream: Dream Divination in the Bible and the Ancient Near
           East},
  series = {Ancient Near East Monographs},
  shortseries = {ANEM},
  number = {21},
  location = {Atlanta},
  publisher = {SBL Press},
  date = {2018}
}

@book{wilken:2003,
  author = {Wilken, Robert Louis},
  title = {The Christians as the Romans Saw Them},
  edition = {2},
  location =  {New Haven and London},
  publisher = {Yale University Press},
  date = {2003}
}
\end{verbatim}

\begin{verbcite}
  \nocite{hamori+stokl:2018, wilken:2003}
\end{verbcite}
\exampleabbreviations
\examplebibliography
\examplereferences{https://sblhs2.com/2018/04/26/multiple-cities-of-publication/}

\section{Journals Identified by Issue Number (12 April 2018)}

\begin{verbatim}
@article{miller:1984,
  author = {Miller, Jr., Patrick D.},
  title = {Meter, Parallelism, and Tropes: The Search for Poetic Style},
  journaltitle = {Journal for the Study of the Old Testament},
  shortjournal = {JSOT},
  issue = {28},
  date = {1984},
  pages = {99-106}
}

@article{stott:2005-2006,
  author = {Stott, Katherine},
  title = {Finding the Lost Book of the Law: Re-reading the Story of \mkbibquote{The
           Book of the Law} (Deuteronomy--2~Kings) in Light of Classical Literature},
  journaltitle = {Journal for the Study of the Old Testament},
  shortjournal = {JSOT},
  volume = {30},
  date = {2005/2006},
  pages = {153-169}
}

@article{roth:1959-1960,
  author = {Roth, Cecil},
  title = {The Zealots and Qumran: The Basic Issue},
  journaltitle = {Revue de Qumran},
  shortjournal = {RevQ},
  volume = {2},
  issue = {5},
  date = {1959/1960},
  pages = {81-84}
}
\end{verbatim}

\examplecite(3){miller:1984}
\examplecite(4){stott:2005-2006}
\examplecite(5){roth:1959-1960}
\exampleabbreviations
\examplebibliography
\examplereferences{https://sblhs2.com/2018/04/12/journals-identified-by-issue-number/}

\section{Modern Author Names (6 April 2018)}

\begin{verbatim}
@book{wellhausen:1883,
  author = {Wellhausen, Julius},
  title = {Prolegomena zur Geschichte Israels},
  edition = {2},
  location = {Berlin},
  publisher = {Reimer},
  date = {1883},
  langid = {german}
}

@book{scott:1989,
  author = {Scott, Bernard Brandon},
  title = {Hear Then the Parable: A Commentary on the Parables of Jesus},
  location = {Philadelphia},
  publisher = {Fortress},
  date = {1989}
}

@book{logan+wedderburn:1983,
  editor = {Logan, Alastair H. B. and Wedderburn, Alexander J. M.},
  title = {The New Testament and Gnosis: Essays in Honour of Robert McL. Wilson},
  location = {Edinburgh},
  publisher = {T\&T Clark},
  date = {1983}
}

@book{barbour:2012,
  author = {Barbour, Jennie},
  title = {The Story of Israel in the Book of Qohelet: Ecclesiastes as Cultural
           Memory},
  location = {Oxford},
  publisher = {Oxford University Press},
  date = {2012}
}

@article{grillo:2017,
  author = {Grillo, Jennie},
  title = {\mkbibquote{From a Far Country}: Daniel in Isaiah’s Babylon},
  journaltitle = {Journal of Biblical Literature},
  shortjournal = {JBL},
  volume = {136},
  date = {2017},
  pages = {363-380}
}

@misc{barbour:seealso,
  author = {Barbour, Jennie},
  title = {See also \mkbibemph{Grillo, Jennie}},
  sorttitle = {zzz}
}

@misc{grillo:seealso,
  author = {Grillo, Jennie},
  title = {See also \mkbibemph{Barbour, Jennie}},
  sortyear = {zzz}
}

@article{eilberg-schwartz:1991,
  author = {Eilberg-Schwartz, Howard},
  title = {The Problem of the Body for the People of the Book},
  journaltitle = {Journal of the History of Sexuality},
  volume = {2},
  date = {1991},
  pages = {1-24}
}

@article{schusslerfiorenza:1986,
  author = {Schüssler Fiorenza, Elisabeth},
  title = {A Feminist Critical Interpretation for Liberation: Martha and Mary; Luke
           10:38–42},
  journaltitle = {Religion and Intellectual Life},
  shortjournal = {RIL},
  volume = {3},
  date = {1986},
  pages = {21-35}
}

@incollection{trebollebarrera:2013,
  author = {family=Trebolle Barrera, given=Julio, shortfamily=Barrera},
  title = {Agreements between LXX\textsuperscript{BL}, Medieval Hebrew Readings, and
           Variants of the Aramaic, Syriac and Vulgate Versions in \mkbibemph{Kaige}
           and Non-\mkbibemph{kaige} Sections of 3–4 Reigns},
  pages = {193-206},
  booktitle = {XIV Congress of the IOSCS: Helsinki, 2010},
  editor = {Peters, Melvin K. H.},
  series = {Septuagint and Cognate Studies},
  shortseries = {SCS},
  number = {59},
  location = {Atlanta},
  publisher = {Society of Biblical Literature},
  date = {2013}
}

@book{hooks:1990,
  author = {family=hooks, given=bell},
  title = {Yearning: Race, Gender, and Cultural Politics},
  location = {Boston},
  publisher = {South End},
  date = {1990}
}
\end{verbatim}

\exampleciteauthor{wellhausen:1883}
\exampleciteauthor{scott:1989}
\exampleciteauthor{tigay:1985}
\exampleciteauthor{wellhausen:1883, scott:1989, tigay:1985}
\begin{verbcite}
  \citeauthor{eilberg-schwartz:1991} has stated ….
\end{verbcite}
\begin{verbcite}
  \citeauthor{eilberg-schwartz:1991} goes on to argue ….
\end{verbcite}
\begin{verbcite}
  \citeauthor{schusslerfiorenza:1986} has stated ….
\end{verbcite}
\begin{verbcite}
  \citeauthor{schusslerfiorenza:1986} goes on to argue ….
\end{verbcite}
\begin{verbcite}
  \citeauthor{trebollebarrera:2013} has stated ….
\end{verbcite}
\begin{verbcite}
  \citeauthor{trebollebarrera:2013} goes on to argue ….
\end{verbcite}
\begin{verbcite}
  As \citeauthor{hooks:1990} offers ….
\end{verbcite}
\begin{verbcite}
  \nocite{logan+wedderburn:1983, barbour:2012, grillo:2017, barbour:seealso,
    grillo:seealso}
\end{verbcite}
\exampleabbreviations
\examplebibliography
\examplereferences{https://sblhs2.com/2018/04/06/modern-author-names/}

\section{Citing Journals and Magazines: Issue Numbers (22 March 2018)}

\begin{verbatim}
@article{yee:2017,
  author = {Yee, Gale A.},
  title = {\mkbibquote{He Will Take the Best of Your Fields}: Royal Feasts and Rural
           Extraction},
  journaltitle = {Journal of Biblical Literature},
  shortjournal = {JBL},
  volume = {136},
  date = {2017},
  pages = {821-838}
}

@article{cross:1999,
  author = {Cross, Frank Moore},
  title = {King Hezekiah's Seal Bears Phoenician Imagery},
  journaltitle = {Biblical Archeology Review},
  shortjournal = {BAR},
  volume = {25},
  number = {2},
  date = {1999},
  pages = {42-45, 60}
}
\end{verbatim}

\examplecite(7){yee:2017}
\examplecite(8){cross:1999}
\exampleabbreviations
\examplebibliography
\examplereferences{https://sblhs2.com/2018/03/22/citing-journals-and-magazines-issue-numbers/}

\section{Citing Smyth's \emph{Greek Grammar} (8 March 2018)}

\begin{verbatim}
@book{smyth:1956,
  shorthand = {Smyth},
  author = {Smyth, Herbert Weir},
  title = {Greek Grammar},
  editor = {Messing, Gordon M.},
  editortype = {reviser},
  location = {Cambridge},
  publisher = {Harvard University Press},
  date = {1956},
  pagination = {section},
  options = {shorthandformat=roman}
}

@book{smyth:1920,
  author = {Smyth, Herbert Weir},
  title = {A Greek Grammar for Colleges},
  location = {New York},
  publisher = {American Book Company},
  date = {1920},
  pagination = {section}
}

@book{smyth:1916,
  author = {Smyth, Herbert Weir},
  title = {A Greek Grammar for Schools and Colleges},
  location = {New York},
  publisher = {American Book Company},
  date = {1916},
  pagination = {section}
}
\end{verbatim}

\examplecite(42)[\pno 1765a]{smyth:1956}
\begin{verbcite}
  \nocite{smyth:1920, smyth:1916}
\end{verbcite}
\exampleabbreviations
\examplebibliography
\examplereferences{https://sblhs2.com/2018/03/08/citing-smyths-greek-grammar/}

\section{Philo of Alexandria (1 March 2018)}

\begin{verbatim}
@book{philo:cherubim,
  author = {Philo},
  title = {On the Cherubim; The Sacrifices of Abel and Cain; The Worse Attacks the
           Better; On the Posterity and Exile of Cain; On the Giants},
  translator = {Colson, F. H. and Whitaker, G. H.},
  series = {Loeb Classical Library},
  shortseries = {LCL},
  location = {Cambridge},
  publisher = {Harvard University Press},
  date = {1929}
}

@ancienttext{philo:cher,
  author = {Philo},
  title = {De cherubim},
  shorttitle = {Cher\adddot},
  translator = {Colson},
  xref = {philo:cherubim}
}

@book{philo:questionsongenesis,
  author = {Philo},
  title = {Questions on Genesis},
  translator = {Marcus, Ralph},
  series = {Loeb Classical Library},
  shortseries = {LCL},
  location = {Cambridge},
  publisher = {Harvard University Press},
  date = {1953}
}

@ancienttext{philo:QG,
  author = {Philo},
  title = {Quaestiones et solutiones in Genesin},
  shorttitle = {QG},
  xref = {philo:questionsongenesis}
}

@book{geljion+runia:2013,
  author = {Geljion, Albert C. and Runia, David T.},
  title = {Philo of Alexandria: \mkbibquote{On Cultivation}; Introduction, Translation
           and Commentary},
  shorttitle = {Philo of Alexandria: \mkbibquote{On Cultivation}},
  series = {Philo of Alexandria Commentary Series},
  shortseries = {PACS},
  number = {4},
  location = {Leiden},
  publisher = {Brill},
  date = {2013}
}

@book{wilson:2011,
  author = {Wilson, Walter T.},
  title = {Philo of Alexandria: \mkbibquote{On Virtues}; Introduction, Translation,
           and Commentary},
  shorttitle = {Philo of Alexandria: \mkbibquote{On Virtues}},
  series = {Philo of Alexandria Commentary Series},
  shortseries = {PACS},
  number = {3},
  location = {Leiden},
  publisher = {Brill},
  date = {2011}
}
\end{verbatim}

\examplecite[paren][(50)]{philo:cher}
\examplecite[paren][(1.6)]{philo:QG}
\begin{verbcite}
  As \citeauthor{philo:cher} states, “when God consorts with the soul, He makes
  what before was a woman into a virgin again” \ptranscite*[(50)]{philo:cher}.
\end{verbcite}
\examplecite(1){geljion+runia:2013}
\examplecite(2){wilson:2011}
\exampleancientsources
\examplesecondarysources
\examplebibliography
\examplereferences{https://sblhs2.com/2018/03/01/philo-of-alexandria/}

\section{Titles in Non-Latin Alphabets (22 February 2018)}

\begin{verbatim}
@book{fidler:2005,
  author = {Fidler, Ruth},
  title = {\mkbibquote{Dreams Speak Falsely?} Dream Theophanies in the Bible: Their
           Place in Ancient Israelite Faith and Traditions},
  language = {Hebrew},
  location = {Jerusalem},
  publisher = {Magnes},
  date = {2005}
}

@article{niehoff:1993,
  author = {Niehoff, Maren R.},
  title = {Associative Thinking in Rabbinic Midrash: The Example of Abraham’s and
           Sarah’s Journey to Egypt},
  language = {Hebrew},
  journaltitle = {Tarbiz},
  volume = {62},
  date = {1993},
  pages = {339–361}
}

@book{taisija:2002,
  author = {Taisija},
  title = {Акафист святому преподобному Симеону Богоприимцу: Творение игум; Таисии
           Леушинской},
  language = {langrussian},
  translatedtitle = {Akathistos for the Holy Simeon, the God-Receiver: A Work by
                     Abbess Taisija of Leušino},
  location = {Saint Petersburg},
  publisher = {Leušinskoe izdatel’stvo},
  date = {2002},
  options = {nonlatintitle}
}

@incollection{cerenc:1995,
  author = {Cerenc, Grigor},
  title = {Յայլմէ ասացեալ բան վասն խորանաց աւետարանիս},
  translatedtitle = {The Writing of Another One on the Canon-Tables of the Gospel},
  booktitle = {Խորանների մեկնություններ},
  translatedbooktitle = {Commentaries on the Canon Tables},
  editor = {Łazaryan, V.},
  location = {Erevan},
  publisher = {Sargis Xač‘enc‘},
  date = {1995},
  pages = {68-78},
  options = {nonlatintitle, nonlatinbooktitle}
}

@article{havea:1998,
  author = {Havea, Jione},
  title = {Tau lave!},
  translatedtitle = {Let’s Talk},
  journaltitle = {Pacific Journal of Theology},
  shortjournal = {PJT},
  series = {2},
  volume = {20},
  date = {1998},
  pages = {63-73}
}
\end{verbatim}

\examplecite(3){fidler:2005}
\examplecite(7){niehoff:1993}
\examplecite(12){taisija:2002}
\examplecite(15){cerenc:1995}
\examplecite(15){havea:1998}
\exampleabbreviations
\examplebibliography
\examplereferences{https://sblhs2.com/2018/02/22/titles-in-non-latin-alphabets/}

\section{Josephus (15 February 2018)}

\begin{verbatim}
@book{josephus:life;againstapion,
  author = {Josephus},
  title = {The Life; Against Apion},
  translator = {Thackeray, Henry St.\@ J.},
  series = {Loeb Classical Library},
  shortseries = {LCL},
  location = {Cambridge},
  publisher = {Harvard University Press},
  date = {1926}
}

@book{josephus:antiquitatum,
  author = {Josephus},
  title = {Antiquitatum Iudaicarum libri VI--X},
  volume = {2},
  maintitle = {Flavii Iosephi opera},
  editor = {Niese, Benedictus},
  location = {Berlin},
  publisher = {Weidmann},
  date = {1888}
}

@book{josephus:judeanantiquities,
  author = {Josephus},
  title = {Judean Antiquities 15},
  editor = {van Henten, Jan Willem},
  series = {Flavius Josephus: Translation and Commentary},
  shortseries = {FJTC},
  number = {7b},
  location = {Leiden},
  publisher = {Brill},
  date = {2014}
}

@book{worksofjosephus,
  author = {Josephus},
  title = {The Works of Flavius Josephus},
  translator = {Whiston, A. M. William},
  volumes = {2},
  location = {London},
  publisher = {Bohn},
  date = {1862}
}

@book{josephus:jewishwar,
  author = {Josephus},
  title = {The Jewish War},
  translator = {Thackery, Henry St.\@ J.},
  series = {Loeb Classical Library},
  shortseries = {LCL},
  location = {Cambridge},
  publisher = {Harvard University Press},
  date = {1927/1928}
}

@book{josephus:jewishantiquities,
  author = {Josephus},
  title = {The Jewish Antiquities},
  translator = {Thackery, Henry St.\@ J.},
  series = {Loeb Classical Library},
  shortseries = {LCL},
  location = {Cambridge},
  publisher = {Harvard University Press},
  date = {1930/1965}
}

@ancienttext{josephus:bj,
  author = {Josephus},
  title = {Bellum judaicum},
  shorttitle = {B.J\adddot},
  xref = {josephus:jewishwar}
}

@ancienttext{josephus:aj,
  author = {Josephus},
  title = {Antiquitates judaicae},
  shorttitle = {A.J\adddot},
  xref = {josephus:jewishantiquities}
}

@ancienttext{josephus:cap,
  author = {Josephus},
  title = {Contra Apionem},
  shorttitle = {C.\@ Ap\adddot},
  xref = {josephus:life;againstapion}
}
\end{verbatim}

\examplecite[][(3.506-521)]{josephus:bj}
\examplecite[ptrans][(2.233-235)]{josephus:aj}
\examplecite[atrans](5)[See also][(3.506-521)]{josephus:bj}
\begin{fverbcite}{8}
  \footnote{All translations of Josephus’s \citetitle*{josephus:cap} follow the
    translation in \cite{josephus:life;againstapion}.}
\end{fverbcite}
\begin{verbcite}
  \nocite{josephus:antiquitatum, josephus:judeanantiquities, worksofjosephus}
\end{verbcite}
\exampleancientsources
\examplesecondarysources
\examplebibliography
\examplereferences{https://sblhs2.com/2018/02/15/josephus/}

\section{Citing Page Numbers for Chapters and Articles (8 February 2018)}

\begin{verbatim}
@incollection{geurts:2017,
  author = {Geurts, Bart},
  title = {Presupposition and Givenness},
  booktitle = {The Oxford Handbook of Pragmatics},
  editor = {Huang, Yan},
  location = {Oxford},
  publisher = {Oxford University Press},
  date = {2017},
  pages = {180-198}
}

@article{wellhausen:1876-1877,
  author = {Wellhausen, Julius},
  title = {Die Composition des Hexateuchs},
  journaltitle = {Jahrbuch für deutsche Theologie},
  shortjournal = {JDT},
  related = {wellhausen:1876, wellhausen:1877},
  relatedtype = {multivolarticle},
  langid = {german}
}

@article{wellhausen:1876,
  volume = {21},
  date = {1876},
  pages = {392-450}
}

@article{wellhausen:1877,
  volume = {22},
  date = {1877},
  pages = {407-479}
}
\end{verbatim}

\examplecite(5){geurts:2017}
\citereset
\examplecite(5)[181]{geurts:2017}
\examplevolcite(6){21}[434]{wellhausen:1876-1877}
\examplevolcite(7){21}[434]{wellhausen:1876-1877}
\exampleabbreviations
\examplebibliography
\citereset
\begin{verbtext}
  \usepackage[style=sbl,citepages=separate]{biblatex}
\end{verbtext}
\makeatletter
{\cbx@opt@citepages@separate
 \examplecite(5)[181]{geurts:2017}
 \examplevolcite(6){21}[434]{wellhausen:1876-1877}
}
\makeatother
\examplereferences{https://sblhs2.com/2018/02/08/citing-page-numbers-for-chapters-and-articles/}

\section{Citing Text Collections 10: LCL (18 January 2018)}

\begin{verbatim}
@mvbook{augustine:confessions,
  author = {Augustine},
  title = {Confessions},
  translator = {Hammond, Carolyn J.-B.},
  volumes = {2},
  series = {Loeb Classical Library},
  shortseries = {LCL},
  location = {Cambridge},
  publisher = {Harvard University Press},
  date = {2014/2016}
}

@ancienttext{augustine:conf,
  author = {Augustine},
  title = {Confessions},
  shorttitle = {Conf\adddot},
  xref = {augustine:confessions}
}

@mvbook{tacitus:histories,
  author = {Tacitus},
  title = {The Histories and The Annals},
  translator = {Moore, Clifford H. and Jackson, John},
  volumes = {4},
  series = {Loeb Classical Library},
  shortseries = {LCL},
  location = {Cambridge},
  publisher = {Harvard University Press},
  date = {1937}
}

@ancienttext{tacitus:ann,
  author = {Tacitus},
  title = {Annales},
  shorttitle = {Ann\adddot},
  xref = {tacitus:histories}
}

@mvbook{apuleius:golden,
  author = {Apuleius},
  title = {The Golden Ass},
  editora = {Adlington, W.},
  editoratype = {translator},
  editorb = {Gaselee, S.},
  editorbtype = {reviser},
  series = {Loeb Classical Library},
  shortseries = {LCL},
  location = {London and New York},
  publisher = {Heinemann and Putnam's Sons},
  date = {1922}
}

@ancienttext{apuleius:metam,
  author = {Apuleius},
  title = {Metamorphoses},
  shorttitle = {Metam\adddot},
  translator = {Adlington and Gaselee},
  xref = {apuleius:golden}
}

@book{aristotle:metaphysics:2,
  author = {Aristotle},
  maintitle = {Metaphysics},
  shorttitle = {Metaphysics},
  volume = {2},
  title = {Books 10–14; Oeconomica; Magna Moralia},
  translator = {Tredennick, Hugh and Armstrong, G. Cyril},
  series = {Loeb Classical Library},
  shortseries = {LCL},
  location = {Cambridge},
  publisher = {Harvard University Press},
  date = {1935},
  options = {usetitle=false},
  sorttitle = {Metaphysics 2}
}

@book{augustine:cityofgod:1,
  author = {Augustine},
  maintitle = {City of God},
  shorttitle = {City of God},
  volume = {1},
  title = {Books 1–3},
  translator = {McCracken, George E.},
  series = {Loeb Classical Library},
  shortseries = {LCL},
  location = {Cambridge},
  publisher = {Harvard University Press},
  date = {1957},
  options = {usetitle=false},
  sorttitle = {City of God 1}
}

@mvbook{aristotle:metaphysics,
  author = {Aristotle},
  title = {Metaphysics},
  translator = {Tredennick, Hugh and Armstrong, G. Cyril},
  volumes = {2},
  series = {Loeb Classical Library},
  shortseries = {LCL},
  location = {Cambridge},
  publisher = {Harvard University Press},
  date = {1933/1935}
}

@mvbook{augustine:cityofgod,
  author = {Augustine},
  title = {City of God},
  translator = {McCracken, George E. and Green, William M. and Wiesen, David S. and
                Levine, Philip and Sanford, Eva M.},
  volumes = {7},
  series = {Loeb Classical Library},
  shortseries = {LCL},
  location = {Cambridge},
  publisher = {Harvard University Press},
  date = {1957/1972}
}

@book{augustine:selectletters,
  author = {Augustine},
  title = {Select Letters},
  translator = {Baxter, James Houston},
  series = {Loeb Classical Library},
  shortseries = {LCL},
  location = {Cambridge},
  publisher = {Harvard University Press},
  date = {1930}
}

@book{aristotle:poetics:book,
  author = {Aristotle},
  title = {Poetics},
  editor = {Halliwell, Stephen},
  translator = {Halliwell, Stephen},
  related = {longinus:sublime:related,demetrius:style:related}
}

@book{longinus:sublime:related,
  author = {Longinus},
  title = {On the Sublime},
  editora = {Fyfe, W. H.},
  editoratype = {translator},
  editorb = {Russell, Donald},
  editorbtype = {reviser},
}

@book{demetrius:style:related,
  author = {Demetrius},
  title = {On Style},
  editor = {Innes, Doreen C.},
  translator = {Innes, Doreen C.},
  series = {Loeb Classical Library},
  shortseries = {LCL},
  location = {Cambridge},
  publisher = {Harvard University Press},
  date = {1995}
}

@book{aristotle:poetics:book:short,
  author = {Aristotle},
  title = {Poetics},
  options = {skipbib}
}

@book{longinus:sublime:book,
  author = {Longinus},
  title = {On the Sublime},
  related = {aristotle:poetics:book:short},
  relatedtype = {see},
  execute = {\nocite{aristotle:poetics:book,aristotle:poetics:book:short}}
}

@book{demetrius:style:book,
  author = {Demetrius},
  title = {On Style},
  related = {aristotle:poetics:book:short},
  relatedtype = {see},
  execute = {\nocite{aristotle:poetics:book,aristotle:poetics:book:short}}
}

@ancienttext{aristotle:poetics,
  author = {Aristotle},
  title = {Poetics},
  translator = {Halliwell, Stephen},
  xref = {aristotle:poetics:book},
  execute = {\nocite{aristotle:poetics:book}}
}

@ancienttext{longinus:sublime,
  author = {Longinus},
  title = {On the Sublime},
  translator = {Fyfe, W. H.},
  xref = {longinus:sublime:book},
  execute = {\nocite{aristotle:poetics:book,longinus:sublime:book}},
}

@ancienttext{demetrius:style,
  author = {Demetrius},
  title = {On Style},
  translator = {Innes, Doreen C.},
  xref = {demetrius:style:book},
  execute = {\nocite{aristotle:poetics:book,demetrius:style:book}},
}
\end{verbatim}

\examplecite[paren][(8.29)]{augustine:conf}
\begin{verbcite}
  \cite[(15)]{tacitus:ann} details the activities of Nero.
\end{verbcite}
\examplecite(4)[For a similar example, see][(11.2)]{apuleius:metam}
\begin{verbcite}
  “These were my words, and in grief of heart I wept bitterly”
  \ptranscite[(8.29)]{augustine:conf}.
\end{verbcite}
\begin{fverbcite}{4}
  \footnote{“Thou, which dost luminate all the cities of the earth by Thy
    feminine light; Thou, which nourishes all the seeds of the world by Thy
    Damp heat, giving Thy Changing light according to the wanderings, near or
    far, of the sun” \ptranscite[(11.2)]{apuleius:metam}.}
\end{fverbcite}
\begin{fverbcite}{3}
  \footnote{Translations from the \citetitle{apuleius:golden} follow that of
    \cite{apuleius:golden}}.
\end{fverbcite}
\begin{verbcite}
  \nocite{aristotle:metaphysics:2, augustine:cityofgod:1,
    aristotle:metaphysics, augustine:cityofgod, augustine:selectletters,
    aristotle:poetics, longinus:sublime, demetrius:style}
\end{verbcite}
\nocite{aristotle:poetics:book:short}
\exampleancientsources
\examplesecondarysources
\examplebibliography
\examplereferences{https://sblhs2.com/2018/01/18/citing-text-collections-10-lcl/}

\section{Greek Magical Papyri (13 October 2017)}

\begin{verbatim}
@book{betz:1996,
  editor = {Betz, Hans Dieter},
  title = {The Greek Magical Papyri in Translation, Including the Demotic Spells},
  edition = {2},
  location = {Chicago},
  publisher = {University of Chicago Press},
  date = {1996}
}

@mvbook{preisendaz:1973-1974,
  shorthand = {PGM},
  editor = {Preisendaz, Karl},
  translator = {Preisendaz, Karl},
  title = {Papyri Graecae Magicae: Die griechischen Zauberpapyri},
  edition = {2},
  volumes = {3},
  location = {Stuttgart},
  publisher = {Teubner},
  date = {1973/1974}
}

@book{PDM:betz,
  shorthand = {PDM},
  title = {Papyri Demoticae Magicae},
  relatedstring = {Demotic texts in \citeshorthand{preisendaz:1973-1974} corpus as
                   collated in},
  related = {betz:1996}
}

@ancienttext{PGM,
  title = {\citeshorthand{preisendaz:1973-1974}},
  xref = {betz:1996}
}

@ancienttext{PDM,
  title = {\citeshorthand{PDM:betz}},
  xref = {PDM:betz}
}
\end{verbatim}

\examplecite[][(III. 410-424)]{PGM}
\examplecite[][(xiv. 554-562)]{PDM}
\examplecite[][The prayer of deliverance in][(I. 195-222)]{PGM}
\examplecite[paren][(IV. 1275-1322; IV. 1331-1389; VII. 686-702)]{PGM}
\begin{verbcite}
  A fourth- or fifth-century prayer of deliverance begins as follows: “This,
  then, is the prayer of deliverance for the first-begotten and first-born
  god: ‘I call upon you, lord. Hear me, holy god who rest among the holy ones,
  at whose side the Glorious Ones stand continually. I call upon you’”
  \parencite[(\pnfmt{I. 195-200} \mkbibbrackets{O'Neil in Betz})]{PGM}
\end{verbcite}
\nocite{preisendaz:1973-1974}
\exampleabbreviations
\examplebibliography
\examplereferences{https://sblhs2.com/2017/10/13/greek-magical-papyri/}

\section{Vetus Latina (VLB) (21 September 2017)}

\begin{verbatim}
@book{VLB:26.2,
  editor = {Gryson, Roger},
  title = {Apocalypsis Johannis},
  series = {Vetus Latina Beuron},
  shortseries = {VLB},
  number = {26.2},
  location = {Freiburg im Breisgau},
  publisher = {Herder},
  date = {2000/2003}
}

@book{VLB:26.2.8,
  editor = {Gryson, Roger},
  title = {Apocalypsis Johannis},
  series = {Vetus Latina Beuron},
  shortseries = {VLB},
  number = {26.2.8},
  location = {Freiburg im Breisgau},
  publisher = {Herder},
  date = {2003}
}
\end{verbatim}

\examplecite(16){VLB:26.2}
\examplecite(16){VLB:26.2.8}
\citereset
\examplecite(16)[625]{VLB:26.2}
\citereset
\examplecite(16)[625 \mkbibparens{upper}]{VLB:26.2}
\examplecite(18)[625 \mkbibparens{middle}]{VLB:26.2}
\examplecite(20)[625 \mkbibparens{lower}]{VLB:26.2}
\exampleabbreviations
\examplebibliography
\examplereferences{https://sblhs2.com/2017/09/21/vetus-latina-vlb/}

\section{Studia Patristica (14 September 2017)}

\begin{verbatim}
@inproceedings{husek:2010,
  author = {Hušek, Vít},
  title = {Human Freedom according to the Earliest Latin Commentaries on Paul’s
           Letters},
  series = {Studia Patristica},
  shortseries = {StPatr},
  number = {44},
  date = {2010},
  pages = {385-390}
}

@inproceedings{tkacz:2010,
  author = {Tkacz, Catherine Brown},
  title = {Esther as a Type of Christ and the Jewish Celebration of Purim},
  series = {Studia Patristica},
  shortseries = {StPatr},
  number = {44},
  date = {2010},
  pages = {183-187}
}
\end{verbatim}

\examplecite(60)[See futher][]{husek:2010}
\examplecite(62){tkacz:2010}
\exampleabbreviations
\examplebibliography
\examplereferences{https://sblhs2.com/2017/09/14/studia-patristica/}

\section{Citing Text Collections 9: Kitchen’s \emph{Ramesside Inscriptions} (2 September 2017)}

\begin{verbatim}
@mvbook{KRI,
  shorthand = {KRI},
  author = {Kitchen, K. A.},
  title = {Ramesside Inscriptions, Historical and Biographical},
  volumes = {8},
  location = {Oxford},
  publisher = {Blackwell},
  date = {1969/1990}
}

@mvbook{RITA,
  shorthand = {RITA},
  author = {Kitchen, K. A.},
  title = {Ramesside Inscriptions Translated and Annotated: Translations},
  volumes = {7},
  location = {Oxford and Chichester},
  publisher = {Blackwell and Wiley-Blackwell},
  date = {1993/2014}
}

@mvbook{RITANC,
  shorthand = {RITANC},
  author = {Kitchen, K. A. and Davies, Benedict G.},
  title = {Ramesside Inscriptions Translated and Annotated: Notes and Comments},
  volumes = {4},
  location = {Oxford and Chichester},
  publisher = {Blackwell and Wiley-Blackwell},
  date = {1993/2014}
}
\end{verbatim}

\begin{verbcite}
  \citeshortauthor{KRI} writes in \volcite{1}{KRI}: “Citations of this work
  should be as KRI, as volume, page and line” \pvolcite{1}[xxxi]{KRI}. Thus,
  one would cite line 10 of the Merenptah Lachish bowl as \volcite{4}[39, line
  10]{KRI}.
\end{verbcite}
\begin{verbcite}
  \nocite{RITA, RITANC}
\end{verbcite}
\exampleabbreviations
\examplereferences{https://sblhs2.com/2017/09/02/citing-text-collections-9-kitchens-ramesside-inscriptions/}

\section{Citing Text Collections 8: \emph{NTApoc} (17 August 2017)}

\begin{verbatim}
@mvbook{NTApoc,
  shorthand = {NTApoc},
  editor = {Schneemelcher, Wilhelm},
  title = {New Testament Apocrypha},
  editora = {Wilson, Robert McL.},
  editoratype = {English trans.\@ ed\adddot},
  edition = {Rev.\@ ed\adddot},
  volumes = {2},
  location = {Cambridge and Louisville},
  publisher = {Clarke and Westminster John Knox},
  date = {2003}
}

@ancienttext{actsandrew,
  title = {Acts of Andrew},
  xref = {NTApoc},
  volume = {2},
  pages = {101-151}
}

@ancienttext{murfrag,
  title = {Muratorian Fragment},
  shorttitle = {Mur.\@ Frag\adddot},
  translator = {Schneemelcher, Wilhelm},
  xref = {NTApoc},
  volume = {1}
}

@ancienttext{actsjohn,
  title = {Acts John},
  xref = {NTApoc},
  volume = {2}
}
\end{verbatim}

\begin{fverbcite}{13}
  \footnote{For the \citetitle{actsandrew}, see \citecollection{actsandrew}.}
\end{fverbcite}
\begin{fverbcite}{22}
  \footnote{The Muratorian Fragment offers the following about Acts of the
    Apostles: “But the acts of all apostles are written in one book. For the
    ‘most excellent Theophilus’ Luke summarises the several things that in his
    own presence have come to pass, as also by the omission of the passion of
    Peter he makes quite clear, and equally by (the omission) of the journey
    of Paul, who from the city (of Rome) proceeded to Spain”
    \ptranscite[(\linesno~34-39)35]{murfrag}.}
\end{fverbcite}
\begin{fverbcite}{22}
  \autocite[For Lycomedes’s lament over his wife Cleopatra’s paralysis,
  see][(20)173]{actsjohn}
\end{fverbcite}
\exampleancientsources
\examplesecondarysources


\subsection{Notes}

SBLHS~§6.1.1 gives the following sequence for publication information:
\begin{itemize}
  \item Editor, compiler, and/or translator
  \item Edition if not the first
  \item Volumes
\end{itemize}
\pkg{biblatex-sbl} follows this order rather than the one given in the blog
for \cite{NTApoc}.

\examplereferences{https://sblhs2.com/2017/08/17/citing-text-collections-8-_ntapoc-_/}

\section{Canon Muratori/The Muratorian Fragment (10 August 2017)}

\begin{verbatim}
@ancienttext{murfrag,
  title = {Muratorian Fragment},
  shorttitle = {Mur.\@ Frag\adddot},
}
\end{verbatim}

\begin{verbcite}
  The Epistle of Jude is referenced in \cite[(68)]{murfrag}.
\end{verbcite}
\begin{verbcite}
  \parencite[(\lineno~68)]{murfrag}
\end{verbcite}
\begin{verbcite}
  line~68 of the \citetitle*{murfrag}
\end{verbcite}
\exampleancientsources
\examplereferences{https://sblhs2.com/2017/08/10/canon-muratorithe-muratorian-fragment/}

\section{Corpus Caesarianum in \emph{BNP} (3 August 2017)}

\begin{verbatim}
@mvreference{BNP,
  shorthand = {BNP},
  editor = {Cancik, Hubert},
  title = {Brill’s New Pauly: Encyclopaedia of the Ancient World},
  volumes = {22},
  location = {Leiden},
  publisher = {Brill},
  date = {2002/2011}
}

@xdata{BNPonline,
  xref = {BNP},
  url = {http://referenceworks.brillonline.com/browse/brill-s-new-pauly}
}

@inreference{elabbadi:alexandria:history,
  author = {El-Abbadi, Mostafa},
  title = {Alexandria: History},
  xref = {BNP},
  volume = {A1},
  pages = {82-85}
}

@inreference{rupke:corpus:caesarianum,
  author = {Rüpke, Jörg},
  title = {Corpus Caesarianum},
  xdata = {BNPonline}
}
\end{verbatim}

\examplecite(1){elabbadi:alexandria:history}
\examplecite(2){rupke:corpus:caesarianum}
\exampleabbreviations
\examplebibliography
\examplereferences{https://sblhs2.com/2017/08/03/corpus-caesarianum-in-bnp/}

\section{Corpus Caesarianum: Anatomy of an Error (27 July 2017)}

\begin{verbatim}
@ancienttext{aulusgellius:noctatt,
  author = {{Aulus Gellius}},
  title = {Noctes attica},
  shorttitle = {Noct.\@ att\adddot}
}

@ancienttext{bellafr,
  title = {Bellum africum},
  shorttitle = {Bell.\@ afr\adddot}
}

@ancienttext{bellalex,
  title = {Bellum alexandrinum},
  shorttitle = {Bell.\@ alex\adddot}
}

@ancienttext{bellhisp,
  title = {Bellum hispaniense},
  shorttitle = {Bell.\@ hisp\adddot}
}

@ancienttext{bion:epitaphadon,
  author = {Bion},
  title = {Epitaphius Adonis},
  shorttitle = {Epitaph.\@ Adon\adddot}
}

@ancienttext{bion:epitaphachil,
  author = {Bion},
  title = {Epithalamium Achillis et Deidameiae},
  shorttitle = {Epith.\@ Achil\adddot}
}

@ancienttext{caesar:bellciv,
  author = {Caesar},
  title = {Bellum civile},
  shorttitle = {Bell.\@ civ.\adddot}
}

@ancienttext{caesar:bellgall,
  author = {Caesar},
  title = {Bellum gallicum},
  shorttitle = {Bell.\@ gall\adddot}
}
\end{verbatim}

\begin{verbcite}
  \nocite{aulusgellius:noctatt, bellafr, bellalex, bellhisp, bion:epitaphadon,
    bion:epitaphachil, caesar:bellciv, caesar:bellgall}
\end{verbcite}
\exampleabbreviations
\examplereferences{https://sblhs2.com/2017/07/27/corpus-caesarianum-anatomy-of-an-error/}

\section{Citing Text Collections 7: \emph{ARAB}}

\begin{verbatim}
@mvbook{ARAB,
  shorthand = {ARAB},
  author = {Luckenbill, Daniel David},
  title = {Ancient Records of Assyria and Babylonia},
  volumes = {2},
  location = {Chicago},
  publisher = {University of Chicago Press},
  date = {1926/1927},
  pagination = {paragraph}
}
\end{verbatim}

\begin{fverbcite}{28}
  \footnote{For a translation of a bull inscription from Sennacherib’s palace,
    see \volcite{2}[407-416]{ARAB}.}
\end{fverbcite}
\begin{fverbcite}{5}
  \footnote{The Prism Inscription of Tiglath-pileser I concludes with a series
    of blessings and curses reminiscent of those found within the Hebrew Bible
    \pvolcite[see]{1}[265-266]{ARAB}.}
\end{fverbcite}
\exampleabbreviations
\examplereferences{https://sblhs2.com/2017/07/20/citing-text-collections-7-arab/}

\section{Citing Text Collections 6: \emph{ANF} and \emph{NPNF} (13 July 2017)}

\begin{verbatim}
@mvcollection{ANF,
  shorthand = {ANF},
  editor = {Roberts, Alexander and Donaldson, James},
  title = {The Ante-Nicene Fathers},
  subtitle = {Translations of the Writings of the Fathers Down to A.D. 325},
  origdate = {1885/1887},
  volumes = {10},
  location = {Peabody, MA},
  publisher = {Hendrickson},
  date = {1994}
}

@ancienttext{clementinehomilies,
  entrysubtype = {ancientbook},
  title = {The Clementine Homilies},
  xref = {ANF}
}

@mvcollection{NPNF,
  shorthand = {NPNF},
  editor = {Schaff, Philip},
  title = {A Select Library of Nicene and Post-Nicene Fathers of the Christian
           Church},
  origdate = {1886/1889},
  volumes = {28},
  series = {2},
  location = {Peabody, MA},
  publisher = {Hendrickson},
  date = {1994}
}

@ancienttext{augustine:letters,
  entrysubtype = {ancientbook},
  author = {Augustine},
  title = {The Letters of St.\@ Augustin},
  translator = {Cunningham, J. G.},
  xref = {NPNF},
  volume = {1/1},
  pages = {209-593}
}

@ancienttext{eusebius:constantine,
  entrysubtype = {inancienttext},
  author = {{Eusebius of Caesarea}},
  title = {The Life of Constantine},
  translator = {Richardson, Ernest Cushing},
  xref = {NPNF},
  volume = {2/1},
  pages = {481-559},
  options = {skipbib=false}
}
\end{verbatim}

\examplevolcite[]{2/12}[85-96]{NPNF}
\examplecite(44)[(28.3.5)252]{augustine:letters}
\examplecite[atrans](44)[(28.3.5)252]{augustine:letters}
\begin{verbcite}
  \nocite{ANF, eusebius:constantine}
\end{verbcite}
\exampleabbreviations
\examplebibliography
\examplereferences{https://sblhs2.com/2017/07/13/citing-text-collections-6-anf-and-npnf/}

\section{Citing Text Collections 5: \emph{COS} (29 June 2017)}

\begin{verbatim}
@mvcollection{COS,
  shorthand = {COS},
  editor = {Hallo, William W. and Younger, Jr., K. Lawson},
  title = {The Context of Scripture},
  volumes = {4},
  location = {Leiden},
  publisher = {Brill},
  date = {1997/2016}
}

@ancienttext{bedouin:allen,
  entrysubtype = {inancientcollection},
  title = {A Report of Bedouin},
  translator = {Allen, James P.},
  xref = {COS},
  volume = {3},
  text = {5},
  pages = {16–17}
}

@ancienttext{summaryinscription8,
  entrysubtype = {inancientcollection},
  title = {Summary Inscription 8},
  translator = {Younger, Jr., K. Lawson},
  xref = {COS}
}
\end{verbatim}

\begin{fverbcite}{13}
  \footnote{For a New Kingdom model letter reporting the arrival of bedouin at
    a border fortress, see \citecollection{bedouin:allen}.}
\end{fverbcite}
\begin{fverbcite}{13}
  \footnote{For a New Kingdom model letter reporting the arrival of bedouin at
    a border fortress, see \cite{bedouin:allen}.}
\end{fverbcite}
\begin{fverbcite}{13}
  \footnote{Egypt’s attempt to exercise rigorous control over its borders is
    clearly reflected in a letter reporting the arrival of bedouin at a border
    fortress: “We have just let the Shasu tribes of Edom pass the Fortress of
    Merneptah-hetephermaat, LPH, of Tjeku, to the pool of Pithom of
    Merneptah-hetephermaat, of Tjeku, in order to revive themselves and revive
    their flocks from the great life force of Pharaoh, LPH, the perfect Sun of
    every land” \ptranscite{bedouin:allen}.}
\end{fverbcite}
\examplevolcite(21)[For the ritual of praise, see]{1.170}[175, §14]{COS}
\begin{fverbcite}{37}
  \footnote{For Shalmaneser’s account of his defeat of Hadad-ezer of Damascus
    and thirteen allied kings, see \volcite{2.113D}[267, \colno~ii,
    \linesno~13-25]{COS}.}
\end{fverbcite}
\begin{fverbcite}{42}
  \footnote{Tiglath-pileser III states: “[…] I filled [the plain] with the
    bodies of their warriors [like gras]s, [together with] their belongings,
    their cattle, their sheep, their asses […] […] within his palace […] […] I
    accepted their plea to [forgive] their rebellion (lit.\@ ‘sin’) and
    s[pared] their land” \pvoltranscite{2.117E}[290,
    \linesno~10′-13′]{summaryinscription8}.}
\end{fverbcite}
\exampleabbreviations
\examplereferences{https://sblhs2.com/2017/06/29/citing-text-collections-5-cos/}

\section{Citing Text Collections 4: \emph{MOTP} (22 June 2017)}

\begin{verbatim}
@mvcollection{MOTP,
  shorthand = {MOTP}
  editor = {Bauckham, Richard, Davila, James R. and Panayotov, Alexander},
  title = {Old Testament Pseudepigrapha: More Noncanonical Scriptures},
  volumes = {2},
  location = {Grand Rapids},
  publisher = {Eerdmans},
  date = {2013/}
}

@ancienttext{apocrseth,
  title = {Apocryphon of Seth},
  shorttitle = {Apocr.\@ Seth},
  translator = {Toepel, Alexander},
  xref = {MOTP},
  volume = {1},
  pages = {33-39}
}

@ancienttext{bknoah,
  title = {Book of Noah},
  shorttitle = {Bk.\@ Noah},
  translator = {Himmelfarb, Martha},
  xref = {MOTP},
  volume = {1},
  pages = {40-46}
}
\end{verbatim}

\begin{fverbcite}{13}
  \footnote{For an introduction to the \citetitle*{apocrseth}, see
    \citecollection{apocrseth}.}
\end{fverbcite}
\begin{fverbcite}{15}
  \footnote{The \citetitle*{bknoah} reports: “Then all the children of Noah
    together with their children came and reported their afflictions to Noah
    their father and told him about the pains their children endured”
    \ptranscite[(3)46]{bknoah}.}
\end{fverbcite}
\exampleancientsources
\examplesecondarysources
\examplereferences{https://sblhs2.com/2017/06/22/citing-text-collections-3-motp/}

\section{Pseudepigraphic Testaments (15 June 2017)}

\begin{verbatim}
@mvcollection{OTP,
  shorthand = {OTP},
  editor = {Charlesworth, James H.},
  title = {Old Testament Pseudepigrapha},
  volumes = {2},
  location = {New York},
  publisher = {Doubleday},
  date = {1983/1985}
}

@article{conybeare:1898,
  translator = {Conybeare, F. C.},
  title = {The Testament of Solomon},
  journaltitle = {Jewish Quarterly Review},
  shortjournal = {JQR},
  volume = {11},
  pages = {1-45},
  date = {1898},
  options = {usetranslator}
}

@ancienttext{t.sol.,
  title = {Testament of Solomon},
  shorttitle = {T. Sol\adddot},
  xref = {conybeare:1898}
}

@ancienttext{t.reu.,
  title = {Testament of Reuben},
  shorttitle = {T. Reu\adddot},
  translator = {Kee, H. C.},
  xref = {OTP},
  volume = {1},
  pages = {782-785}
}
\end{verbatim}

\begin{verbcite}
  “So King Solomon called the boy one day, and questioned him, saying: ‘Do I
  not love thee more than all the artisans who are working in the Temple of
  God? Do I not give thee double wages and a double supply of food? How is it
  that day by day and hour by hour thou growest thinner?’”
  \ptranscite[(1.3)]{t.sol.}.
\end{verbcite}
\begin{fverbcite}{15}
  \footnote{The testament goes on to say, “Do not devote your attention to the
    beauty of women, my children, nor occupy your minds with their activities”
    \ptranscite[(4.1)783]{t.reu.}.}
\end{fverbcite}
\exampleancientsources
\examplesecondarysources
\examplebibliography

\subsection{Notes}

SBL is not consistent with how they cite an ancient source with translator and
text collection. The most common option is:

Author, \emph{Title} Source division (trans.\@ A. N. Translator, Collection
Vol:Page).

\pkg{biblatex-sbl} adopts this format even when it departs from the blog or
handbook (as in the case of the \citetitle*{t.reu.} above).

\examplereferences{https://sblhs2.com/2017/06/15/pseudepigraphic-testaments/}

\section{Citing Text Collections 3: \emph{OTP} (8 June 2017)}

\begin{verbatim}
@mvcollection{OTP,
  shorthand = {OTP},
  editor = {Charlesworth, James H.},
  title = {Old Testament Pseudepigrapha},
  volumes = {2},
  location = {New York},
  publisher = {Doubleday},
  date = {1983/1985}
}

@ancienttext{apoc.zeph.,
  title = {Apocalypse of Zephaniah},
  shorttitle = {Apoc.\@ Zeph\adddot},
  translator = {Wintermute, O. S.},
  xref = {OTP},
  volume = {1},
  pages = {508-515}
}

@ancienttext{sib.or.,
  title = {Sibylline Oracles},
  shorttitle = {Sib.\@ Or\adddot},
  translator = {Collins, J. J.},
  xref = {OTP},
  volume = {1},
  pages = {317-472}
}

@ancienttext{ezek.trag.,
  title = {Ezekiel the Tragedian},
  shorttitle = {Ezek.\@ Trag\adddot},
  translator = {Robertson, R. G.},
  xref = {OTP},
  volume = {2},
  pages = {803-820}
}

@ancienttext{jos.asen.,
  title = {Joseph and Aseneth},
  shorttitle = {Jos.\@ Asen\adddot},
  translator = {Burchard, C.},
  xref = {OTP},
  volume = {2},
  pages = {177-248}
}
\end{verbatim}

\begin{fverbcite}{13}
  \footnote{For an example of an apocalypse likely written 100 BCE–70 CE, see
    the \fullcite{apoc.zeph.}.}
\end{fverbcite}
\begin{fverbcite}{13}
  \footnote{The imagery of flying angels blowing trumpets was common to
    apocalyptic literature, as in \citetitle[(9.1)]{apoc.zeph.}: “Then a great
    angel came forth having a golden trumpet in his hand, and he blew it three
    times over my head” \ptranscitecollection[514]{apoc.zeph.}.}
\end{fverbcite}
\examplecite(27)[For a Christian insertion, see][(12.30-34)445]{sib.or.}
\examplecite(31)[Moses addresses God at the burning bush in][(90-95)812]{ezek.trag.}
\begin{fverbcite}{17}
  \footnote{For the account of Levi attempting to save Pharaoh’s son, see
    \cite[(29.1-6)]{jos.asen.}.}
\end{fverbcite}
\exampleancientsources
\examplesecondarysources
\examplereferences{https://sblhs2.com/2017/06/08/citing-text-collections-3-otp/}

\section{Citing Text Collections 2: \emph{ANET} (1 June 2017)}

\begin{verbatim}
@collection{ANET,
  shorthand = {ANET},
  editor = {Pritchard, James B.},
  title = {Ancient Near Eastern Texts Relating to the Old Testament},
  edition = {3},
  location = {Princeton},
  publisher = {Princeton University Press},
  date = {1969}
}

@ancienttext{legendsargon,
  entrysubtype = {inancientcollection},
  title = {The Legend of Sargon},
  translator = {Speiser, E. A.},
  xref = {ANET},
  pages = {119}
}

@ancienttext{lamentationoversumer,
  entrysubtype = {inancientcollection},
  title = {Lamentation over the Destruction of Sumer and Ur},
  translator = {Kramer, S. N.},
  xref = {ANET},
  pages = {455-463}
}
\end{verbatim}

\begin{fverbcite}{13}
  \footnote{For an ancient Near Eastern account similar to Moses’s birth
    story, see \citecollection{legendsargon}.}
\end{fverbcite}
\begin{fverbcite}{13}
  \footnote{For an ancient Near Eastern account similar to Moses’s birth
    story, see \citetitle*{legendsargon} in \citecollection{legendsargon}.}
\end{fverbcite}
\begin{fverbcite}{13}
  \footnote{Many scholars have noted similarities between Moses’s birth story
    and the statement of Sargon: “she [Sargon’s mother] set me in a basket of
    rushes, with bitumen she sealed my lid” \ptranscite{legendsargon}.}
\end{fverbcite}
\examplecite(21)[508, \colno~i, \linesno~30′-38′]{ANET}
\begin{fverbcite}{37}
  \footnote{See further Sargon II’s annalistic record of the taking of Samaria
    \parencite[284-285, \linesno~23-26]{ANET}.}
\end{fverbcite}
\begin{fverbcite}{42}
  \footnote{“Lugal[marda] stood aside from his city (Marda), / Ninzuanna
    forsook her beloved dwelling, / ‘Oh her destroyed city, destroyed house,’
    bitterly she wept” \ptranscite[614,
    \linesno~136-138]{lamentationoversumer}.}
\end{fverbcite}
\exampleabbreviations
\examplereferences{https://sblhs2.com/2017/06/01/citing-text-collections-2-anet/}

\section{Citing Text Collections 1 (30 May 2017)}

\begin{verbatim}
@collection{ANET,
  shorthand = {ANET},
  editor = {Pritchard, James B.},
  title = {Ancient Near Eastern Texts Relating to the Old Testament},
  edition = {3},
  location = {Princeton},
  publisher = {Princeton University Press},
  date = {1969}
}

@mvcollection{ANF:firsted,
  shorthand = {ANF},
  editor = {Roberts, Alexander and Donaldson, James},
  title = {The Ante-Nicene Fathers},
  subtitle = {Translations of the Writings of the Fathers Down to A.D. 325},
  volumes = {10},
  date = {1885/1887}
}

@mvcollection{APOT,
  shorthand = {APOT},
  editor = {Charles, Robert H.},
  title = {The Apocrypha and Pseudepigrapha of the Old Testament},
  volumes = {2},
  location = {Oxford},
  publisher = {Clarendon},
  date = {1913}
}

@mvcollection{DSSSE,
  shorthand = {DSSSE},
  editor = {García Martínez, Florentino and Tigchelaar, Eibert J. C.},
  title = {The Dead Sea Scrolls Study Edition},
  edition = {2},
  volumes = {2},
  location = {Leiden},
  publisher = {Brill},
  date = {1999}
}

@mvcollection{COS,
  shorthand = {COS},
  editor = {Hallo, William W. and Younger, Jr., K. Lawson},
  title = {The Context of Scripture},
  volumes = {4},
  location = {Leiden},
  publisher = {Brill},
  date = {1997/2016}
}

@mvcollection{MOTP,
  shorthand = {MOTP},
  editor = {Bauckham, Richard and Davila, James R. and Panayotov, Alexander},
  title = {Old Testament Pseudepigrapha: More Noncanonical Scriptures},
  volumes = {2},
  location = {Grand Rapids},
  publisher = {Eerdmans},
  date = {2013/2025}
}

@collection{NHL,
  shorthand = {NHL},
  editor = {Robinson, James M.},
  title = {The Nag Hammadi Library in English},
  edition = {4},
  location = {Leiden},
  publisher = {Brill},
  date = {1996}
}

@collection{NHScr,
  shorthand = {NHScr},
  editor = {Meyer, Marvin},
  title = {The Nag Hammadi Scriptures: The International Edition},
  location = {New York},
  publisher = {HarperOne},
  date = {2007}
}

@mvcollection{NPNF:firsted,
  shorthand = {NPNF},
  editor = {Schaff, Philip},
  title = {A Select Library of Nicene and Post-Nicene Fathers of the Christian
           Church},
  volumes = {28},
  series = {2},
  date = {1886/1889}
}

@mvbook{NTApoc,
  shorthand = {NTApoc},
  editor = {Schneemelcher, Wilhelm},
  title = {New Testament Apocrypha},
  volumes = {2},
  editora = {Wilson, Robert McL.},
  editorastring = {Rev.\@ ed.\@ English trans.\@ ed\adddot},
  location = {Cambridge and Louisville},
  publisher = {Clarke and Westminster John Knox},
  date = {2003}
}

@mvcollection{OTP,
  shorthand = {OTP},
  editor = {Charlesworth, James H.},
  title = {Old Testament Pseudepigrapha},
  volumes = {2},
  location = {New York},
  publisher = {Doubleday},
  date = {1983/1985}
}

@series{PG,
  shorthand = {PG},
  title = {Patrologia Graeca},
  editor = {Migne, J.-P.},
  volumes = {161},
  location = {Paris},
  date = {1857/1886}
}

@series{PL,
  shorthand = {PL},
  title = {Patrologia Latina},
  editor = {Migne, J.-P.},
  volumes = {217},
  location = {Paris},
  date = {1844/1855}
}

@mvcollection{TUAT,
  shorthand = {TUAT},
  editor = {Kaiser, Otto and others},
  title = {Texte aus der Umwelt des Alten Testaments},
  location = {Gütersloh and Gütersloher},
  publisher = {Mohn and Verlagshaus},
  date = {1984/},
  langid = {german}
}
\end{verbatim}

\begin{verbcite}
  \nocite{ANET, ANF:firsted, APOT, DSSSE, COS, MOTP, NHL, NHScr, NPNF:firsted,
    OTP, PG, PL, TUAT}
\end{verbcite}
\exampleabbreviations
\examplereferences{https://sblhs2.com/2017/05/30/citing-text-collections-1/}

\section{Citing Reference Works 10: Cambridge History of Judaism (25 May 2017)}

\begin{verbatim}
@xdata{CHJ,
  series = {Cambridge History of Judaism},
  shortseries = {CHJ},
  location = {Cambridge},
  publisher = {Cambridge University Press},
}

@collection{CHJ1,
  editor = {Davies, W. D. and Finkelstein, Louis},
  title = {Introduction},
  subtitle = {The Persian Period},
  shorttitle = {Introduction: The Persian Period},
  xdata = {CHJ},
  number = {1},
  date = {1984}
}

@collection{CHJ2,
  editor = {Davies, W. D. and Finkelstein, Louis},
  title = {The Hellenistic Age},
  xdata = {CHJ},
  number = {2},
  date = {1989}
}

@collection{CHJ3,
  editor = {Horbury, William and Davies, W. D. and Sturdy, John},
  title = {The Early Roman Period},
  number = {2},
  xdata = {CHJ},
  date = {1999}
}

@collection{CHJ4,
  editor = {Katz, Steven T.},
  title = {The Late Roman-Rabbinic Period},
  xdata = {CHJ},
  number = {4},
  date = {2006}
}

@collection{CHJ5,
  editor = {Lieberman, Phillip I.},
  title = {Jews in the Medieval Islamic World, The Islamic World},
  xdata = {CHJ},
  number = {5},
  date = {2021}
}

@collection{CHJ6,
  editor = {Chazan, Robert},
  title = {The Middle Ages: The Christian World, The Christian World},
  xdata = {CHJ},
  number = {6},
  date = {2018}
}

@collection{CHJ7,
  editor = {Karp, Jonathan and Sutcliffe, Adam},
  title = {The Early Modern World, 1500–1815},
  xdata = {CHJ},
  number = {7},
  date = {2017}
}

@collection{CHJ8,
  editor = {Hart, Mitchell B. and Michels, Tony},
  title = {The Modern World, 1815–2000},
  xdata = {CHJ},
  number = {8},
  date = {2017}
}

@incollection{stern:1984,
  author = {Stern, Ephraim},
  title = {The Archeology of Persian Palestine},
  pages = {90-93},
  crossref = {CHJ1}
}

@incollection{boyce:1984,
  author = {Boyce, Mary},
  title = {Persian Religion in the Achemenid Age},
  pages = {283-284},
  crossref = {CHJ1}
}

@incollection{smallwood:1999,
  author = {Smallwood, E. Mary},
  title = {The Diaspora in the Roman Period before CE 70},
  pages = {172-177},
  crossref = {CHJ3}
}

@incollection{bahat:1999,
  author = {Bahat, Dan},
  title = {The Herodian Temple},
  pages = {38-58},
  crossref = {CHJ3}
}
\end{verbatim}

\examplecite(22){stern:1984}
\examplecite(30){boyce:1984}
\examplecite(17){smallwood:1999}
\examplecite(23){bahat:1999}
\begin{verbcite}
  \nocite{CHJ2, CHJ4, CHJ5, CHJ6, CHJ7, CHJ8}
\end{verbcite}
\exampleabbreviations
\examplebibliography
\examplereferences{https://sblhs2.com/2017/05/25/citing-reference-works-10-cambridge-history-of-judaism/}

\section{Citing Reference Works 9: Cambridge Ancient History (23 May 2017)}

\begin{verbatim}
@xdata{CAH,
  series = {Cambridge Ancient History},
  shortseries = {CAH},
  location = {Cambridge},
  publisher = {Cambridge University Press}
}

@collection{CAH1,
  editor = {Bury, J. B. and Cook, S. A. and Adcock, F. E.},
  title = {Egypt and Babylonia to 1580 B.C.\isdot},
  xdata = {CAH},
  number = {1},
  date = {1923}
}

@collection{CAH1.2,
  editor = {Edwards, I. E. S. and Gadd, C. J. and Hammond, N. G. L.},
  title = {Early History of the Middle East},
  edition = {3},
  xdata = {CAH},
  number = {1.2},
  date = {1971}
}

@collection{CAH3,
  editor = {Bury, J. B. and Cook, S. A. and Adcock, F. E.},
  title = {The Assyrian Empire},
  xdata = {CAH},
  number = {3},
  date = {1925}
}

@collection{CAH3.2,
  editor = {Boardman, John and Edwards, I. E. S. and Sollberger, E. and
            Hammond, N. G. L.},
  title = {The Assyrian and Babylonian Empires and Other States of the Near East,
           from the Eighth to the Sixth Centuries B.C.\isdot},
  shorttitle = {Assyrian and Babylonian Empires},
  edition = {2},
  xdata = {CAH},
  number = {3.2},
  date = {1991}
}

@collection{CAH14,
  editor = {Cameron, Averil and Ward-Perkins, Bryan and Whitby, Michael},
  title = {Late Antiquity},
  subtitle = {Empire and Successors, A.D. 425--600},
  xdata = {CAH},
  number = {14},
  date = {2000}
}

@incollection{smith:1925,
  author = {Smith, Sidney},
  title = {Sennacherib and Esarhaddon},
  pages = {61-70},
  crossref = {CAH3}
}

@incollection{hogarth:1925,
  author = {Hogarth, D. G.},
  title = {The Hittites of Assyria},
  pages = {138-141},
  crossref = {CAH3}
}

@incollection{grayson:1991,
  author = {Grayson, A. K.},
  title = {Assyria},
  subtitle = {Sennacherib and Esarhaddon (704--669 B.C.)},
  pages = {103-105},
  crossref = {CAH3.2}
}

@incollection{oates:1991,
  author = {Oates, Joan},
  title = {The Fall of Assyria (635--609 B.C.)},
  pages = {189-193},
  crossref = {CAH3.2}
}

@incollection{langdon:1923,
  author = {Langdon, Stephen H.},
  title = {Early Babylonia and Its Cities},
  pages = {264-372},
  crossref = {CAH1}
}

@incollection{cook:1923,
  author = {Cook, S. A.},
  title = {The Semites},
  pages = {186-189},
  crossref = {CAH1}
}
\end{verbatim}

\examplecite(21){smith:1925}
\examplecite(29){hogarth:1925}
\examplecite(47){grayson:1991}
\examplecite(95){oates:1991}
\examplecite(16){langdon:1923}
\examplecite(22){cook:1923}
\examplecite(31){gadd:1971}
\examplecite(71){vaux:1971}
\examplecite(65){liebs:2000}
\examplecite(68){cameron:2000}
\exampleabbreviations
\examplebibliography
\examplereferences{https://sblhs2.com/2017/05/23/citing-reference-works-9-cambridge-ancient-history/}

\section{Citing Reference Works 8: Handbooks, Companions, and the Like (18 May 2017)}

\begin{verbatim}
@collection{purdue:2001,
  editor = {Perdue, Leo G.},
  title = {The Blackwell Companion to the Hebrew Bible},
  location = {Oxford},
  publisher = {Blackwell},
  date = {2001}
}

@incollection{meyers:2001,
  author = {Meyers, Carol},
  title = {Early Israel and the Rise of the Israelite Monarchy},
  pages = {61-86},
  crossref = {purdue:2001}
}

@collection{rogerson+lieu:2006,
  editor = {Rogerson, J. W. and Lieu, Judith M.},
  title = {The Oxford Handbook of Biblical Studies},
  location = {Oxford},
  publisher = {Oxford University Press},
  date = {2006}
}

@incollection{dell:2006,
  author = {Dell, Katharine J.},
  title = {Wisdom},
  pages = {409-419},
  crossref = {rogerson+lieu:2006}
}

@incollection{davies:2006,
  author = {Davies, Philip},
  title = {Qumran Studies},
  pages = {99-107},
  crossref = {rogerson+lieu:2006}
}
\end{verbatim}

\examplecite(13)[64-66]{meyers:2001}
\examplecite(21)[415-416]{dell:2006}
\citereset
\begin{verbtext}
  \usepackage[style=sbl,citepages=separate]{biblatex}
\end{verbtext}
\makeatletter
{\cbx@opt@citepages@separate
 \examplecite(13)[64-66]{meyers:2001}
}
\makeatother
\examplecite(21)[415-416]{dell:2006}
\examplecite(22)[99]{davies:2006}
\examplebibliography
\examplereferences{https://sblhs2.com/2017/05/18/citing-reference-works-8-handbooks-companions-and-the-like/}

\section{Citing Reference Works 8: English-Language Dictionaries (16 May 2017)}

\begin{verbatim}
@reference{webster11,
  title = {Merriam-Webster’s Collegiate Dictionary},
  edition = {11},
  options = {skipbib}
}

@reference{oxford2,
  title = {Oxford English Dictionary},
  edition = {2},
  options = {skipbib}
}

@reference{OED2,
  shorthand = {OED},
  title = {Oxford English Dictionary},
  edition = {2},
  options = {skipbib}
}
\end{verbatim}

\examplecite[][\sv~\mkbibquote{pericope}]{webster11}
\examplecite[][\sv~\mkbibquote{pericope}]{oxford2}
\examplecite[][\sv~\mkbibquote{pericope}]{OED2}
\exampleabbreviations

\section{Musonius Rufus (11 May 2017)}

\begin{verbatim}
@inseries{lutz:1947,
  author = {Lutz, Cora E.},
  title = {Musonius Rufus: \mkbibquote{The Roman Socrates}},
  series = {Yale Classical Studies},
  shortseries = {YCS},
  date = {1947},
  number = {10},
  pages = {3-147}
}

@ancienttext{musoniusrufus:diatr,
  author = {{Musonius Rufus}},
  title = {Diatribai},
  shorttitle = {Diatr\adddot},
  translator = {Lutz},
  xref = {lutz:1947}
}

@ancienttext{musoniusrufusfrag:abbrev,
  author = {{Musonius Rufus}},
  title = {\textup{fragment(s)}},
  shorttitle = {\textup{frag(s)\adddot}},
  sorttitle = {frag(s)\adddot}
}

@ancienttext{musoniusrufus:frag,
  author = {{Musonius Rufus}},
  title = {\bibhyperlink{shorttitle:\detokenize{\textup{frag(s)\adddot}}}
           {\textup{\iffieldnum{titleaddon}{\bibstring{fragment}}
           {\bibstring{fragments}}}}},
  sorttitle = {frag(s)\adddot},
  translator = {Lutz},
  xref = {lutz:1947},
  options = {skipbiblist},
  execute = {\nocite{musoniusrufusfrag:abbrev}}
}
\end{verbatim}

\examplecite[paren][(21.33-35)]{musoniusrufus:diatr}
\examplecite[paren][(38)]{musoniusrufus:frag}
\examplecite[ptrans][(21.33-35)]{musoniusrufus:diatr}
\examplecite[ptrans][(38)]{musoniusrufus:frag}
\nocite{musoniusrufusfrag:abbrev}
\exampleancientsources
\examplesecondarysources
\examplebibliography
\examplereferences{https://sblhs2.com/2017/05/11/musonius-rufus/}

\section{Historia Augusta (9 May 2017)}

\begin{verbatim}
@mvbook{scriptoreshistoriaeaugustae,
  translator = {Magie, David},
  title = {The Scriptores historiae Augustae},
  series = {Loeb Classical Library},
  shortseries = {LCL},
  location = {Cambridge and London},
  publisher = {Harvard University Press and Heinemann},
  date = {1921/1932},
  options = {usetranslator}
}

@ancienttext{histaug,
  title = {Historia Augusta},
  shorttitle = {Hist.\@ Aug\adddot},
  xref = {scriptoreshistoriaeaugustae}
}

@ancienttext{histaug:sev,
  maintitle = {Historia Augusta},
  shortmaintitle = {Hist.\@ Aug\adddot},
  title = {Severus},
  shorttitle = {Sev\adddot},
  xref = {scriptoreshistoriaeaugustae},
  execute = {\nocite{histaug}}
}

@ancienttext{histaug:opilmacr,
  maintitle = {Historia Augusta},
  shortmaintitle = {Hist.\@ Aug\adddot},
  title = {Opilius Macrinus},
  shorttitle = {Opil.\@ Macr\adddot},
  xref = {scriptoreshistoriaeaugustae},
  execute = {\nocite{histaug}}
}
\end{verbatim}

\examplecite[full]{histaug:sev}
\examplecite[][(1.4)]{histaug:sev}
\examplecite[][(6.5)]{histaug:opilmacr}
\nocite{histaug}
\exampleancientsources
\examplesecondarysources
\examplebibliography
\examplereferences{https://sblhs2.com/2017/05/09/historia-augusta/}

\section{PG Citations: Update (4 May 2017)}

\begin{verbatim}
@series{PG,
  shorthand = {PG},
  title = {Patrologia Graeca},
  editor = {Migne, J.-P.},
  volumes = {161},
  location = {Paris},
  date = {1857/1886}
}

@ancienttext{gregory:orationestheologicae,
  author = {{Gregory of Nazianzus}},
  title = {Orationes theologicae},
  xref = {PG},
  volume = {36}
}
\end{verbatim}

\examplecite(6)[(4.19)128c]{gregory:orationestheologicae}
\exampleabbreviations
\examplereferences{https://sblhs2.com/2017/05/04/pg-citations-update/}

\section{PG Citations (2 May 2017)}

\begin{verbatim}
@series{PG,
  shorthand = {PG},
  title = {Patrologia Graeca},
  editor = {Migne, J.-P.},
  volumes = {161},
  location = {Paris},
  date = {1857/1886}
}

@ancienttext{methodius:symp,
  author = {{Methodius of Olympus}},
  title = {Symposium \mkbibparens{Convivium decem virginum}},
  shorttitle = {Symp\adddot},
  xref = {PG},
  volume = {18}
}
\end{verbatim}

\examplecite[][(2.4)52c]{methodius:symp}
\exampleancientsources
\examplesecondarysources
\examplereferences{https://sblhs2.com/2017/05/02/pg-citations/}

\section{Citing Reference Works 7: Greek Language Tools (20 April 2017)}

\begin{verbatim}
@book{BDF,
  shorthand = {BDF},
  author = {Blass, Friedrich and Debrunner, Albert and Funk, Robert W.},
  title = {A Greek Grammar of the New Testament and Other Early Christian Literature},
  location = {Chicago},
  publisher = {University of Chicago Press},
  date = {1961},
  pagination = {section},
  options = {shorthandformat=roman}
}

@book{burton:1903,
  author = {Burton, Ernest DeWitt},
  title = {Syntax of the Moods and Tenses in New Testament Greek},
  edition = {5},
  location = {Chicago},
  publisher = {University of Chicago Press},
  date = {1903},
  pagination = {section}
}

@book{burton,
  shorthand = {Burton},
  crossref = {burton:1903},
  options = {shorthandformat=roman}
}

@book{dana+mantey:1927,
  author = {Dana, H. E. and Mantey, Julius R.},
  title = {A Manual Grammar of the Greek New Testament},
  location = {New York},
  publisher = {Macmillan},
  date = {1927},
  pagination = {section}
}

@book{funk:2013,
  author = {Funk, Robert W.},
  title = {A Beginning-Intermediate Grammar of Hellenistic Greek},
  edition = {3},
  location = {Salem, OR},
  publisher = {Polebridge},
  date = {2013},
  pagination = {section}
}

@book{funk,
  shorthand = {Funk},
  crossref = {funk:2013},
  options = {shorthandformat=roman}
}

@book{goodwin:1930,
  author = {Goodwin, William W.},
  title = {Greek Grammar},
  editora = {Gulick, Charles Burton},
  editorastring = {revby},
  location = {Boston},
  publisher = {Ginn},
  date = {1930},
  pagination = {section}
}

@mvbook{moulton:1908-1976,
  author = {Moulton, James Hope},
  withauthor = {Howard, W. F. and Turner, Nigel},
  sortkey = {Moulton, James Hope, with W. F. Howard and Nigel Turner},
  title = {A Grammar of New Testament Greek},
  volumes = {4},
  location = {Edinburgh},
  publisher = {T\&T Clark},
  date = {1908/1976}
}

@book{moulton:vol1,
  author = {Moulton, James Hope},
  title = {Prolegomena},
  titleaddon = {\mkbibordinal{3}~\bibsstring{edition}},
  volume = {1},
  date = {1908/1976},
  crossref = {moulton:1908-1976}
}

@book{moulton:vol2,
  author = {Moulton, James Hope and Howard, W. F.},
  title = {Accidence and Word Formation: With an Appendix on Semitisms in the New
           Testament},
  volume = {2},
  date = {1908/1976},
  crossref = {moulton:1908-1976}
}

@book{moulton:vol4,
  author = {Turner, Nigel},
  title = {Style},
  volume = {4},
  date = {1908/1976},
  crossref = {moulton:1908-1976}
}

@book{smyth:1956,
  shorthand = {Smyth},
  author = {Smyth, Herbert Weir},
  title = {Greek Grammar},
  editor = {Messing, Gordon M.},
  editortype = {reviser},
  location = {Cambridge},
  publisher = {Harvard University Press},
  date = {1956},
  pagination = {section},
  options = {shorthandformat=roman}
}

@book{robertson,
  shorthand = {Robertson},
  xdata = {robertson:1923},
  pagination = {section},
  options = {shorthandformat=roman}
}

@book{robertson:1923,
  author = {Robertson, A. T.},
  title = {A Grammar of the Greek New Testament in the Light of Historical Research},
  edition = {4},
  location = {London},
  publisher = {Hodder \& Stoughton},
  date = {1923},
  options = {shorthandformat=roman}
}

@book{zerwick:2011,
  author = {Zerwick, Maximilian},
  title = {Biblical Greek Illustrated by Examples},
  origlanguage = {from the 4th Latin ed\adddot},
  translator = {Smith, Joseph},
  series = {Scripta Pontificii Instituti Biblici},
  number = {114},
  location = {Rome},
  publisher = {Editrice Pontificio Istituto biblico},
  date = {1963},
  related = {zerwick:repr},
  relatedtype = {reprint},
  pagination = {section}
}

@book{zerwick:repr,
  series = {Subsidia Biblica},
  shortseries = {SubBi},
  number = {41},
  location = {Rome},
  publisher = {Gregorian \& Biblical Press},
  date = {2011}
}

@book{zerwick,
  shorthand = {Zerwick},
  xdata = {zerwick:2011},
  options = {shorthandformat=roman}
}
\end{verbatim}

\examplecite(76)[151]{BDF}
\examplecite(67)[148-152]{BDF}
\examplecite(58)[\pno 151.1, 4]{BDF}
\examplecite(71)[150]{BDF}
\examplecite(14)[151.2]{BDF}
\examplecite(17)[89-94]{burton:1903}
\examplecite(36)[193.1]{burton}
\examplecite(63)[155]{dana+mantey:1927}
\examplecite(16)[431]{funk:2013}
\examplecite(22)[310-311]{funk}
\examplecite(83)[718]{goodwin:1930}
\examplecite(22)[108-110]{moulton:vol1}
\examplecite[afull](5)[178]{moulton:vol2}
\examplecite[afull](56)[71]{moulton:vol4}
\examplecite(33)[523]{robertson:1923}
\examplecite(21)[\pno 11.10e \mkbibparens{523}]{robertson}
\examplecite(42)[\pno 1765a]{smyth:1956}
\examplecite(19)[360-362]{zerwick:2011}
\examplecite(19)[360-362]{zerwick}
\exampleabbreviations
\examplebibliography

\subsection{Notes}

\begin{itemize}
  \item SBLHS~§6.2.21 states “It is unnecessary when citing a single volume to
    give information about the total number of volumes in the series.”
    Accordingly, although this blog post includes the total volumes for the
    three individual \citeshortauthor{moulton:1908-1976} volumes,
    \pkg{biblatex-sbl} omits them.
  \item The format of the translator in \citeshortauthor{zerwick:2011} differs
    from a similar example in SBLHS~§6.2.20. I've opted to use the latter as I
    think it's more consistent with how SBL generally prefers formatting in
    the bibliography.
\end{itemize}

\examplereferences{https://sblhs2.com/2017/04/20/citing-reference-works-7-greek-language-tools/}

\section{Citing Reference Works 6: Hebrew Language Tools (18 April 2017)}

\begin{verbatim}
@book{BHRG,
  shorthand = {BHRG},
  author = {Merwe, Christo H. J. and van der Naudé, Jackie A. and Kroeze, Jan H},
  title = {A Biblical Hebrew Reference Grammar},
  series = {Biblical Languages: Hebrew},
  number = {3},
  location = {Sheffield},
  publisher = {Sheffield Academic},
  date = {1999},
  pagination = {section},
  options = {shorthandformat=roman},
}

@book{BL,
  shorthand = {BL},
  author = {Bauer, Hans and Leander, Pontus},
  title = {Historische Grammatik der hebräischen Sprache des Alten Testamentes},
  origlocation = {Halle},
  origpublisher = {Niemeyer},
  origdate = {1922},
  location = {Hildesheim},
  publisher = {Olms},
  date = {1991},
  pagination = {section},
  options = {shorthandformat=roman},
  langid = {german}
}

@book{davidson:1901,
  author = {Davidson, A. B.},
  title = {Hebrew Syntax},
  edition = {3},
  location = {Edinburgh},
  publisher = {T\&T Clark},
  date = {1901},
  pagination = {section},
  options = {shorthandformat=roman}
}

@mvbook{GKB,
  shorthand = {GKB},
  author = {Gesenius, Wilhelm},
  title = {Hebräische Grammatik},
  editor = {Kautzsch, Emil and Bergsträsser, Gotthelf},
  parts = {2},
  volumes = {3},
  location = {Leipzig},
  publisher = {Vogel},
  date = {1918/1929},
  pagination = {section},
  options = {shorthandformat=roman},
  langid = {german}
}

@book{GKC,
  shorthand = {GKC},
  author = {Gesenius, Wilhelm},
  title = {Gesenius' Hebrew Grammar},
  editor = {Kautzsch, Emil},
  translator = {Cowley, Arthur E.},
  edition = {2},
  location = {Oxford},
  publisher = {Clarendon},
  date = {1910},
  pagination = {section},
  options = {shorthandformat=roman}
}

@book{IBHS,
  shorthand = {IBHS},
  author = {Waltke, Bruce K. and O'Connor, Michael},
  title = {An Introduction to Biblical Hebrew Syntax},
  location = {Winona Lake, IN},
  publisher = {Eisenbrauns},
  date = {1990},
  pagination = {section}
}

@mvbook{Jouon,
  shorthand = {Joüon},
  author = {Joüon, Paul},
  title = {A Grammar of Biblical Hebrew},
  editor = {Muraoka, T.},
  editortype = {translatorrev},
  volumes = {2},
  location = {Rome},
  publisher = {Pontifical Biblical Institute},
  date = {1991},
  pagination = {section},
  options = {shorthandformat=roman}
}

@book{lambdin:1971,
  author = {Lambdin, Thomas O.},
  title = {Introduction to Biblical Hebrew},
  location = {New York},
  publisher = {Scribner’s Sons},
  date = {1971},
  pagination = {section},
  options = {shorthandformat=roman}
}

@book{williams:1976,
  author = {Williams, Ronald J},
  title = {Hebrew Syntax: An Outline},
  edition = {2},
  location = {Toronto},
  publisher = {University of Toronto Press},
  date = {1976},
  pagination = {section},
  options = {shorthandformat=roman}
}

@book{yeivin:1980,
  author = {Yeivin, Israel},
  title = {Introduction to the Tiberian Masorah},
  editor = {Revell, E. J.},
  translator = {Revell, E. J.},
  series = {Masoretic Studies},
  shortseries = {MasS},
  number = {5},
  location = {Missoula, MT},
  publisher = {Scholars Press},
  date = {1980},
  pagination = {section}
}
\end{verbatim}

\examplecite(76)[\pno 112a]{GKC}
\examplecite(67)[112-114]{GKC}
\examplecite(58)[\pno 112a, d]{GKC}
\examplecite(51)[30.1]{BHRG}
\examplecite(3)[\pno 25.3.2.i.a]{BHRG}
\examplecite(71)[\pno 48d′]{BL}
\examplecite(36)[\pno 39c]{davidson:1901}
\examplecite(33)[\pno 12a--f]{GKB}
\examplecite(76)[\pno 112a, d]{GKC}
\examplecite(13)[23.2]{IBHS}
\examplecite(25)[\pno 30.5.2b, example 3]{IBHS}
\examplecite(76)[56-58]{Jouon}
\examplecite(5)[\pno 107c]{lambdin:1971}
\examplecite(6)[446]{williams:1976}
\examplecite(31)[132 \mkbibparens{111}]{yeivin:1980}
\exampleabbreviations
\examplebibliography
\examplereferences{https://sblhs2.com/2017/04/18/citing-reference-works-6-hebrew-language-tools/}

\section{Citing Reference Works 5: Topical Dictionaries and Encyclopedias (13 April 2017)}

\begin{verbatim}
@mvreference{ABD,
  shorthand = {ABD},
  editor = {Freedman, David Noel},
  title = {Anchor Bible Dictionary},
  volumes = {6},
  location = {New York},
  publisher = {Doubleday},
  date = {1992}
}

@inreference{hanson+etal:apocalypses,
  author = {Hanson, Paul D. and Grayson, Kirk and Collins, john J. and Yarbro Collins,
            Adela},
  title = {Apocalypses and Apocalypticism},
  xref = {ABD},
  volume = {1},
  pages = {279-292}
}

@inreference{yarbrocollins:earlychristian,
  author = {Yarbro Collins, Adela},
  title = {Apocalypses and Apocalypticism: Early Christian},
  xref = {ABD},
  volume = {1},
  pages = {288-292}
}

@mvreference{ANRW,
  shorthand = {ANRW},
  editor = {Temporini, Hildegard and Haase, Wolfgang},
  title = {Aufstieg und Niedergang der römischen Welt},
  subtitle = {Geschichte und Kultur Roms im Spiegel der neueren Forschung},
  titleaddon = {Part 2, \mkbibemph{Principat}},
  location = {Berlin},
  publisher = {de Gruyter},
  date = {1972/},
  langid = {german}
}

@inreference{anderson:pepaideumenos,
  author = {Anderson, Graham},
  title = {The \mkbibemph{pepaideumenos} in Action},
  subtitle = {Sophists and Their Outlook in the Early Empire},
  shorttitle = {\mkbibemph{Pepaideumenos}},
  xref = {ANRW},
  volume = {33},
  part = {1},
  pages = {80-208}
}

@mvreference{DBI,
  shorthand = {DBI},
  editor = {Hayes, John},
  title = {Dictionary of Biblical Interpretation},
  volumes = {2},
  location = {Nashville},
  publisher = {Abingdon},
  date = {1999}
}

@inreference{tiroshsamuelson:kabbalah,
  author = {Tirosh-Samuelson, Hava},
  title = {Kabbalah},
  xref = {DBI},
  volume = {2},
  pages = {1-7}
}

@reference{DDD,
  shorthand = {DDD},
  editor = {van der Toorn, Karel and Becking, Bob and van der Horst, Pieter},
  title = {Dictionary of Deities and Demons in the Bible},
  edition = {2},
  location = {Leiden and Grand Rapids},
  publisher = {Brill and Eerdmans},
  date = {1999}
}

@inreference{assmann:amun,
  author = {Assmann, Jan},
  title = {Amun},
  xref = {DDD},
  pages = {28-32}
}

@reference{DJG,
  shorthand = {DJG},
  editor = {Green, Joel B. and McKnight, Scot},
  title = {Dictionary of Jesus and the Gospels},
  location = {Downers Grove, IL},
  publisher = {InterVarsity Press},
  date = {1992}
}

@inreference{wise:languagespalestine,
  author = {Wise, Michael O},
  title = {Languages of Palestine},
  xref ={DJG},
  pages = {434-443}
}

@reference{DLNT,
  shorthand = {DLNT},
  editor = {Martin, Ralph P. and Davids, Peter H.},
  title = {Dictionary of the Later New Testament and Its Developments},
  location = {Downers Grove, IL},
  publisher = {InterVarsity Press},
  date = {1997}
}

@reference{DNTB,
  shorthand = {DNTB},
  editor = {Evans, Craig A. and Porter, Stanley E. Porter},
  title = {Dictionary of New Testament Background},
  location = {Downers Grove, IL},
  publisher = {InterVarsity Press},
  date = {2000}
}

@reference{DOTHB,
  shorthand = {DOTHB},
  editor = {Arnold, Bill T. and Williamson, H. G. M.},
  title = {Dictionary of the Old Testament: Historical Books},
  location = {Downers Grove, IL},
  publisher = {InterVarsity Press},
  date = {2005}
}

@reference{DPL,
  shorthand = {DPL},
  editor = {Hawthorne, Gerald F. and Martin, Ralph P.},
  title = {Dictionary of Paul and His Letters},
  location = {Downers Grove, IL},
  publisher = {InterVarsity Press},
  date = {1993}
}

@mvreference{EBR,
  shorthand = {EBR},
  editor = {Klauck, Hans-Josef and others},
  title = {Encyclopedia of the Bible and Its Reception},
  location = {Berlin},
  publisher = {de Gruyter},
  date = {2009/}
}

@inreference{crawford:genesisiii,
  author = {Crawford, Sidnie White},
  title = {Genesis: III. Judaism A. Second Temple and Hellenistic Judaism},
  xref = {EBR},
  volume = {9},
  pages = {1156-1160}
}

@reference{EDB,
  shorthand = {EDB},
  editor = {Freedman, David Noel},
  title = {Eerdmans Dictionary of the Bible},
  location = {Grand Rapids},
  publisher = {Eerdmans},
  date = {2000},
  pagination = {subverbo}
}

@inreference{dever:abel,
  author = {Dever, William G.},
  title = {Abel-Beth-Maacah},
  xref = {EDB},
  pages = {3-4}
}

@mvreference{EncJud,
  shorthand = {EncJud},
  editor = {Skolnik, Fred and Berenbaum, Michael},
  title = {Encyclopedia Judaica},
  edition = {2},
  volumes = {22},
  location = {Detroit},
  publisher = {Macmillan Reference USA},
  date = {2007}
}

@inreference{stone:isaactest,
  author = {Stone, Michael E.},
  title = {Isaac, Testament Of},
  xref = {EncJud},
  volume = {10},
  pages = {36}
}

@reference{HBD,
  shorthand = {HDB},
  editor = {Powell, Mark Allan and others},
  title = {HarperCollins Bible Dictionary},
  edition = {3},
  location = {San Francisco},
  publisher = {HarperOne},
  date = {2011},
  pagination = {subverbo}
}

@inreference{mattingly:philistines,
  author = {Mattingly, Gerald L.},
  title = {Philistines},
  xref = {HBD},
  pages = {797-800}
}

@inreference{schaub:holywar,
  author = {Schaub, Marilyn M.},
  title = {Holy War},
  xref = {HBD},
  pages = {387}
}

@reference{HWBI,
  shorthand = {HWBI},
  editor = {Taylor, Marion Ann},
  title = {Handbook of Women Biblical Interpreters},
  location = {Grand Rapids},
  publisher = {Baker Academic},
  date = {2012}
}

@inreference{popelevison:truth,
  author = {Pope-Levison, Priscilla},
  title = {Truth, Sojouner},
  xref ={HWBI},
  pages = {509-511}
}

@mvreference{IDB,
  shorthand = {IDB},
  editor = {Buttrick, George A.},
  title = {The Interpreter’s Dictionary of the Bible},
  volumes = {4},
  location = {New York},
  publisher = {Abingdon},
  date = {1962},
  pagination = {subverbo}
}

@inreference{dentan:apple,
  author = {Dentan, Robert C.},
  title = {Apple of the Eye},
  xref = {IDB},
  volume = {1},
  pages = {176}
}

@reference{IDBSup,
  shorthand = {IDBSup},
  author = {Crim, Keith},
  title = {Interpreter’s Dictionary of the Bible: Supplementary Volume},
  location = {Nashville},
  publisher = {Abingdon},
  date = {1976},
  pagination = {subverbo}
}

@inreference{sperling:mount,
  author = {Sperling, David},
  title = {Mount, Mountain},
  xref = {IDBSup},
  pages = {608-609}
}

@mvreference{ISBE,
  shorthand = {ISBE},
  editor = {Bromiley, Geoffrey W.},
  title = {International Standard Bible Encyclopedia},
  edition = {Rev\adddotspace ed\adddot},
  volumes = {4},
  location = {Grand Rapids},
  publisher = {Eerdmans},
  date = {1979/1988},
  pagination = {subverbo}
}

@inreference{burke:home,
  author = {Burke, David G.},
  title = {Home},
  xref = {ISBE},
  volume = {2},
  pages = {746-749}
}

@mvreference{JE,
  shorthand = {JE},
  editor = {Singer, Isidore},
  title = {The Jewish Encyclopedia},
  volumes = {12},
  location = {New York},
  publisher = {Funk \& Wagnalls},
  date = {1901/1906}
}

@inreference{jacobs:typography,
  author = {Jacobs, Joseph},
  title = {Typography},
  xref = {JE},
  volume = {12},
  pages = {295-335}
}

@mvreference{NEAEHL,
  shorthand = {NEAEHL},
  editor = {Stern, Ephraim},
  title = {The New Encyclopedia of Archaeological Excavations in the Holy Land},
  volumes = {4},
  location = {Jerusalem and New York},
  publisher = {{Israel Exploration Society and Carta} and Simon \& Schuster},
  date = {1993}
}

@inreference{negev:petra,
  author = {Negev, Abraham},
  title = {Petra},
  xref = {NEAEHL},
  volume = {4},
  pages = {1181-1193}
}

@inreference{dothan+dunayevsky:qasiletell,
  author = {Dothan, Trude and Dunayevsky, Imanuel},
  title = {Qasile, Tell: Excavations in Area A},
  xref = {NEAEHL},
  volume = {4},
  pages = {1204-1207}
}

@inreference{reich+shukron:jerusalem,
  author = {Reich, Ronny and Shukron, Eli},
  title = {Jerusalem: The Gihon Spring and Eastern Slope of the City of David},
  xref = {NEAEHL},
  volume = {5},
  pages = {1801-1807}
}

@mvreference{NIDB,
  shorthand = {NIDB},
  editor = {Sakenfeld, Katharine Doob},
  title = {New Interpreter's Dictionary of the Bible},
  volumes = {5},
  location = {Nashville},
  publisher = {Abingdon},
  date = {2006/2009},
  pagination = {subverbo}
}

@inreference{meyers:menorah,
  author = {Meyers, Carol},
  title = {Menorah},
  xref = {NIDB},
  volume = {4},
  pages = {42-43}
}

@inreference{graf:arab,
  author = {Graf, David},
  title = {Arab, Arabian, Arabia},
  xref = {NIDB},
  volume = {1},
  pages = {212-220}
}

@reference{OCD,
  shorthand = {OCD},
  editor = {Hornblower, Simon and Spawforth, Antony},
  title = {Oxford Classical Dictionary},
  edition = {4},
  location = {Oxford},
  publisher = {Oxford University Press},
  date = {2012}
}

@inreference{taylor:protogoras,
  author = {Taylor, Christopher C. W.},
  title = {Protagoras},
  xref = {OCD},
  pages = {1227-1228}
}

@reference{ODCC,
  shorthand = {ODCC},
  editor = {Cross, F. L. and Livingston, E. A.},
  title = {The Oxford Dictionary of the Christian Church},
  edition = {3},
  location = {Oxford},
  publisher = {Oxford University Press},
  date = {2005},
  pagination = {subverbo}
}

@mvreference{OEANE,
  shorthand = {OEANE},
  editor = {Meyers, Eric M.},
  title = {The Oxford Encyclopedia of Archaeology in the Near East},
  volumes = {5},
  location = {New York},
  publisher = {Oxford University Press},
  date = {1997}
}

@inreference{pitard:aleppo,
  author = {Pitard, Wayne T.},
  title = {Aleppo},
  xref = {OEANE},
  volume = {1},
  pages = {63-65}
}

@mvreference{PW,
  shorthand = {PW},
  editor = {Wissowa, Georg and Kroll, Wilhelm},
  title = {Paulys Real-Encyclopädie der classischen Altertumswissenschaft},
  volumes = {50},
  parts = {84},
  location = {Stuttgart},
  publisher = {Metzler and Druckenmüller},
  date = {1894/1980},
  langid = {german},
  options = {shorthandformat=roman}
}

@inreference{honigmann:libanos,
  author = {Honigmann, Ernest},
  title = {Libanos},
  xref = {PW},
  volume = {13},
  part = {1},
  pages = {1-11}
}

@reference{RGG,
  shorthand = {RGG},
  editor = {Betz, Hans Dieter and others},
  title = {Religion in Geschichte und Gegenwart},
  edition = {4},
  location = {Tübingen},
  publisher = {Mohr Siebeck},
  date = {1998/2007},
  langid = {german}
}

@mvreference{RlA,
  shorthand = {RlA},
  editor = {Ebeling, Erich and others},
  title = {Reallexikon der Assyriologie},
  location = {Berlin},
  publisher = {de Gruyter},
  date = {1928/},
  langid = {german}
}

@inreference{westbrook:mitgift,
  author = {Westbrook, Raymond},
  title = {Mitgift},
  xref = {RlA},
  volume = {8},
  pages = {273-283}
}

@mvreference{RPP,
  shorthand = {RPP},
  editor = {Betz, Hans Dieter and others},
  title = {Religion Past and Present: Encyclopedia of Theology and Religion},
  volumes = {14},
  location = {Leiden},
  publisher = {Brill},
  date = {2007/2013}
}

@inreference{wandrey:ascension,
  author = {Wandrey, Irina},
  title = {Ascension and Martyrdom of Isaiah},
  xref = {RPP},
  volume = {1},
  pages = {428}
}

@inreference{blum:abraham,
  author = {Blum, Erhard},
  title = {Abraham: Old Testament},
  xref = {RPP},
  volume = {1},
  pages = {11-13}
}

@mvreference{TRE,
  shorthand = {TRE},
  editor = {Krause, Gerhard and Müller, Gerhard},
  title = {Theologische Realenzyklopädie},
  volumes = {36},
  location = {Berlin},
  publisher = {de Gruyter},
  date = {1976/2004},
  langid = {german}
}

@inreference{albertz:magie,
  author = {Albertz, Rainer},
  title = {Magie: Altes Testament},
  xref = {TRE},
  volume = {21},
  pages = {691-695},
  langid = {german}
}
\end{verbatim}

\examplecite(17)[Onycha]{HBD}
\examplevolcite(18){1}[Baladan]{NIDB}
\examplecite(12)[Alamoth, Sheminith]{HBD}
\begin{fverbcite}{38}
  \avolcites{1}[Baladan]{NIDB}{4}[Menna]{NIDB}
\end{fverbcite}
\examplecite(46)[799]{mattingly:philistines}
\examplecite(38)[42]{meyers:menorah}
\examplecite(49)[43]{meyers:menorah}
\examplecite(51)[748]{burke:home}
\examplecite[afull](51)[748, §IA2]{burke:home}
\examplecite(14){hanson+etal:apocalypses}
\examplecite(15)[289]{yarbrocollins:earlychristian}
\examplevolcite(71){1}[Baal \mkbibparens{Place}]{ABD}
\examplecite(76){anderson:pepaideumenos}
\examplecite(23)[5]{tiroshsamuelson:kabbalah}
\examplecite(42)[30]{assmann:amun}
\examplecite(84)[441-443]{wise:languagespalestine}
\examplecite(9)[1158]{crawford:genesisiii}
\examplecite(25)[Abilene]{EDB}
\examplecite(26)[3]{dever:abel}
\examplecite(40){stone:isaactest}
\examplecite(18)[Honey]{HBD}
\examplecite(80){schaub:holywar}
\examplecite(54)[510]{popelevison:truth}
\examplevolcite(22){1}[Ara]{IDB}
\examplecite(27){dentan:apple}
\examplecite(32)[Hillel]{IDBSup}
\examplecite(14)[608]{sperling:mount}
\examplevolcite(37){4}[Sosipater]{ISBE}
\examplecite[afull](51)[748]{burke:home}
\examplecite(12)[301]{jacobs:typography}
\examplecite(51)[1184]{negev:petra}
\examplecite(18)[1205]{dothan+dunayevsky:qasiletell}
\examplecite(43)[1805-1806]{reich+shukron:jerusalem}
\examplevolcite(5){1}[Baladan]{NIDB}
\examplecite(72)[215]{graf:arab}
\examplecite(16)[1227]{taylor:protogoras}
\examplecite(57)[Mendicant Friars]{ODCC}
\examplecite(17)[65]{pitard:aleppo}
\examplecite(56)[8-10]{honigmann:libanos}
\examplecite(42)[282]{westbrook:mitgift}
\examplecite(42)[428]{wandrey:ascension}
\examplecite(67)[12]{blum:abraham}
\examplecite(12)[692]{albertz:magie}
\begin{verbcite}
  \nocite{DLNT, DNTB, DOTHB, DPL, RGG}
\end{verbcite}
\exampleabbreviations
\examplebibliography
\examplereferences{https://sblhs2.com/2017/04/13/citing-reference-works-5-topical-dictionaries-and-encyclopedias/}

\section{Citing Reference Works 3: Dictionaries (Word) (4 April 2017)}

\begin{verbatim}
@mvreference{EDNT,
  shorthand = {EDNT},
  editor = {Balz, Horst and Schneider, Gerhard},
  title = {Exegetical Dictionary of the New Testament},
  volumes = {3},
  location = {Grand Rapids},
  publisher = {Eerdmans},
  date = {1990/1993},
  pagination = {subverbo}
}

@inreference{lattke:kakia,
  author = {Lattke, Michael},
  title = {κακία},
  xref = {EDNT},
  volume = {2},
  pages = {237}
}

@mvreference{NIDNTT,
  shorthand = {NIDNTT},
  editor = {Brown, Colin},
  title = {New International Dictionary of New Testament Theology},
  volumes = {4},
  location = {Grand Rapids},
  publisher = {Zondervan},
  date = {1975/1985}
}

@inreference{becker:arrabon,
  author = {Becker, Oswald},
  title = {ἀρραβών},
  xref = {NIDNTT},
  volume = {2},
  pages = {39-40}
}

@inreference{becker+vorlander+brown:gift+pledge+corban,
  author = {Becker, Oswald and Vorländer, Herwart and Brown, Colin},
  title = {Gift, Pledge, Corban},
  xref = {NIDNTT},
  volume = {2},
  pages = {39-44}
}

@mvreference{NIDNTTE,
  shorthand = {NIDNTTE},
  editor = {Silva, Moisés},
  title = {New International Dictionary of New Testament Theology and Exegesis},
  edition = {2},
  volumes = {5},
  location = {Grand Rapids},
  publisher = {Zondervan},
  date = {2014}
}

@inreference{silva:arche,
  author = {Silva, Moisés},
  title = {ἀρχή},
  xref = {NIDNTTE},
  volume = {1},
  pages = {412-418}
}

@mvreference{NIDOTTE,
  shorthand = {NIDOTTE},
  editor = {VanGemeren, Willem A.},
  title = {New International Dictionary of Old Testament Theology and Exegesis},
  volumes = {5},
  location = {Grand Rapids},
  publisher = {Zondervan},
  date = {1997}
}

@inreference{hadley:alyl,
  author = {Hadley, Judith M.},
  title = {אֱלִיל},
  xref = {NIDOTTE},
  volume = {1},
  pages = {411}
}

@inreference{johnston:bth+bta,
  author = {Johnston, Gordon H.},
  title = {בטה/בטא},
  xref = {NIDOTTE},
  volume = {1},
  pages = {642-644}
}

@mvreference{TDNT,
  shorthand = {TDNT},
  editor = {Kittel, Gerhard and Friedrich, Gerhard},
  title = {Theological Dictionary of the New Testament},
  translator = {Bromiley, Geoffrey W.},
  volumes = {10},
  location = {Grand Rapids},
  publisher = {Eerdmans},
  date = {1964/1976}
}

@inreference{delling:magos,
  author = {Delling, Gerhard},
  title = {μάγος},
  xref = {TDNT},
  volume = {4},
  pages = {356-359}
}

@inreference{delling:magos+ktl,
  author = {Delling, Gerhard},
  title = {μάγος κτλ},
  xref = {TDNT},
  volume = {4},
  pages = {356-359}
}

@inreference{herntrich+schrenk:leimma+ktl,
  author = {Herntrich, Volkmar and Schrenk, Gottlob},
  title = {λεῖμμα κτλ},
  xref = {TDNT},
  volume = {4},
  pages = {194-214}
}

@inreference{meyer:manna,
  author = {Meyer, Rudolf},
  title = {μάννα},
  xref = {TDNT},
  volume = {4},
  pages = {462-466}
}

@inreference{maurer:skeuos,
  author = {Maurer, Christian},
  title = {σκεῦος},
  xref = {TDNT},
  volume = {7},
  pages = {358-367}
}

@mvreference{TDOT,
  shorthand = {TDOT},
  editor = {Botterweck, G. Johannes and Ringgren, Helmer and Fabry, Heinz-Josef},
  title = {Theological Dictionary of the Old Testament},
  translator = {Willis, John T.},
  volumes = {15},
  location = {Grand Rapids},
  publisher = {Eerdmans},
  date = {1974/2006}
}

@inreference{ringgren:znh,
  author = {Ringgren, Helmer},
  title = {זָנַח},
  xref = {TDOT},
  volume = {4},
  pages = {105-106}
}

@inreference{hamp+botterweck:dyn,
  author = {Hamp, Vinzenz and Botterweck, G. Johannes},
  title = {דִּין},
  xref = {TDOT},
  volume = {3},
  pages = {187-194}
}

@mvreference{THAT,
  shorthand = {THAT},
  editor = {Jenni, Ernst},
  witheditor = {Westermann, Claus},
  witheditortype = {withassistance},
  title = {Theologisches Handwörterbuch zum Alten Testament},
  volumes = {2},
  location = {Munich and Zürich},
  publisher = {Kaiser and Theologischer Verlag},
  date = {1971/1976},
  langid = {german}
}

@mvreference{ThWAT,
  shorthand = {ThWAT},
  editor = {Botterweck, G. Johannes and Ringgren, Helmer and Fabry, Heinz-Josef},
  title = {Theologisches Wörterbuch zum Alten Testament},
  volumes = {10},
  location = {Stuttgart},
  publisher = {Kohlhammer},
  date = {1970/2000},
  langid = {german}
}

@mvreference{TLNT,
  shorthand = {TLNT},
  author = {Spicq, Ceslas},
  title = {Theological Lexicon of the New Testament},
  editor = {Ernest, James D.},
  translator = {Ernest, James D.},
  volumes = {3},
  location = {Peabody, MA},
  publisher = {Hendrickson},
  date = {1994}
}

@inreference{spicq:apecho,
  author = {Spicq, Ceslas},
  title = {ἀπέχω},
  xref = {TLNT},
  volume = {1},
  pages = {162-168}
}

@inreference{spicq:hikanos,
  author = {Spicq, Ceslas},
  title = {ἱκανός},
  xref = {TLNT},
  volume = {2},
  pages = {217-222}
}

@mvreference{TLOT,
  shorthand = {TLOT},
  editor = {Jenni, Ernst and Westermann, Claus},
  title = {Theological Lexicon of the Old Testament},
  translator = {Biddle, Mark E.},
  volumes = {3},
  location = {Peabody, MA},
  publisher = {Hendrickson},
  date = {1997}
}

@inreference{jenni:lbs,
  author = {Jenni, Ernst},
  title = {לבשׁ},
  xref = {TLOT},
  volume = {2},
  pages = {642-644}
}

@inreference{martin-achard:anh,
  author = {Martin-Achard, Robert},
  title = {ענה II},
  xref = {TLOT},
  volume = {2},
  pages = {931-937}
}

@inreference{westermann+albertz:glh,
  author = {Westermann, Claus and Albertz, Rainer},
  title = {גלה},
  xref = {TLOT},
  volume = {1},
  pages = {314-320}
}

@reference{TWNT,
  shorthand = {TWNT},
  editor = {Kittel, Gerhard and Friedrich, Gerhard},
  title = {Theologische Wörterbuch zum Neuen Testament},
  location = {Stuttgart},
  publisher = {Kohlhammer},
  date = {1932/1979},
  langid = {german}
}

@mvreference{TWOT,
  shorthand = {TWOT},
  editor = {Harris, R. Laird and Archer, Jr., Gleason L. and Waltke, Bruce K.},
  title = { Theological Wordbook of the Old Testament},
  volumes = {2},
  location = {Chicago},
  publisher = {Moody Press},
  date = {1980}
}

@inreference{hartley:qvh,
  author = {Hartley, John E.},
  title = {קָוָה},
  xref = {TWOT},
  volume = {2},
  pages = {791-792}
}
\end{verbatim}

\examplecite(23)[39]{becker:arrabon}
\examplecite(17)[357]{delling:magos}
\examplecite(17){delling:magos+ktl}
\examplecite(45){becker+vorlander+brown:gift+pledge+corban}
\examplecite(32){herntrich+schrenk:leimma+ktl}
\examplecite(41){jenni:lbs}
\examplecite(21){martin-achard:anh}
\examplecite(36){meyer:manna}
\examplecite(38)[464]{meyer:manna}
\examplecite(51)[Ἅγαβος]{EDNT}
\examplecite(42){lattke:kakia}
\examplecite(26)[42]{vorlander:doron}
\examplecite[afull](45){becker+vorlander+brown:gift+pledge+corban}
\examplecite(65)[413]{silva:arche}
\examplecite(57){hadley:alyl}
\examplecite(58)[643]{johnston:bth+bta}
\examplecite(36)[361]{maurer:skeuos}
\examplecite[afull](17)[357]{delling:magos}
\examplecite[afull](17){delling:magos+ktl}
\examplecite[afull](32){herntrich+schrenk:leimma+ktl}
\examplecite(44)[105]{ringgren:znh}
\examplecite(87)[191-194]{hamp+botterweck:dyn}
\examplecite(15)[164]{spicq:apecho}
\examplecite(4)[221]{spicq:hikanos}
\examplecite[full](41){jenni:lbs}
\examplecite(43)[317]{westermann+albertz:glh}
\examplecite(67)[791]{hartley:qvh}
\begin{verbcite}
  \nocite{THAT, ThWAT, TWNT}
\end{verbcite}
\exampleabbreviations
\examplebibliography
\examplereferences{https://sblhs2.com/2017/04/04/citing-reference-works-3-dictionaries-word/}

\section{Citing Reference Works 2: Lexica (30 March 2017)}

\begin{verbatim}
@reference{BAG,
  shorthand = {BAG},
  author = {Bauer, Walter and Arndt, William F. and Gingrich, F. Wilbur},
  title = {Greek-English Lexicon of the New Testament and Other Early Christian
           Literature},
  location = {Chicago},
  publisher = {University of Chicago Press},
  date = {1957},
  options = {shorthandformat=roman},
  pagination = {subverbo}
}

@reference{BAGD,
  shorthand = {BAGD},
  author = {Bauer, Walter and Arndt, William F. and Gingrich, F. Wilbur and Danker,
            Frederick W.},
  title = {Greek-English Lexicon of the New Testament and Other Early Christian
           Literature},
  edition = {2},
  location = {Chicago},
  publisher = {University of Chicago Press},
  date = {1979},
  options = {shorthandformat=roman},
  pagination = {subverbo}
}

@reference{BDAG,
  shorthand = {BDAG},
  author = {Danker, Frederick W. and Bauer, Walter and Arndt, William F. and Gingrich,
            F. Wilbur},
  title = {Greek-English Lexicon of the New Testament and Other Early Christian
           Literature},
  edition = {3},
  location = {Chicago},
  publisher = {University of Chicago Press},
  date = {2000},
  options = {shorthandformat=roman},
  pagination = {subverbo}
}

@reference{BDB,
  shorthand = {BDB},
  author = {Brown, Francis and Driver, S. R. and Briggs, Charles A.},
  title = {A Hebrew and English Lexicon of the Old Testament},
  location = {Oxford},
  publisher = {Clarendon},
  date = {1907},
  options = {shorthandformat=roman},
  pagination = {subverbo}
}

@mvreference{CAD,
  shorthand = {CAD},
  editor = {Gelb, Ignace J. and others},
  title = {The Assyrian Dictionary of the Oriental Institute of the University of
           Chicago},
  volumes = {21},
  location = {Chicago},
  publisher = {The Oriental Institute of the University of Chicago},
  date = {1956/2010},
  pagination = {subverbo}
}

@reference{CDME,
  shorthand = {CDME},
  author = {Faulkner, Raymond O.},
  title = {A Concise Dictionary of Middle Egyptian},
  location = {Oxford},
  publisher = {Griffith Institute},
  date = {1962},
  pagination = {subverbo}
}

@mvreference{CHD,
  shorthand = {CHD},
  editor = {Güterbock, Hans G. and Hoffner, Jr., Harry A. and van den Hout, Theo
            P. J.},
  title = {The Hittite Dictionary of the Oriental Institute of the University of
           Chicago},
  location = {Chicago},
  publisher = {The Oriental Institute of the University of Chicago},
  date = {1980/},
  pagination = {subverbo}
}

@mvreference{DCH,
  shorthand = {DCH},
  editor = {Clines, David J. A.},
  title = {Dictionary of Classical Hebrew},
  volumes = {9},
  location = {Sheffield},
  publisher = {Sheffield Phoenix},
  date = {1993/2016},
  pagination = {subverbo}
}

@mvreference{DNWSI,
  shorthand = {DNWSI},
  author = {Hoftijzer, Jacob and Jongeling, Karen},
  title = {Dictionary of the North-West Semitic Inscriptions},
  volumes = {2},
  location = {Leiden},
  publisher = {Brill},
  date = {1995},
  pagination = {subverbo}
}

@mvreference{DULAT,
  shorthand = {DULAT},
  author = {del Olmo Lete, Gregorio and Sanmartín, Joaquín},
  title = {A Dictionary of the Ugaritic Language in the Alphabetic Tradition},
  translator = {Watson, W. G. E.},
  editor = {Watson, W. G. E.},
  edition = {3},
  volumes = {2},
  location = {Leiden},
  publisher = {Brill},
  date = {2015},
  pagination = {subverbo}
}

@reference{GELS,
  shorthand = {GELS},
  author = {Muraoka, Takamitsu},
  title = {A Greek-English Lexicon of the Septuagint},
  location = {Leuven},
  publisher = {Peeters},
  date = {2009},
  pagination = {subverbo}
}

@reference{HAL,
  shorthand = {HAL},
  author = {Koehler, Ludwig and Baumgartner, Walter and Stamm, Johann J.},
  title = {Hebräisches und aramäisches Lexicon zum Alten Testament},
  edition = {3},
  location = {Leiden},
  publisher = {Brill},
  year = {1995, 2004},
  sortyear = {1995},
  pagination = {subverbo},
  langid = {german}
}

@mvreference{HALOT,
  shorthand = {HALOT},
  author = {Koehler, Ludwig and Baumgartner, Walter and Stamm, Johann J.},
  title = {The Hebrew and Aramaic Lexicon of the Old Testament},
  shorttitle = {HALOT},
  editor = {Richardson, Mervyn E. J.},
  editortype = {Translated and edited under the supervision of},
  volumes = {2},
  location = {Leiden},
  publisher = {Brill},
  date = {2001},
  pagination = {subverbo}
}

@reference{HED,
  shorthand = {HED},
  author = {Puhvel, Jaan},
  title = {Hittite Etymological Dictionary},
  location = {Berlin},
  publisher = {Mouton},
  date = {1984/},
  pagination = {subverbo}
}

@reference{Jastrow,
  shorthand = {Jastrow},
  editor = {Jastrow, Morris},
  editortype = {compiler},
  title = {A Dictionary of the Targumim, the Talmud Babli and Yerushalmi, and the
           Midrashic Literature with an Index of Scriptural Quotations},
  location = {London and New York},
  publisher = {Luzac and G. P. Putnam’s Sons},
  date = {1903},
  pagination = {subverbo},
  options = {shorthandformat=roman}
}

@reference{KBL,
  shorthand = {KBL},
  author = {Koehler, Ludwig and Baumgartner, Walter},
  title = {Lexicon in Veteris Testamenti libros},
  edition = {2},
  location = {Leiden},
  publisher = {Brill},
  date = {1958},
  pagination = {subverbo},
  options = {shorthandformat=roman},
  langid = {latin}
}

@mvreference{Lane,
  shorthand = {Lane},
  author = {Lane, Edward W.},
  title = {An Arabic-English Lexicon},
  volumes = {8},
  origlocation = {London},
  origpublisher = {Williams \& Norgate},
  origdate = {1863},
  location = {Beirut},
  publisher = {Libr.\@ du Liban},
  date = {1980},
  pagination = {subverbo},
  options = {shorthandformat=roman}
}

@reference{LEH,
  shorthand = {LEH},
  editor = {Lust, Johan and Eynikel, Erik and Hauspie, Katrin},
  title = {Greek-English Lexicon of the Septuagint},
  edition = {Rev\adddot},
  location = {Stuttgart},
  publisher = {Deutsche Bibelgesellschaft},
  date = {2003},
  pagination = {subverbo},
  options = {shorthandformat=roman}
}

@reference{LexSyr,
  shorthand = {LexSyr},
  author = {Brockelmann, Carl},
  title = {Lexicon Syriacum},
  edition = {2},
  location = {Halle},
  publisher = {Niemeyer},
  date = {1928},
  pagination = {subverbo}
}

@reference{LS,
  shorthand = {LS},
  author = {Liddell, Henry George and Scott, Robert},
  title = {An Intermediate Greek-English Lexicon: Founded upon the Seventh Edition of
           Liddell and Scott’s Greek-English Lexicon},
  location = {New York},
  publisher = {Harper \& Brothers},
  date = {1889},
  pagination = {subverbo},
  options = {shorthandformat=roman}
}

@reference{LSJ,
  shorthand = {LSJ},
  author = {Liddell, Henry George and Scott, Robert and Jones, Henry Stuart},
  title = {A Greek-English Lexicon},
  edition = {\mkbibordinal{9} \bibsstring{edition} with revised supplement},
  location = {Oxford},
  publisher = {Clarendon},
  date = {1996},
  pagination = {subverbo},
  options = {shorthandformat=roman}
}

@reference{OLD,
  shorthand = {OLD},
  editor = {Glare, P. G. W.},
  title = {Oxford Latin Dictionary},
  location = {Oxford},
  publisher = {Clarendon},
  date = {1982},
  pagination = {subverbo}
}

@reference{PGL,
  shorthand = {PGL},
  editor = {Lampe, Geoffrey W. H.},
  title = {Patristic Greek Lexicon},
  location = {Oxford},
  publisher = {Clarendon},
  date = {1961},
  pagination = {subverbo}
}

@reference{SyrLex,
  shorthand = {SyrLex},
  author = {Sokoloff, Michael},
  title = {A Syriac Lexicon: A Translation from the Latin, Correction, Expansion, and
           Update of C.~Brockelmann’s Lexicon Syriacum},
  location = {Winona Lake, IN and Piscataway, NJ},
  publisher = {Eisenbrauns and Gorgias},
  date = {2009},
  pagination = {subverbo}
}

@mvreference{WAS,
  shorthand = {WÄS},
  author = {Erman, Adolf and Hermann Grapow},
  title = {Wörterbuch der ägyptischen Sprache},
  volumes = {5},
  location = {Leipzig and Berlin},
  publisher = {Hinrichs and Akademie},
  date = {1926/1931},
  pagination = {subverbo}
}
\end{verbatim}


\examplecite(17)[ἐνθύημα]{LEH}
\examplevolcite(21){20}[ubšukkinakku]{CAD}
\examplecite(17)[ἐνθύημα, λεαίνω]{LEH}
\begin{fverbcite}{21}
  \volcites{20}[ubšukkinakku]{CAD}{21}[zaqātu]{CAD}
\end{fverbcite}
\examplevolcite(21){21}[zaqāpu A]{CAD}
\examplecite(25)[\sv~\mkbibquote{יָתֵיב ,יָתֵב}]{Jastrow}
\examplevolcite(21){21}[\sv~\mkbibquote{zaqāpu A}, 1c2′]{CAD}
\examplecite(42)[παρρησία]{BDAG}
\examplecite(31)[מָחָה]{BDB}
\examplecite(31)[\nopp 562a]{BDB}
\examplevolcite(21){21}[\sv~\mkbibquote{zaqāpu A}, 1c2′]{CAD}
\examplecite(14)[nsry]{CDME}
\examplevolcite(5){7}[קנה II]{DCH}
\examplecite(15)[mlk\textsubscript{5}]{DNWSI}
\examplecite(37)[b-r-k I]{DULAT}
\examplecite(28)[θερμασία]{GELS}
\examplecite(17)[חֹזֶה II]{HALOT}
\examplecite(25)[\sv~\mkbibquote{יָתֵיב ,יָתֵב}]{Jastrow}
\examplecite(17)[ἐνθύημα]{LEH}
\examplecite(80)[ὁρμή]{LS}
\examplecite(8)[κύρτος]{LSJ}
\examplecite(27)[coagmentatio]{OLD}
\examplecite(49)[ἐπιστασία]{PGL}
\examplecite(14)[\mkbibbrackets{plš}]{SyrLex}
\examplecite(14)[ܦܠܫ]{SyrLex}
\begin{verbcite}
  \nocite{BAG, BAGD, CHD, HAL, HED, KBL, Lane, LexSyr, WAS}
\end{verbcite}
\exampleabbreviations
\examplereferences{https://sblhs2.com/2017/03/30/citing-reference-works-2-lexica/}

\section{Citing a Chapter from a Single-Authored Work (23 March 2017)}

\begin{verbatim}
@collection{kraft+nickelsburg:1986,
  editor = {Kraft, Robert A. and Nickelsburg, George W. E.},
  title = {Early Judaism and Its Modern Interpreters},
  shorttitle = {Early Judaism},
  location = {Philadelphia and Atlanta},
  publisher = {Fortress and Scholars Press},
  date = {1986}
}

@incollection{attridge:1986,
  author = {Attridge, Harold W.},
  title = {Jewish Historiography},
  pages = {311-343},
  crossref = {kraft+nickelsburg:1986}
}

@book{younger:2016,
  author = {Younger, Jr., K. Lawson},
  title = {A Political History of the Arameans: From Their Origins to the End of Their
           Polities},
  series = {Archaeology and Biblical Studies},
  shortseries = {ABS},
  number = {13},
  location = {Atlanta},
  publisher = {SBL Press},
  date = {2016}
}

@inbook{younger:origins:2016,
  author = {Younger, Jr., K. Lawson},
  title = {The Origins of the Arameans},
  pages = {35-107},
  crossref = {younger:2016}
}
\end{verbatim}

\examplecite(15){attridge:1986}
\examplecite(16){younger:origins:2016}
\exampleabbreviations
\examplebibliography
\examplereferences{https://sblhs2.com/2017/03/23/citing-chapter-single-authored-work/}

\section{Festschrift (16 March 2017)}

\begin{verbatim}
@collection{calduch-benages+vermeylen:1999,
  editor = {Calduch-Benages, Núria and Vermeylen Jacques},
  title = {Treasures of Wisdom: Studies in Ben Sira and the Book of Wisdom;
           Festschrift M. Gilbert},
  series = {Bibliotheca Ephemeridum Theologicarum Lovaniensium},
  shortseries = {BETL},
  number = {143},
  location = {Leuven},
  publisher = {Leuven University Press},
  date = {1999}
}

@collection{campbell+hartin:2016,
  editor = {Campbell, Joan Cecelia and Hartin, P. J.},
  title = {Exploring Biblical Kinship: Festschrift in Honor of John J. Pilch},
  series = {Catholic Biblical Quarterly Monograph Series},
  shortseries = {CBQMS},
  number = {55},
  location = {Washington, DC},
  publisher = {The Catholic Biblical Association of America},
  date = {2016}
}

@collection{kuntzmann:1995,
  editor = {Kuntzmann, Raymond},
  title = {Ce dieu qui vient: Études sur l’Ancien et le Nouveau Testament offertes au
           professeur Bernard Renaud à l’occasion de son soixante-cinquième
           anniversaire},
  series = {Lectio Divina},
  shortseries = {LD},
  number = {159},
  location = {Paris},
  publisher = {Cerf},
  date = {1995},
  langid = {french}
}

@collection{hogan+etal:2017,
  editor = {Hogan, Karina Martin and Goff, Matthew and Wasserman, Emma},
  title = {Pedagogy in Ancient Judaism and Early Christianity},
  series = {Early Judaism and Its Literature},
  shortseries = {EJL},
  number = {41},
  location = {Atlanta},
  publisher = {SBL Press},
  date = {2017}
}
\end{verbatim}

\begin{verbcite}
  \nocite{calduch-benages+vermeylen:1999, campbell+hartin:2016,
    kuntzmann:1995, hogan+etal:2017}
\end{verbcite}
\exampleabbreviations
\examplebibliography
\examplereferences{https://sblhs2.com/2017/03/16/festschrift/}

\section{Separating Multiple Series (2 March 2017)}

\begin{verbatim}
@book{gerhardsson:1961,
  author = {Gerhardsson, Birger},
  title = {Memory and Manuscript},
  subtitle = {Oral Tradition and Written Transmission in Rabbinic Judaism and Early
              Christianity},
  series = {Acta Seminarii Neotestamentici Upsaliensis},
  shortseries = {ASNU},
  number = {22},
  location = {Lund and Copenhagen},
  publisher = {Gleerup and Munksgaard},
  date = {1961}
}

@book{evans:2004,
  editor = {Evans, Craig A.},
  title = {Of Scribes and Sages},
  subtitle = {Early Jewish Interpretation and Transmission of Scripture},
  volumes = {2},
  note = {\citeseries{LSTS} 50--51\multiseriesdelim\citeseries{SSEJC} 9--10},
  location = {Edinburgh},
  publisher = {T\&T Clark},
  date = {2004}
}

@series{LSTS,
  series = {Library of Second Temple Studies},
  shortseries = {LSTS},
  options = {skipbib}
}

@series{SSEJC,
  series = {Studies in Scripture in Early Judaism and Christianity},
  shortseries = {SSEJC},
  options = {skipbib}
}
\end{verbatim}

\examplecite(15){gerhardsson:1961}
\examplecite(16){evans:2004}
\nocite{LSTS, SSEJC}
\exampleabbreviations
\examplebibliography
\examplereferences{https://sblhs2.com/2017/03/02/separating-multiple-series/}

\section{Lengthy Titles (28 February 2017)}

\begin{verbatim}
@book{young:1880,
  author = {Young, Robert},
  title = {Analytical Concordance to the Bible on an Entirely New Plan: Containing
           Every Word in Alphabetical Order, Arranged under Its Hebrew or Greek
           Original\bibellipsis},
  location = {New York},
  publisher = {American Book Exchange},
  date = {1880}
}
\end{verbatim}

\examplecite(4){young:1880}
\examplebibliography
\examplereferences{https://sblhs2.com/2017/02/28/lengthy-titles/}

\section{Twitter (23 February 2017)}

\begin{verbatim}
@online{twitterpage:sblpress,
  website = {SBL Press Twitter page},
  url = {http://tinyurl.com/hwsjd9y}
}

@online{twitter:sblpress,
  title = {SBL Press},
  website = {Twitter},
  url = {http://tinyurl.com/hwsjd9y}
}

@online{twittercomment:sblpress,
  entrysubtype = {comment},
  author = {{SBL Press}},
  type = {Twitter comment},
  date = {2017-02-12},
  url = {http://twitter.com/SBLPress}
}

@article{barone:2017,
  author = {Barone, Joshua},
  title = {Met Museum Makes 375,000 Images Free},
  journaltitle = {New York Times},
  date = {2017-02-07},
  url = {http://tinyurl.com/h6z4ob7}
}
\end{verbatim}

\examplecite[]{twitterpage:sblpress}
\examplecite[]{twitter:sblpress}
\examplecite(2){twittercomment:sblpress}
\begin{fverbcite}{4}
  \footnote{In a comment on Twitter on 12 February 2017, SBL Press
  (\href{https://twitter.com/SBLPress}{@SBLPress}) stated…}
\end{fverbcite}
\begin{fverbcite}{5}
  \footnote{Commenting on the recent \emph{New York Times} article about the
  Met Museum \parencite{barone:2017}, Jane Doe stated on Twitter (12 February
  2017) ….}
\end{fverbcite}
\examplebibliography
\examplereferences{https://sblhs2.com/2017/02/23/twitter/}

\section{Facebook (21 February 2017)}

\begin{verbatim}
@online{fb:sblwomenscholars,
  title = {Society of Biblical Literature – Women Scholars},
  website = {Facebook},
  url = {http://tinyurl.com/z2cyvcb}
}

@online{fbpage:sblwomenscholars,
  website = {Society of Biblical Literature – Women Scholars Facebook page},
  url = {http://tinyurl.com/z2cyvcb}
}

@online{fbcomment:smith,
  entrysubtype = {comment},
  author = {Smith, John},
  type = {personal Facebook comment},
  date = {2017-02-12}
}

@online{fbcomment:doe,
  entrysubtype = {comment},
  author = {Doe, Jane},
  type = {comment on \cite{fbpage:sblwomenscholars}},
  date = {2017-02-12}
}

@article{barone:2017,
  author = {Barone, Joshua},
  title = {Met Museum Makes 375,000 Images Free},
  journaltitle = {New York Times},
  date = {2017-02-07},
  url = {http://tinyurl.com/h6z4ob7}
}
\end{verbatim}

\examplecite(5){fb:sblwomenscholars}
\examplecite[]{fbpage:sblwomenscholars}
\examplecite[]{fbcomment:smith}
\begin{verbcite}
  In a private comment on Facebook on 12 February 2017, John Smith stated …
\end{verbcite}
\examplecite[]{fbcomment:doe}
\begin{verbcite}
  Commenting on the recent New York Times article about the Met Museum
  \parencite{barone:2017}, Jane Doe stated privately on Facebook (12 February
  2017) ….
\end{verbcite}
\examplebibliography
\examplereferences{https://sblhs2.com/2017/02/21/facebook/}

\section{\emph{Progymnasmata} (16 February 2017)}

\begin{verbatim}
@ancienttext{nicolaus:prog,
  author = {Nicolaus},
  title = {Progymnasmata},
  shorttitle = {Prog\adddot},
  xref = {felten}
}

@ancienttext{aphthonius:prog,
  author = {Aphthonius},
  title = {Progymnasmata},
  shorttitle = {Prog\adddot},
  xref = {rabe}
}

@ancienttext{theon:prog,
  author = {Theon},
  title = {Progymnasmata},
  shorttitle = {Prog\adddot},
  xref = {spengel}
}

@ancienttext{libanius:prog,
  author = {Libanius},
  title = {Progymnasmata},
  shorttitle = {Prog\adddot},
  xref = {foerster}
}

@book{felten,
  shorthand = {Felten},
  editor = {Felten, Joseph},
  title = {Nicolai Progymnasmata},
  location = {Leipzig},
  publisher = {Teubner},
  date = {1913},
  options = {shorthandformat=roman}
}

@book{foerster,
  shorthand = {Foerster},
  editor = {Foerster, Richard},
  title = {Progymnasmata, Argumenta orationum Demosthenicarum},
  volume = {8},
  maintitle = {Libanii Opera},
  location = {Leipzig},
  publisher = {Teubner},
  date = {1915},
  options = {shorthandformat=roman}
}

@book{gibson,
  shorthand = {Gibson},
  author = {Gibson, Craig A.},
  title = {Libanius's Progymnasmata: Model Exercises in Greek Prose Composition and Rhetoric},
  series = {Writings from the Greco-Roman World},
  shortseries = {WGRW},
  number = {27},
  location = {Atlanta},
  publisher = {Society of Biblical Literature},
  date = {2008},
  options = {shorthandformat=roman}
}

@book{kennedy:2003,
  author = {Kennedy, George A.},
  title = {Progymnasmata: Greek Textbooks of Prose Composition and Rhetoric},
  series = {Writings from the Greco-Roman World},
  shortseries = {WGRW},
  number = {10},
  location = {Atlanta},
  publisher = {Society of Biblical Literature},
  date = {2003}
}

@book{patillonandbolognesi,
  shorthand = {Patillon and Bolognesi},
  editor = {Patillon, Michel and Bolognesi, Giancarlo},
  title = {Aelius Théon: Progymnasmata},
  edition = {Edition Budé},
  location = {Paris},
  publisher = {Belles Lettres},
  date = {1997},
  options = {shorthandformat=roman},
  langid = {fr}
}

@book{rabe:1913,
  shorthand = {Rabe},
  editor = {Rabe, Hugo},
  title = {Hermogenis Opera},
  location = {Leipzig},
  publisher = {Teubner},
  date = {1913},
  options = {shorthandformat=roman}
}


@book{rabe,
  shorthand = {Rabe},
  editor = {Rabe, Hugo},
  title = {Aphthonii Progymnasmata},
  location = {Leipzig},
  publisher = {Teubner},
  date = {1926},
  options = {shorthandformat=roman}
}

@book{spengel,
  shorthand = {Spengel},
  author = {von Spengel, Leonhard},
  title = {Rhetores Graeci},
  volumes = {3},
  location = {Leipzig},
  publisher = {Teubner},
  date = {1853/1856},
  options = {shorthandformat=roman}
}

@inreference{russell:2003,
  author = {Russell, Donald A.},
  title = {Progymnasmata},
  pages = {1253},
  booktitle = {The Oxford Classical Dictionary},
  editor = {Hornblower, Simon and Spawforth, Antony},
  edition = {3},
  location = {Oxford},
  publisher = {Oxford University Press},
  date = {2003}
}
\end{verbatim}

\examplecite[][(4)]{nicolaus:prog}
\examplecite[][(3)]{aphthonius:prog}
\examplevolcite[]{2}[(4)\nopnfmt{73,28--74,15}]{theon:prog}
\examplecite[][(1)]{libanius:prog}
\examplecite[][(1.2)]{libanius:prog}
\examplecite[][(1.2.2)]{libanius:prog}
\begin{verbcite}
  \nocite{gibson, kennedy:2003, patillonandbolognesi, rabe:1913, russell:2003}
\end{verbcite}
\exampleabbreviations
\examplebibliography

\subsection{Notes}

\pkg{biblatex-sbl} follows the post at
\url{https://sblhs2.com/2017/04/13/citing-reference-works-5-topical-dictionaries-and-encyclopedias/}
and formats the title of the \emph{Oxford Classical Dictionary} article as
“Progymnasmata” rather than \emph{Progymnasmata}.

\examplereferences{https://sblhs2.com/2017/02/16/progymnasmata/}

\section{Jacoby and \emph{FGrHist} (26 January 2017)}

\begin{verbatim}
@mvcollection{FGrHist,
  shorthand = {FGrHist},
  editor = {Jacoby, Felix},
  title = {Die Fragmente der griechischen Historiker},
  location = {Leiden},
  publisher = {Brill},
  date = {1954/1964},
  pagination = {author}
}
\end{verbatim}

\examplecite[][854 F 3a]{FGrHist}
\examplecite[][854 T 1]{FGrHist}
\examplecite[][1 F 38--103]{FGrHist}
\examplecite[][235]{FGrHist}
\exampleabbreviations
\examplereferences{https://sblhs2.com/2017/01/26/jacoby-fgrhist/}

\section{Sifre Numbers: Update (24 January 2017)}

\begin{verbatim}
@mvbook{neusner:1998,
  author = {Neusner, Jacob},
  title = {Sifré Numbers},
  parts = {4},
  volume = {12},
  maintitle = {The Components of the Rabbinic Documents: From the Whole to the Parts},
  series = {South Florida Academic Commentary Series},
  number = {104-107},
  location = {Atlanta},
  publisher = {Scholars Press},
  date = {1998}
}
\end{verbatim}

\begin{verbcite}
  \nocite{neusner:1998}
\end{verbcite}
\examplebibliography
\examplereferences{https://sblhs2.com/2017/01/24/sifre-numbers-update/}

\section{Cf., See, and See Also (12 January 2017)}

\begin{verbatim}
@book{boase+frechette:2016,
  author = {Boase, Elizabeth and Frechette, Christopher G.},
  title = {Bible through the Lens of Trauma},
  series = {Semeia Studies},
  shortseries = {SemeiaSt},
  number = {86},
  location = {Atlanta},
  publisher = {SBL Press},
  date = {2016}
}

@book{finkelstein+mazar:2007,
  author = {Finkelstein, Israel and Mazar, Amihai},
  title = {The Quest for the Historical Israel: Debating Archaeology and the History
           of Early Israel},
  editor = {Schmidt, Brian B.},
  series = {Archaeology and Biblical Studies},
  shortseries = {ABS},
  number = {17},
  location = {Atlanta},
  publisher = {Society of Biblical Literature},
  date = {2007}
}
\end{verbatim}

\begin{fverbcite}{2}
  \autocite[For a more recent study of trauma in the Bible,
    see][]{boase+frechette:2016}
\end{fverbcite}
\begin{fverbcite}{7}
  \autocite[This is the position taken here. For an alternative position,
    see][]{finkelstein+mazar:2007}
\end{fverbcite}
\exampleabbreviations
\examplebibliography
\examplereferences{https://sblhs2.com/2017/01/12/cf-see-also/}

\section{\emph{Idem} (10 January 2017)}

\begin{verbatim}
@book{moore:2010,
  author = {Moore, Stephen D.},
  title = {The Bible in Theory: Critical and Postcritical Essays},
  series = {Resources for Biblical Study},
  shortseries = {RBS},
  number = {57},
  location = {Atlanta},
  publisher = {Society of Biblical Literature},
  date = {2010}
}

@book{moore:2014,
  author = {Moore, Stephen D.},
  title = {Untold Tales from the Book of Revelation: Sex and Gender, Empire and
           Ecology},
  series = {Resources for Biblical Study},
  shortseries = {RBS},
  number = {79},
  location = {Atlanta},
  publisher = {SBL Press},
  date = {2014}
}

@incollection{cadwallader:2016,
  author = {Cadwallader, Alan H.},
  title = {One Grave, Two Women, One Man: Complicating Family Life at Colossae},
  booktitle = {Stones, Bones, and the Sacred: Essays on Material Culture and Ancient
               Religion in Honor of Dennis E. Smith},
  editor = {Cadwallader, Alan H.},
  series = {Early Christianity and Its Literature},
  shortseries = {ECL},
  number = {21},
  location = {Atlanta},
  publisher = {SBL Press},
  date = {2016},
  pages = {157-194}
}

@incollection{harrison:2016,
  author = {Harrison, James R.},
  title = {Introduction: Excavating the Urban Life of Roman Corinth},
  booktitle = {The First Urban Churches 2: Roman Corinth},
  editor = {Harrison, James R. and Welborn, L. L.},
  series = {Writings from the Greco-Roman World Supplement Series},
  shortseries = {WGRWSup},
  number = {8},
  location = {Atlanta},
  publisher = {SBL Press},
  date = {2016},
  pages = {1-45}
}
\end{verbatim}

\begin{fverbcite}{8}
  \autocites{moore:2010}{moore:2014}
\end{fverbcite}
\examplecite(9)[159]{cadwallader:2016}
\examplecite(9)[44]{harrison:2016}
\begin{verbcite}
  \citereset
\end{verbcite}
\examplecite(8){moore:2010}
\begin{fverbcite}{9}
  \autocites[See][25]{moore:2010}[57]{moore:2014}
\end{fverbcite}
\exampleabbreviations
\examplebibliography
\examplereferences{https://sblhs2.com/2017/01/10/idem/}

\section{Subsequent Bibliographic References (5 January 2017)}

\begin{verbatim}
@book{gruca-macaulay:2016,
  author = {Gruca-Macaulay, Alexandra},
  title = {Lydia as a Rhetorical Construct in Acts},
  shorttitle = {Lydia as a Rhetorical Construct},
  series = {Emory Studies in Early Christianity},
  shortseries = {ESEC},
  number = {18},
  location = {Atlanta},
  publisher = {SBL Press},
  date = {2016}
}

@incollection{patterson:2016,
  author = {Patterson, Stephen J.},
  title = {The Baptists of Corinth: Paul, the Partisans of Apollos, and the History of
           Baptism in Nascent Christianity},
  booktitle = {Stones, Bones, and the Sacred: Essays on Material Culture and Ancient
               Religion in Honor of Dennis E. Smith},
  editor = {Cadwallader, Alan H.},
  series = {Early Christianity and Its Literature},
  shortseries = {ECL},
  number = {21},
  location = {Atlanta},
  publisher = {SBL Press},
  date = {2016},
  pages = {315-327}
}
\end{verbatim}

\examplecite(4){gruca-macaulay:2016}
\examplecite(6){patterson:2016}
\examplecite(8)[5]{gruca-macaulay:2016}
\examplecite(9)[320]{patterson:2016}
\begin{verbcite}
  \citereset
\end{verbcite}
\examplecite(4){gruca-macaulay:2016}
\examplecite(5)[5]{gruca-macaulay:2016}
\examplecite(6){patterson:2016}
\examplecite(7)[321]{patterson:2016}
\examplecite(6)[320]{patterson:2016}
\examplecite(7)[320]{patterson:2016}
\begin{fverbcite}{9}
  \autocites[5]{gruca-macaulay:2016}[320]{patterson:2016}
\end{fverbcite}
\examplecite(10)[322]{patterson:2016}
\exampleabbreviations
\examplebibliography

\subsection{Notes}

SBL now discourages the use of \emph{ibid.}, so \pkg{biblatex-sbl} disables
this feature by default. See
\url{https://sblhs2.com/2018/02/01/cms-update-ibid/}.

\examplereferences{https://sblhs2.com/2017/01/05/subsequent-bibliographic-references/}

\section{Research Methods (29 December 2016)}

\begin{verbatim}
@abbreviation{SRI,
  entrysubtype = {acronym},
  shorthand = {SRI},
  definition = {sociorhetorical interpretation}
}
\end{verbatim}

\begin{verbcite}
  \nocite{SRI}
\end{verbcite}
\exampleabbreviations
\examplereferences{https://sblhs2.com/2016/12/29/research-methods/}

\section{Sifre Deuteronomy (22 December 2016)}

\begin{verbatim}
@book{Finkelstein,
  shorthand = {Finkelstein},
  author = {Finkelstein, Louis},
  title = {Siphre ad Deuteronomium},
  location = {New York},
  publisher = {Jewish Theological Seminary of America},
  date = {1969},
  options = {shorthandformat=roman}
}

@ancienttext{sifredeut,
  shorttitle = {Sifre Deut\adddot},
  title = {Sifre Deuteronomy},
  xref = {Finkelstein}
}

@book{hammer:1986,
  author = {Hammer, Reuven},
  title = {Sifre: A Tannaitic Commentary on the Book of Deuteronomy},
  series = {Yale Judaica Series},
  number = {24},
  location = {New Haven},
  publisher = {Yale University Press},
  date = {1986}
}

@mvbook{neusner:1987,
  author = {Neusner, Jacob},
  title = {Sifre to Deuteronomy: An Analytical Translation},
  volumes = {2},
  series = {Brown Judaic Studies},
  shortseries = {BJS},
  number = {98, 101},
  location = {Atlanta},
  publisher = {Scholars Press},
  date = {1987}
}
\end{verbatim}

\examplecite[][(49)]{sifredeut}
\examplecite[][(49 on 11:22)]{sifredeut}
\begin{verbcite}
  \nocite{hammer:1986, neusner:1987}
\end{verbcite}
\exampleancientsources
\examplesecondarysources
\examplebibliography
\examplereferences{https://sblhs2.com/2016/12/22/sifre-deuteronomy/}

\section{Herodian of Antioch (20 December 2016)}

\begin{verbatim}
@ancienttext{herodian:hist,
  author = {Herodian},
  title = {History of the Empire from the Death of Marcus},
  shorttitle = {Hist\adddot},
  translator = {Whittaker}
}
\end{verbatim}

\examplecite[paren][(2.2)]{herodian:hist}
\examplecite[ptrans][(2.2)]{herodian:hist}
\exampleancientsources
\examplereferences{https://sblhs2.com/2016/12/20/herodian-antioch/}

\section{Sifre Numbers (15 December 2016)}

\begin{verbatim}
@book{Horovitz,
  shorthand = {Horovitz},
  editor = {Horovitz, Haim Shaul Horovitz},
  title = {Sifre ʻal sefer Ba-midbar ve-sifre zuṭa},
  location = {Leipzig},
  publisher = {Fock},
  date = {1917},
  options = {shorthandformat=roman}
}

@ancienttext{sifrenum,
  shorttitle = {Sifre Num\adddot},
  title = {Sifre Numbers},
  xref = {Horovitz}
}

@book{neusner:1998,
  author = {Neusner, Jacob},
  title = {Sifré Numbers},
  parts = {4},
  volume = {12},
  maintitle = {The Components of the Rabbinic Documents: From the Whole to the Parts},
  series = {South Florida Academic Commentary Series},
  number = {104-107},
  location = {Atlanta},
  publisher = {Scholars Press},
  date = {1998}
}
\end{verbatim}

\examplecite[][(42)]{sifrenum}
\examplecite[][(42.2)]{sifrenum}
\examplecite[][(42.2.3)]{sifrenum}
\examplecite[][(\pnfmt{42.2.3F})]{sifrenum}
\examplecite[][(42)\nopnfmt{46,9--10}]{sifrenum}
\begin{verbcite}
  \nocite{neusner:1998}
\end{verbcite}
\exampleancientsources
\examplesecondarysources
\examplebibliography
\examplereferences{https://sblhs2.com/2016/12/15/sifre-numbers/}

\section{Polybius of Megalopolis (13 December 2016)}

\begin{verbatim}
@ancienttext{polybius:hist,
  author = {Polybius},
  title = {Historiae},
  shorttitle = {Hist\adddot},
  translator = {Paton}
}
\end{verbatim}

\examplecite[paren][(2.2)]{polybius:hist}
\examplecite[ptrans][(2.2)]{polybius:hist}
\exampleabbreviations
\examplereferences{https://sblhs2.com/2016/12/13/polybius-megalopolis/}

\section{Diodorus Siculus (29 November 2016)}

\begin{verbatim}
@ancienttext{diodorussiculus:bibhist,
  author = {{Diodorus Siculus}},
  title = {Bibliotheca historica},
  shorttitle = {Bib\addotspace Hist\adddot},
  translator = {Oldfather}
}
\end{verbatim}

\examplecite[paren][(9.1)]{diodorussiculus:bibhist}
\examplecite[ptrans][(9.1)]{diodorussiculus:bibhist}
\exampleabbreviations
\examplereferences{https://sblhs2.com/2016/11/29/diodorus-siculus/}

\section{Le Monde de la Bible (22 November 2016)}

\begin{verbatim}
@book{lance:1990,
  author = {Lance, Darrell H.},
  title = {Archéologie et Ancien Testament},
  series = {Le Monde de la Bible},
  shortseries = {MdB},
  number = {21},
  location = {Geneva},
  publisher = {Labor et Fides},
  date = {1990},
  langid = {french}
}

@article{magness:2003,
  author = {Magness, Jodi},
  title = {Dernieres nouvelles de Qumrân},
  journaltitle = {Le Monde de la Bible},
  shortjournal = {MdB},
  volume = {151},
  date = {2003},
  pages = {14-17},
  langid = {french}
}

@journal{WoB,
  journaltitle = {The World of the Bible},
  shortjournal = {WoB}
}

@article{cannuyer:2016,
  author = {Cannuyer, Christian},
  title = {Vivere l’eternità di Rê e Osiride},
  journaltitle = {Il Mondo della Bibbia},
  shortjournal = {MdelB},
  volume = {132},
  date = {2016},
  pages = {2-9},
  langid = {italian}
}

@article{zimmerling:2016,
  author = {Zimmerling, Peter},
  title = {Was ist Mystik? Hintergründe und Zugänge},
  shorttitle = {Was ist Mystik?},
  journaltitle = {Welt und Umwelt der Bibel},
  shortjournal = {WUB},
  volume = {3},
  date = {2016},
  pages = {8-13},
  langid = {german}
}

@book{mondedelabible:2016,
  author = {{Le Monde de la Bible}},
  sortname = {{Monde de la Bible}},
  title = {Des chrétiens vers Pékin: Sur la route de la soie},
  location = {Montrouge cedex},
  publisher = {Bayard},
  date = {2016},
  langid = {french}
}

@book{worldofthebible:2015,
  author = {{The World of the Bible}},
  sortname = {{World of the Bible}},
  title = {The Bible and the Koran},
  location = {Montrouge cedex},
  publisher = {Bayard},
  date = {2015}
}

@book{alexander+etal:1996,
  author = {Alexander, Pat and Drane, John William and Field, David and Millard, Alan
            and Huser, Etienne},
  title = {Le monde de la Bible},
  shorttitle = {Monde de la Bible},
  sorttitle = {Monde de la Bible},
  location = {Bale and Turnhout},
  publisher = {Brunnen and Brepols},
  date = {1996}
}

@book{lemaire:1998,
  author = {Lemaire, André},
  title = {Le monde de la Bible},
  shorttitle = {Monde de la Bible},
  sorttitle = {Monde de la Bible},
  location = {Paris},
  publisher = {Gallimard},
  date = {1998},
  langid = {french}
}
\end{verbatim}

\begin{verbcite}
  \nocite{lance:1990, magness:2003, WoB, cannuyer:2016, zimmerling:2016,
    mondedelabible:2016, worldofthebible:2015, alexander+etal:1996,
    lemaire:1998}
\end{verbcite}
\exampleabbreviations
\examplebibliography
\examplereferences{https://sblhs2.com/2016/11/22/le-monde-de-la-bible/}

\section{Separating Author Names (8 November 2016)}

\begin{verbatim}
@book{talbert:1992,
  author = {Talbert, Charles H.},
  title = {Reading John},
  subtitle = {A Literary and Theological Commentary on the Fourth Gospel and the
              Johannine Epistles},
  location = {New York},
  publisher = {Crossroad},
  date = {1992}
}

@book{tigay:1985,
  editor = {Tigay, Jeffrey H.},
  title = {Empirical Models for Biblical Criticism},
  shorttitle = {Empiracle Models},
  location = {Philadelphia},
  publisher = {University of Pennsylvania Press},
  date = {1985}
}

@book{robinson+koester:1971,
  author = {Robinson, James M. and Koester, Helmut},
  title = {Trajectories through Early Christianity},
  location = {Philadelphia},
  publisher = {Fortress},
  date = {1971}
}

@book{kaltner+mckenzie:2002,
  editor = {Kaltner, John and McKenzie, Steven L.},
  title = {Beyond Babel},
  subtitle = {A Handbook for Biblical Hebrew and Related Languages},
  series = {Resources for Biblical Study},
  shortseries = {RBS},
  number = {42},
  location = {Atlanta},
  publisher = {Society of Biblical Literature},
  date = {2002}
}

@book{boda+floyd+toffelmire:2015,
  editor = {Boda, Mark J. and Floyd, Michael H. and Colin M. Toffelmire},
  title = {The Book of the Twelve and the New Form Criticism},
  series = {Ancient Near East Monographs},
  shortseries = {ANEM},
  number = {10},
  location = {Atlanta},
  publisher = {SBL Press},
  date = {2015}
}

@book{oates+etal:2001,
  editor = {Oates, John F. and Willis, William H. and Bagnall, Roger S. and Worp,
            Klass A.},
  title = {Checklist of Editions of Greek and Latin Papyri, Ostraca, and Tablets},
  edition = {5},
  series = {Bulletin of the American Society of Papyrologists, Supplements},
  shortseries = {BASPSup},
  number = {9},
  location = {Oakville, CT},
  publisher = {American Society of Papyrologists},
  date = {2001}
}

@book{harrison+welborn:forthcoming,
  editor = {Harrison, James R. and Welborn, L. L.},
  title = {The First Urban Churches 2},
  subtitle = {Roman Corinth},
  shorttitle = {Roman Corinth},
  series = {Writings from the Greco-Roman World Supplement Series},
  shortseries = {WGRWSup},
  location = {Atlanta},
  publisher = {SBL Press},
  pubstate = {forthcoming}
}
\end{verbatim}

\examplecite[]{talbert:1992}
\examplecite[]{tigay:1985}
\examplecite[]{robinson+koester:1971}
\examplecite[]{kaltner+mckenzie:2002}
\examplecite[]{boda+floyd+toffelmire:2015}
\examplecite[]{oates+etal:2001}
\begin{verbcite}
  \nocite{harrison+welborn:forthcoming}
\end{verbcite}
\exampleabbreviations
\examplebibliography

\subsection{Notes}

The Blog incorrectly uses “Jeffrey H. Tigay, ed.” rather than “Tigay, Jeffrey
H., ed.” for the bibliography format of this reference.

\examplereferences{https://sblhs2.com/2016/11/08/separating-author-names/}

\section{Hyphens, En Dashes, and Em Dashes (1 November 2016)}

\begin{verbatim}
@book{moore:2010,
  author = {Moore, Stephen D.},
  title = {The Bible in Theory: Critical and Postcritical Essays},
  series = {Resources for Biblical Study},
  shortseries = {RBS},
  number = {57},
  location = {Atlanta},
  publisher = {Society of Biblical Literature},
  date = {2010}
}

@book{moore:2006,
  author = {Moore, Stephen D.},
  title = {Empire and Apocalypse: Postcolonialism and the New Testament},
  location = {Sheffield},
  publisher = {Sheffield Phoenix},
  date = {2006}
}
\end{verbatim}

\begin{verbcite}
  \nocite{moore:2010, moore:2006}
\end{verbcite}
\exampleabbreviations
\examplebibliography
\examplereferences{https://sblhs2.com/2016/11/01/hyphens-en-dashes-em-dashes/}

\section{Formatting Titles (27 October 2016)}

% TODO: finish section

\begin{verbcite}
  Ron Hendel's \citetitle*{hendel:2016}
\end{verbcite}
\begin{verbcite}
  the \citejournal*{mckenzie:2010}
\end{verbcite}
\begin{verbcite}
  Philo's \citetitle*{philo:contempl}
\end{verbcite}
\begin{verbcite}
  \citeauthor{mckenzie:2010}'s \citejournal{mckenzie:2010} article
  \citetitle*{mckenzie:2010}
\end{verbcite}
\begin{verbcite}
  chapter 3, \citetitle*{hendel:2016:ch3}, in Ron Hendel's
  \citefield{hendel:2016:ch3}[booktitle]{booktitle}
\end{verbcite}
\begin{verbcite}
  \citeauthor{aymer:2016}'s essay \citetitle*{aymer:2016} in the volume
  \citetitle*{byron+lovelace:2016}
\end{verbcite}
\begin{verbcite}
  part 3 of \citefield{byron+lovelace:2016:part3}[booktitle]{booktitle},
  \citetitle*{byron+lovelace:2016:part3}
\end{verbcite}
\exampleabbreviations
\examplebibliography
\examplereferences{https://sblhs2.com/2016/10/27/formatting-titles/}

\section{Plutarch’s \emph{Moralia} (25 October 2016)}

\begin{verbatim}
@ancienttext{plutarch:isos,
  author = {Plutarch},
  title = {De Iside et Osiride},
  shorttitle = {Is.\@ Os\adddot},
  xref = {plutarch:isisoris}
}

@ancienttext{plutarch:alexfort,
  author = {Plutarch},
  title = {De Alexandri magni fortuna aut virtute},
  shorttitle = {Alex.\@ fort\adddot},
  xref = {plutarch:romanquestions}
}

@ancienttext{plutarch:quaestconv,
  author = {Plutarch},
  title = {Quaestionum convivialum libri IX},
  shorttitle = {Quaest.\@ conv\adddot},
  translator = {Sandbach},
  xref = {plutarch:tabletalk7}
}

@ancienttext{plutarch:cohibira,
  author = {Plutarch},
  title = {De cohibenda ira},
  shorttitle = {Cohib.\@ ira\adddot},
  xref = {plutarch:canvirtue}
}

@book{plutarch:isisoris,
  author = {Plutarch},
  title = {Isis and Oris; The E at Delphi; The Oracles at Delphi No Longer Given in
           Verse; The Obsolescence of Oracles},
  translator = {Babbitt, Frank Cole},
  series = {Loeb Classical Library},
  shortseries = {LCL},
  location = {Cambridge},
  publisher = {Harvard University Press},
  date = {1936}
}

@book{plutarch:romanquestions,
  author = {Plutarch},
  title = {The Roman Questions; The Greek Questions; Greek and Roman Parallel Stories;
           On the Fortune of the Romans; On the Fortune or the Virtue of Alexander;
           Were the Athenians More Famous in War or in Wisdom?},
  translator = {Babbitt, Frank Cole},
  series = {Loeb Classical Library},
  shortseries = {LCL},
  location = {Cambridge and London},
  publisher = {Harvard University Press and William Heinemann},
  date = {1936}
}

@book{plutarch:tabletalk7,
  author = {Plutarch},
  title = {Table-Talk, Books 7–9; The Dialogue on Love},
  translator = {Minar, Jr., Edwin L. and Sandbach, F. H. and Helmbold, W. C.},
  series = {Loeb Classical Library},
  shortseries = {LCL},
  location = {Cambridge and London},
  publisher = {Harvard University Press and William Heinemann},
  date = {1961}
}

@book{plutarch:canvirtue,
  author = {Plutarch},
  title = {Can Virtue be Taught?; On Moral Virtue; On the Control of Anger; On
           Tranquility of Mind; On Brotherly Love; On Affection for Offspring; Whether
           Vice Be Sufficient to Cause Unhappiness; Whether the Affections of the Soul
           are Worse than those of the Body; Concerning Talkativeness; On Being a
           Busybody},
  translator = {Helmbold, W. C.},
  series = {Loeb Classical Library},
  shortseries = {LCL},
  location = {Cambridge},
  publisher = {Harvard University Press},
  date = {1939}
}
\end{verbatim}

\examplecite[][(46 \mkbibparens{369a})]{plutarch:isos}
\examplecite[][(1.5)]{plutarch:alexfort}
\examplecite[][(9.2.3)]{plutarch:quaestconv}
\examplecite[][(1.5 \mkbibparens{328c})]{plutarch:alexfort}
\examplecite[paren][(9.2.3 \mkbibparens{738a})]{plutarch:quaestconv}
\begin{verbcite}
  Surely we should allow no place to anger even in jest, for that brings
  enmity in where friendliness was; nor in learned discussions, for that turns
  love of learning into strife. \ptranscite[(16
  \mkbibparens{462b})]{plutarch:cohibira}
\end{verbcite}
\exampleancientsources
\examplesecondarysources
\examplebibliography

\subsection{Notes}

\begin{itemize}
  \item The SBLHS blog has a typo for the translator. It should be
    \emph{Helmbold}, not \emph{Helmhold}.
  \item I've also chosen to use leave out the LCL qualifier for the translator
    as elsewhere SBL does not include the edition.
\end{itemize}

\examplereferences{https://sblhs2.com/2016/10/25/plutarchs-moralia/}

\section{Inscriptions and Papyri (20 October 2016)}

\begin{verbatim}
@mvbook{IEph,
  shorthand = {IEph},
  editor = {Wankel, Hermann and others},
  title = {Die Inschriten von Ephesos},
  volumes = {8},
  location = {Bonn},
  publisher = {Habelt},
  date = {1979/1984},
  pagination = {text},
  options = {shorthandformat=roman},
  langid = {german}
}

@mvbook{P.Ryl,
  shorthand = {P.Ryl},
  editor = {Hunt, Arthur S. and others},
  title = {Catalogue of the Greek and Latin Papyri in the John Rylands Library,
           Manchester},
  volumes = {4},
  location = {Manchester},
  publisher = {Manchester University Press},
  date = {1911/1952},
  pagination = {text},
  options = {shorthandformat=roman}
}

@mvbook{BGU,
  shorthand = {BGU},
  title = {Aegyptische Urkunden aus den Königlichen Staatlichen Museen zu Berlin,
           Griechische Urkunden},
  volumes = {15},
  location = {Berlin},
  publisher = {Weidmann},
  date = {1895/1937},
  pagination = {text},
  langid = {german}
}

@mvbook{IG,
  shorthand = {IG},
  title = {Inscriptions Graecae},
  edition = {Editio Minor},
  location = {Berlin},
  publisher = {de Gruyter},
  date = {1924/},
  pagination = {text}
}
\end{verbatim}

\examplevolcite[]{1a}{IEph}
\examplevolcite[]{3}{P.Ryl}
\examplevolcite[]{2}{BGU}
\examplevolcite[]{1a}[7.2]{IEph}
\examplevolcite[]{3}[457]{P.Ryl}
\examplevolcite[]{2}[447]{BGU}
\examplevolcite[]{7.1}[3025]{IEph}
\examplevolcite[]{1a}[7.2, \lno~11]{IEph}
\examplevolcite[]{1a}[8, \llno~16-17]{IEph}
\examplevolcite[]{2}{IG}
\examplevolcite[]{9.2}[21]{IG}
\exampleabbreviations
\examplereferences{https://sblhs2.com/2016/10/20/inscriptions-papyri/}

\section{\emph{Aufstieg und Niedergang der römischen Welt (ANRW)} (18 October
  2016)}

\begin{verbatim}
@mvcollection{ANRW,
  shorthand = {ANRW},
  editor = {Temporini, Hildegard and Haase, Wolfgang},
  maintitle = {Aufstieg und Niedergang der römischen Welt},
  mainsubtitle = {Geschichte und Kultur Roms im Spiegel der neueren Forschung},
  part = {2},
  title = {Principat},
  location = {Berlin},
  publisher = {de Gruyter},
  date = {1972/},
  langid = {german},
  options = {usetitle=false}
}

@incollection{anderson:pepaideumenos,
  author = {Anderson, Graham},
  title = {The \mkbibemph{pepaideumenos} in Action},
  subtitle = {Sophists and Their Outlook in the Early Empire},
  shorttitle = {\mkbibemph{Pepaideumenos}},
  xref = {ANRW},
  volume = {33},
  part = {1},
  pages = {80-208}
}
\end{verbatim}

\examplecite(76){anderson:pepaideumenos}
\examplecite(79)[86]{anderson:pepaideumenos}
\exampleabbreviations
\examplebibliography
\examplereferences{https://sblhs2.com/2016/10/18/anrw/}

\section{Placement of Citations: Traditional Style (11 October 2016)}

\begin{verbatim}
@commentary{jewett:2007,
  author = {Jewett, Robert},
  title = {Romans: A Commentary},
  series = {Hermeneia},
  location = {Minneapolis},
  publisher = {Fortress},
  date = {2007}
}

@commentary{kasemann:1981,
  author = {Käsemann, Ernst},
  title = {Commentary on Romans},
  translator = {Bromiley, Geoffrey W.},
  location = {Grand Rapids},
  publisher = {Eerdmans},
  date = {1981}
}

@book{ramsay:1906,
  author = {Ramsay, William Mitchell},
  title = {The Letters to the Seven Churches of Asia and Their Place in the Plan of
           the Apocalypse},
  shorttitle = {Letters to the Seven Churches},
  edition = {2},
  location = {London},
  publisher = {Hodder \& Stoughton},
  date = {1906}
}

@book{ricoeur:1978,
  author = {Ricoeur, Paul},
  title = {The Rule of Metaphor: Multi-disciplinary Studies in the Creation of Meaning
           in Language},
  editor = {Czerny, Robert},
  editortype = {translator},
  witheditor = {McLaughlin, Kathleen and family=Constell, given=John,
                postnominal=S.J.},
  location = {London},
  publisher = {Routledge \& Kegan Paul},
  date = {1978}
}

@book{koller:2004,
  author = {Koller, Veronika},
  title = {Metaphor and Gender in Business Media Discourse: A Critical Cognitive
           Study},
  location = {Hampshire},
  publisher = {Palgrave Macmillan},
  date = {2004}
}

@article{gordon:1953,
  author = {Gordon, Cyrus H.},
  title = {Near East Seals in Princeton and Philadelphia},
  journaltitle = {Orientalia \mkbibparens{NS}},
  shortjournal = {Or},
  volume = {22},
  date = {1953},
  pages = {242-250}
}

@book{gordon:1965,
  author = {Gordon, Cyrus H.},
  title = {Ugaritic Textbook},
  series = {Analecta Orientalia},
  shortseries = {AnOr},
  number = {38},
  location = {Rome},
  publisher = {Pontifical Biblical Institute},
  date = {1965}
}

@article{desilva:1998,
  author = {DeSilva, David A.},
  title = {Honor Discourse and the Rhetorical Strategy of the Apocalypse of John},
  journaltitle = {Journal for the Study of the New Testament},
  shortjournal = {JSNT},
  volume = {71},
  date = {1998},
  pages = {79-110}
}

@book{richards:1965,
  author = {Richards, I. A.},
  title = {The Philosophy of Rhetoric},
  location = {Oxford},
  publisher = {Oxford University Press},
  date = {1965}
}

@book{evans+green:2006,
  author = {Evans, Vyvyan and Green, Melanie},
  title = {Cognitive Linguistics: An Introduction},
  location = {New York},
  publisher = {Routledge},
  date = {2006}
}
\end{verbatim}

\begingroup
\DeclareDocumentCommand{\autocites}{omom}{\footnotemark{}}
\setcounter{footnote}{5}
\begin{verbcite}
  Using the latter term, Jewett claims that believers are “members of the
  realm of Christ” in that “their very being is shaped by Spirit rather than
  flesh,” and Ernst Käsemann states: “Commitment to one or the other power
  makes one a member of a worldwide domain which can be defined by the
  alternatives of righteousness and unrighteousness, Christ and Adam, Spirit
  and flesh, or Spirit and
  law.”\autocites[489]{jewett:2007}[220]{kasemann:1981}
\end{verbcite}
\endgroup
\par
\strut\quad\raisebox{2ex}{\hypertarget{Hfootnote.1}{}}6.
\cites[489]{jewett:2007}[220]{kasemann:1981}.
\par
\begingroup
\DeclareDocumentCommand{\autocite}{om}{\footnotemark{}}
\setcounter{footnote}{2}
\begin{verbcite}
  In the \citetitle{ramsay:1906}, originally published in 1904, classical
  archaeologist and New Testament scholar Sir William Ramsay famously paired
  detailed descriptions of the cities of Revelation, drawing upon ancient
  texts, archaeological resources, and his own experiences in Turkey, with
  discussions of the corresponding messages. This approach is based upon his
  assumption that John “imparts to [the letters] many touches, specially
  suitable to the individual Churches … showing his intimate knowledge of them
  all.”\autocite[39]{ramsay:1906}
\end{verbcite}
\endgroup
\par
\strut\quad\raisebox{2ex}{\hypertarget{Hfootnote.2}{}}3.
\cites[39]{ramsay:1906}.
\begin{fverbcite}{3}
  \autocites[16]{ricoeur:1978}[16]{koller:2004}
\end{fverbcite}
\begin{fverbcite}{97}
  \footnote{\citeauthor{gordon:1953} argued that \emph{brh} in Isa 27:1 should
  be translated “evil,” based on an Arabic cognate
  \mkbibparens{\cite*[243]{gordon:1953}; \cite[see also][376]{gordon:1965}}.}
\end{fverbcite}
\begin{fverbcite}{77}
  This is in line with \citeauthor*{desilva:1998}’s treatment of Rev 2–3 as
  part of the text’s use of honor discourse. \cite[See][]{desilva:1998}.
\end{fverbcite}
\examplecite(52){richards:1965}
\begin{fverbcite}{53}
  \footnote{It is interesting to note that Richards also seems to anticipate
  Lakoff and Johnson’s basic definition of metaphor when he writes that
  metaphor includes “those processes in which we perceive or think of or feel
  about one thing in terms of another” \autocite[116-117]{richards:1965}.}
\end{fverbcite}
\examplecite(54){evans+green:2006}
\begin{fverbcite}{55}
  \footnote{Entailments are “rich inferences” or knowledge (“sometimes quite
  detailed”) that we can infer from conceptual metaphors
  \autocite[298-299]{evans+green:2006}.}
\end{fverbcite}
\exampleabbreviations
\examplebibliography
\examplereferences{https://sblhs2.com/2016/10/11/placement-citations-traditional/}

\section{Citing Films (6 October 2016)}

\begin{verbatim}
@movie{scorsese:aviator,
  author = {Scorsese, Martin},
  authortype = {director},
  title = {The Aviator},
  organization = {Forward Pass},
  date = {2004}
}

@movie{scorsese:aviator:dvd,
  author = {Scorsese, Martin},
  authortype = {dir\adddot},
  title = {The Aviator},
  howpublished = {DVD},
  location = {Burbank, CA},
  organization = {Warner},
  date = {2005}
}

@video{scorsese+schoonmaker+mann:aviator,
  entrysubtype = {invideo},
  author = {Scorsese, Martin and Schoonmaker, Thelma and Mann, Michael},
  title = {Feature Commentary},
  volume = {\autocap{d}isc 2},
  maintitle = {The Aviator},
  howpublished = {DVD},
  editor = {Scorsese, Martin},
  editortype = {director},
  location = {Burbank, CA},
  organization = {Warner},
  date = {2005}
}

@review{ebert:2004,
  author = {Ebert, Roger},
  revdtitle = {The Aviator},
  website = {RogerEbert.com},
  date = {2004},
  url = {http://tinyurl.com/n6onocs}
}

@article{dargis:2004,
  author = {Dargis, Manohla},
  title = {Savoring a Legend before It Curdled},
  journaltitle = {The New York Times},
  date = {2004},
  url = {http://tinyurl.com/j82pukj}
}
\end{verbatim}

\begin{verbcite}
  Ironically, \citetitle*{scorsese:aviator} could be
  \citeauthor{scorsese:aviator}’s ultimate exemplar for understanding how
  Jesus is “nothing less than one of us.”
\end{verbcite}
\begin{verbcite}
  \nocite{scorsese:aviator:dvd, scorsese+schoonmaker+mann:aviator, ebert:2004,
    dargis:2004}
\end{verbcite}
\examplebibliography
\examplereferences{https://sblhs2.com/2016/10/06/citing-films/}

\section{Separating Publication Information (4 October 2016)}

\begin{verbatim}
@book{kaltner+mckenzie:2002,
  editor = {Kaltner, John and McKenzie, Steven L.},
  title = {Beyond Babel: A Handbook for Biblical Hebrew and Related Languages},
  series = {Resources for Biblical Study},
  shortseries = {RBS},
  number = {42},
  location = {Atlanta},
  publisher = {Society of Biblical Literature},
  date = {2002}
}

@book{oates+etal:2001,
  editor = {Oates, John F. and Willis, William H. and Bagnall, Roger S. and Worp,
            Klass A.},
  title = {Checklist of Editions of Greek and Latin Papyri, Ostraca, and Tablets},
  edition = {5},
  series = {Bulletin of the American Society of Papyrologists, Supplements},
  shortseries = {BASPSup},
  number = {9},
  location = {Oakville, CT},
  publisher = {American Society of Papyrologists},
  date = {2001}
}
\end{verbatim}

\examplecite(4)[xii]{kaltner+mckenzie:2002}
\examplecite(5)[10]{oates+etal:2001}
\exampleabbreviations
\examplebibliography
\examplereferences{https://sblhs2.com/2016/10/04/separating-publication-information/}

\section{Citing a Specific Printing (29 September 2016)}

\begin{verbatim}
@collection{aland:2005,
  editor = {Aland, Kurt},
  title = {Synopsis quattuor evangeliorum: Locis parallelis evangeliorum apocryphorum
           et patrum adhibitis},
  edition = {15},
  printing = {4},
  location = {Stuttgart},
  publisher = {Deutsche Bibelgesellschaft},
  date = {2005}
}
\end{verbatim}

\examplecite(1){aland:2005}
\examplebibliography
\examplereferences{https://sblhs2.com/2016/09/29/citing-specific-printing/}

\section{Livy (27 September 2016)}

\begin{verbatim}
@ancienttext{livy:aburbecond,
  author = {Livy},
  title = {Ab urbe condita},
  shorttitle = {Ab urbe cond\adddot},
  translator = {Foster}
}
\end{verbatim}

\examplecite[paren][(2.2)]{livy:aburbecond}
\examplecite[ptrans][(2.2)]{livy:aburbecond}
\exampleabbreviations
\examplereferences{https://sblhs2.com/2016/09/27/livy/}

\section{Program Units, Meetings, and Fields of Study (20 September 2016)}

\begin{verbatim}
@unpublished{niditch:1994,
  author = {Niditch, Susan},
  title = {Oral Culture, and Written Documents},
  shorttitle = {Oral Culture},
  type = {paper},
  eventtitle = {the Annual Meeting of the New England Region of the Society of Biblical Literature},
  venue = {Worcester, MA},
  eventdate = {1994-03-25}
}
\end{verbatim}

\examplecite(31)[13-17]{niditch:1994}
\examplebibliography
\examplereferences{https://sblhs2.com/2016/09/20/program-units-meetings-fields/}

\section{Brown Judaic Studies (13 September 2016)}

\begin{verbatim}
@collection{johnsonhodge:2013,
  editor = {Johnson Hodge, Caroline and Olyan, Saul M. and Ullucci, Daniel and Wasserman, Emma},
  title = {\mkbibquote{The One Who Sows Bountifully}: Essays in Honor of Stanley K. Stowers},
  series = {Brown Judaic Studies},
  shortseries = {BJS},
  number = {356},
  location = {Providence, RI},
  publisher = {Brown Judaic Studies},
  date = {2013}
}
\end{verbatim}

\begin{verbcite}
  \nocite{johnsonhodge:2013}
\end{verbcite}
\exampleabbreviations
\examplebibliography
\examplereferences{https://sblhs2.com/2016/09/13/brown-judaic-studies/}
\end{document}
